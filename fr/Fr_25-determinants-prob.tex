\section*{Exercices}
\addcontentsline{toc}{section}{Exercices}


\medskip {\bf Remarques:} 
\begin{enumerate}
\item Une question marquée d'un astérisque $ ^\ast$ (ou deux !) indique une question de niveau bonus.
 \item Vous devez justifier toutes vos réponses.
\end{enumerate}
\bigskip



\begin{prob} \label{prob21.1} Calculez le déterminant des matrices suivantes. Ici $\lambda$ est un paramètre réel variable. Utilisez les opérations sur les lignes et/ou colonnes appropriées lorsque cela est utile; n'utilisez la formule de Laplace qu'en dernier recours !
\medskip
\begin{enumerate}[a)]
\item 
$\bmatrix
2&-1\\3&0 \endbmatrix $.
\medskip


\item\sov~$\bmatrix 
2&-1&3\\3&0&-5\\1&1&2 \endbmatrix $.
\medskip
% 30
\item 
$\bmatrix
1&1&1\\1&2&3\\0&1&1 \endbmatrix $.
\medskip
% -1
\item\sov~$\bmatrix 3&4&-
1\\1&0&3\\2&5&-4\endbmatrix$.\medskip
% -10
\item 
$\bmatrix 0&1&0&0\\
2&0&0&0\\
0&0&3&0\\
0&0&0&4\\
\endbmatrix$.
\medskip
% -24
\item\sov~$\bmatrix \lambda-6&0&0\\0&\lambda&3\\0&4&\lambda+4\endbmatrix $.\medskip
%-(\lambda -6)(\lambda +6)(\lambda +2)
 \item $\bmatrix 2&0&0&0&0&0\cr0&1&0&0&0&0\cr0&0&0&1&0&0\cr
0&0&0&0&0&3\cr0&0&1&0&0&0\cr0&0&0&0&3&0\endbmatrix$.
\medskip

\item\sov~$\bmatrix
-\lambda&2&2\\ 2&-\lambda&2\\ 2&2&-\lambda \endbmatrix$.
\medskip 


\end{enumerate}

\end{prob} \begin{prob} \label{prob21.2} Supposons que $\left|\begin{matrix} a&b&c\cr d&e&f\cr g&h&i\end{matrix} \right|=3$. Alors, 
trouvez les valeurs suivantes :
\medskip
\begin{enumerate}[a)]
 
\item $\left|\begin{matrix}  d&e&f \\ a&b&c\\ g&h&i\end{matrix}\right|$.
\medskip
%-3
\item\sov~$\left|\begin{matrix}  b&a&c\\ e&d&f\\ h&g&i\end{matrix}\right|$.
\medskip
%-3
\item $\left|\begin{matrix}  b&3a&c\\ e&3d&f\\ h&3g&i\end{matrix}\right|$.
\medskip
%-9
\item\sov~$\left|\begin{matrix}  b&3a&c-4b\\ e&3d&f-4e\\ h&3g&i-4h\end{matrix}\right|$.
\medskip
%-9
\item $\left|\begin{matrix} 4g&a&d-2a\cr4h&b&e-2b\cr4i&c&f-2c \end{matrix}\right|$.
%12
\medskip

\end{enumerate}

\end{prob} \begin{prob} 
\label{prob21.3} 

 

\begin{enumerate}[a)]


\item Si $Q$ est une matrice $3\times 3$ telle que $\det(Q)=2$, déterminez $\det((3Q)^{-1})$.
\medskip
% $\frac{1}{54}$
\item\sov~Si $B$ est une matrice $4\times 4$ telle que $\det(2BB^T)=64$, déterminez $|\det(3B^2B^T)|$.
\medskip

\item Si $A$ et $B$ sont deux matrices $4\times 4$ telles que $\det (A)=2$
et $\det(B)=-1$, déterminez $\det(3AB^TA^{-2}BA^TB^{-1})$.
\smallskip

\item\sov~Calculez le d\'eterminant de la matrice égale au produit suivant:
$\scriptsize\bmatrix 1&2\cr3&4 \endbmatrix\bmatrix 5&6\cr7&8\endbmatrix 
\bmatrix9&10\cr11&12\endbmatrix\bmatrix 13&14\cr15&16\endbmatrix.$
\smallskip
%16
\item  Trouvez toutes les valeurs de $x\in\R$ pour lesquelles la matrice $\scriptsize\bmatrix 0&x&-
4\cr2&3&-2\cr1&4&1\endbmatrix$ n'est pas inversible.
\medskip
%-5
\end{enumerate}
\end{prob} \begin{prob} \label{prob21.4} Pour chacun des énoncés suivants, indiquez s'il est (toujours) vrai ou s'il est (possiblement) faux.   
\begin{enumerate}[$\bullet$]
\item Si vous dites que l'\'enonc\'e est faux, donnez un contre-exemple.   
\item Si vous dites que l'\'enonc\'e est vrai, donnez une explication claire - en citant un théorème ou en donnant une {\it preuve valide dans tous les cas}. 
\end{enumerate}

\medskip Dans ce qui suit, $A$ et $B$ sont des matrices $n \times n$ (avec $n>1$) et $k$ est un scalaire.

\medskip
\begin{enumerate}[a)]
\item $\det (AB) = \det (A) \, \det (B)$.
\medskip

\item\sov~$\det (A +B) = \det (A) +\det (B)$.
\medskip

\item $\det (k A)= k \det (A)$.
\medskip

\item\sov~$\det (k A)= k^n \det (A)$.
\medskip

\item $\det  (A^T) = \det (A)$.
\medskip

\item\sov~Si $A$ et $B$ sont identiques, sauf pour la première ligne o\`u celle de $A$ est le double de celle de $B$, alors $\det(A)=2 \det(B)$.
\medskip

\item$^{\ast\ast}$ Si $A=\bmatrix \cc_1 +\bb_1 & \cc_2 & \cdots& \cc_n \endbmatrix$ est donn\'ee en blocs de colonnes (ce qui signifie que la première colonne de $A$ est $\cc_1 +\bb_1$ et que $\cc_2, \cc_3, \dots, \cc_n$ sont les colonnes suivantes de $A$), alors $\det(A)=   \det \bmatrix \cc_1  & \cc_2 & \cdots& \cc_n \endbmatrix + \det \bmatrix  \bb_1 & \cc_2 & \cdots& \cc_n \endbmatrix $\,. 
\medskip


\end{enumerate}
\end{prob} \begin{prob} \label{prob21.5}
\medskip
\begin{enumerate}[a)]
\item 
\medskip Si $A$ est une matrice $2\times2$ et que $\vv_1, \vv_2, \vv_3, \vv_4$ sont des vecteurs de $\R^3$ satisfaisant $$\bmatrix \vv_1 &\vv_2 \endbmatrix= A  \bmatrix \vv_3 &\vv_4 \endbmatrix,$$ alors montrez que $\vv_1\times \vv_2 =(\det(A))\, \vv_3\times \vv_4$.
\medskip

\item\sov~Si $\uu, \vv$ et $\ww$ sont des vecteurs de $\R^3$, utilisez les propriétés des déterminants des matrices $3\times 3$ pour montrer que $$ \uu\cdot \vv\times \ww=  \ww\cdot \uu\times \vv= \vv\cdot \ww\times \uu\,.$$


\item Soit $B$ une matrice $1\times n$, soit $D$ une matrice $n\times n$  et soit $a\in \R$ un scalaire. Montrez que $\det \scriptsize\bmatrix a&B\\0&D\endbmatrix= a\,\det(D)$. (La matrice ici est exprimée sous forme de blocs.)
\medskip

\item$^\ast$\footnote{C'est pour ceux d'entre vous qui connaissent les «~preuves par recurrence~».    Sinon recherchez «~recurrence mathématique~» sur internet.} Soit $D$ une matrice $n\times n$ et soit $B$ une matrice de $m\times n$. Montrer que $\det \scriptsize\bmatrix I_m&B\\0&D\endbmatrix= \det(D)$. (La matrice $\scriptsize\bmatrix I_m&0\\0&B\endbmatrix$ ici,  de taille $(m+n)\times (m+n)$, est exprimée sous forme de blocs.)
\medskip


\item$^\ast$\footnote{Utilisez la même technique que dans la question précédente.} Soit $A$ une matrice $m\times m$. Montrer que $\det \scriptsize\bmatrix A&B\\0&I_n\endbmatrix= \det(A)$. 
\medskip
 
\item Soient $A, B$ et $D$ des matrices respectivement de tailles $m\times m$, $m \times n$ et $n \times n$. En remarquant que $ \scriptsize\bmatrix A&B\\0&D\endbmatrix =\bmatrix I_m&0\\0&D\endbmatrix\bmatrix A&B\\0&I_n\endbmatrix$, montrez que $\det \scriptsize\bmatrix A&B\\0&D\endbmatrix= \det(A) \det(D)$. 
\medskip

\item\sov~ Soient $A, B, C$ et $D$ des matrices respectivement de tailles $m\times m$,   $m \times n$, $n \times m$ et $n \times n$. Supposons que $D$ soit inversible. En remarquant que $ \scriptsize\bmatrix A&B\\C&D\endbmatrix \bmatrix I_m&0\\-D^{-1}C&I_n\endbmatrix  =\bmatrix A-BD^{-1}C&B\\0&D\endbmatrix$, montrez que $\det \scriptsize\bmatrix A&B\\C&D\endbmatrix= \det (A-BD^{-1}C) \det(D)$.
\medskip
 
\item Soient $A, B, C$ et $D$ des matrices $n\times n$. On suppose que $D$ soit inversible et que $CD=DC$. Soit $E=\scriptsize\bmatrix A&B\\C&D\endbmatrix$,  une matrice $(2n)\times (2n)$, exprimée ici sous forme de blocs. Utilisez la question pr\'ec\'edente et les propriétés du déterminant pour montrer que $\det \scriptsize\bmatrix A&B\\C&D\endbmatrix= \det (AD-BC)$.  \medskip

\item$^{\ast\ast\ast}$ \footnote{Ceci s'adresse à ceux d'entre vous qui connaissent la {\it continuité} et le fait que les matrices inversibles $n\times n$ sont {\it denses} dans l'espace des matrices $n\times n$. Recherchez \og\ Matrice inversible\ \fg sur internet. Il existe une autre preuve de cette identité qui n'utilise pas l'argument de la densité, donnée par J.R. Silvester dans {\it Determinants of block matrices}, Math. Gaz., 84(501) (2000), pp. 460-467.} Supposons que $A, B, C$ et $D$ sont des matrices $n\times n$ telles que $CD=DC$. Montrez que $\det \scriptsize\bmatrix A&B\\C&D\endbmatrix= \det (AD-BC)$.  \medskip

\item$^{\ast\ast}$ Soit $A$ une matrice $3 \times 3$ qui satisfait $AA^T=I_3$. \medskip

\begin{enumerate}[i)]
\item Montrez qu'on a aussi $A^TA=I_3$.
\medskip

\item Notons $\set{\ee_1, \ee_2, \ee_3}$ la base ordonnée standard (orthonorm\'ee) de $\R^3$. En utilisant le fait que le produit scalaire $\vv\cdot \ww$ est égal au produit matriciel $\vv^T \ww$ (en écrivant les vecteurs comme des matrices $3 \times 1 $), montrez que $\set{A\ee_1, A\ee_2, A\ee_3}$ est aussi un ensemble orthogonal, qui est même {\it orthonorm\'e}.
\medskip

\item Si $\uu$ et $\vv$ sont deux vecteurs quelconques de $\R^3$, utilisez le Théorème \ref{expansion} pour montrer que $$\uu \times \vv=  \displaystyle \sum_{i=1}^3 (\uu \times \vv)\cdot \ee_i$$ et $$A\uu \times A\vv= \dsize\sum_{i=1}^3 (A\uu \times A\vv)\cdot A\ee_i\,.$$
\medskip

\item En rappelant la définition du déterminant d'une matrice $3\times 3$, montrez que $$(\uu \times \vv)\cdot \ee_i=\det\bmatrix \uu&\vv&\ee_i\endbmatrix$$ et   $$(A\uu \times A\vv)\cdot A\ee_i=\det\bmatrix A\uu&A\vv&A\ee_i\endbmatrix,$$ o\`u les matrices $\bmatrix \uu&\vv&\ee_i\endbmatrix$ et $ \bmatrix A\uu&A\vv&A\ee_i\endbmatrix$ à droite sont données sous forme de blocs en colonnes.
\medskip

\item En utilisant ce que vous savez sur la multiplication par blocs et les propriétés des déterminants (cf les exercices ci-dessus), montrez que $$\det\bmatrix A\uu&A\vv&A\ee_i\endbmatrix=\det(A)\det \bmatrix  \uu& \vv& \ee_i\endbmatrix.$$
\medskip

\item Prouvez maintenant que pour tous vecteurs $\uu, \vv \in \R^3$, (quand $AA^T=I_3$) $$A\uu \times A\vv =\det(A) \, (\uu\times \vv)\,.$$
\medskip

\item Donnez un exemple de matrice $A$ qui ne satisfait pas $AA^T=I_3$ et de deux vecteurs $\uu,\vv \in \R^3$ pour lesquels $A\uu \times A\vv \not=\det(A) \, (\uu\times \vv)$.
\medskip


\end{enumerate}

\end{enumerate}
\end{prob}

