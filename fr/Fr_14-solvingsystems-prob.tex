\section*{Exercices}
\addcontentsline{toc}{section}{Exercices}




 \begin{prob} \label{prob12.1} Trouvez la MER des matrices suivantes:
\medskip
\begin{enumerate}[a)]
\item $\bmatrix 1&0&0&3\cr 0&0&1&5\cr 0&1&0&4 \endbmatrix $.
\medskip 
 
\item\sov~
$\bmatrix 1&0&1&2\cr 0&1&1&2\endbmatrix $.
\medskip
 
\item $\bmatrix 0&2&3\\1&0&1\\ 0&1&0 \endbmatrix $.
\medskip
 
\item\sov~$\bmatrix 1&2&-1&-1\\2&4&-1&3\\ -3&-6&1&-7\endbmatrix$.
\medskip
 
\item $\bmatrix 1& 1& 1& 1\\ 0& 0& 1& -1\\ 1& 1& 2& 0\\ 1& 1& 0& 2 \endbmatrix$.\medskip
 
\item\sov~$\bmatrix 1 & -1 & 1 & 0 \\
 0 & 1 & 1 & 1 \\
 1 & 2 & 4 & 3 \\
 1 & 0 & 2 & 1\endbmatrix$.\medskip
 
\item $\bmatrix 1&0&1&3\\ 1&2&-1&-1\\ 0&1&-2&5\endbmatrix $.\medskip
 
\end{enumerate}

\end{prob} \begin{prob} \label{prob12.2} Trouvez la solution générale pour chacun des systèmes linéaires dont les matrices augmentées sont les suivantes:
\medskip
\begin{enumerate}[a)]

\item $\bmatrix  0 & 1 &|& 0 \\
 0 & 0 &|& 0 \endbmatrix$.
\medskip
  
\item\sov~$ \bmatrix 1 & 0 & -1 &|&0 \\
 0 & 1 & 2 &|&0\\
 0 & 0 & 0&|&1 \\
 0 & 0 & 0 &|&0\endbmatrix$.
\medskip
 
\item $\bmatrix  1 & 0 & 0 & 3 &|& 3 \\
 0 & 1 & 0 & 2 &|& -5 \\
 0 & 0 & 1 & 1 &|& -1 \\
 0 & 0 & 0 & 0 &|& 0\endbmatrix$.
\medskip
 
\item\sov~$\bmatrix  1 & 2 & 0 & 3 & 0 &|& 7 \\
 0 & 0 & 1 & 0 & 0 &|& 1 \\
 0 & 0 & 0 & 0 & 1 &|& 2 \endbmatrix$.
\medskip
 

\item $ \bmatrix 
1& 0& 0& -1& 0&|& 10\\  
0& 1& 0& 1&1&|& 1\\ 
0& 0& 1& 1& 1&|& 4\\ 
0& 0& 0& 0& 0&|& 0\endbmatrix$.
\medskip
 
\end{enumerate}

\end{prob} 

\begin{prob} \label{prob12.3}
Si $A$ et $C$ sont des matrices de même taille, expliquez pourquoi les propositions suivantes sont vraies :

\begin{enumerate}[a)]
\item Réflexivité : $A \sim A$.
\item Symmétrie : si $A \sim B$, alors $B \sim A$.
\item Transitivité : si $A \sim B$ et $B \sim C$, alors $A \sim C$.
\end{enumerate}
Ces trois relations font de la relation «~$\sim$~» ce qu'on appelle une \emph{relation d'équivalence} (regardez sur Wikipédia !).
\end{prob}  