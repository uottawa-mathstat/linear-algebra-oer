
\begin{sol}{prob01.1} Écrire les nombres complexes suivants sous forme cartésienne : $a + b \,i$ avec $a,\, b \in \R$.
\medskip
\begin{enumerate}[a)]
\item    $(2+i)(2+2 i)=2+6i$. \medskip  
\item $ \dfrac 1{1+i}=\dfrac{1}{2}- \dfrac{1}{2} i$.\medskip 
\item $  \dfrac{8+3i}{5-3i}=\dfrac{31}{34}-\dfrac{39}{34}i$.\medskip 
\item $\dfrac{5+5 \,i}{1-i}= 5 i$.\medskip
% 
\item $\dfrac{(1+2i)(2+5i)}{3+4i}=\dfrac{ 12}{25} +\dfrac{59}{25}i  $.\smallskip

\item $\dfrac{1-i}{2-i}+\dfrac{2+i}{1-i}= \dfrac{11}{10}+\dfrac{13 }{10}i$.\smallskip
% 
\item $\dfrac 1{(1-i)(3-2i)}=\dfrac{1}{26}+\dfrac{5 }{26}i$.
%
\end{enumerate}
\medskip

\end{sol} 

\bigskip
\begin{sol}{prob01.2} Trouver la forme polaire des nombres complexes suivants : (c'est-à-dire soit sous la forme $r e^{i\theta}$, soit sous la forme $r(\cos \theta + i \sin \theta)$, avec $r\ge 0$ et  $-\pi <\theta \le \pi$)\medskip
\begin{enumerate}[a)]

\item ${3\sqrt{3}-3i}=6(\cos (-\pi /6)+i \sin (-\pi /6))=6\, e^{-\frac{\pi}{6}i} $.\smallskip
\item $\dfrac{3\sqrt{3}-3i} {\sqrt{2}+i\sqrt{2}}=3(\cos (-5\pi /12)+i \sin (-5\pi /12)) =3\, e^{-\frac{5\pi}{12}i} $.\smallskip

\item $\dfrac{1-\sqrt{3}\,i}{-1+i}=\sqrt{2}(\cos (11\pi/12)+i\sin (11\pi /12))=\sqrt{2}\, e^{\frac{11\pi}{12}i} $.\smallskip
\item $\dfrac{5+5\sqrt{3}\,i}{\sqrt{2}-\sqrt{2}\,i}=5(\cos (7\pi /12)+i\sin (7\pi/12))= 5\,e^{\frac{7\pi}{12}i}$.\smallskip
\item $\dfrac{3+3\sqrt{3}\,i} {-2+2i}=\dfrac{3\sqrt{2}}{2}(\cos (5\pi /12)-i\sin (5\pi/12))=\dfrac{3\sqrt{2}}{2}\,e^{\frac{5\pi}{12}i}$.
\end{enumerate}
\medskip

\end{sol} 


 
\bigskip
\begin{sol}{prob01.4}   Si $z$ est un nombre complexe,
\medskip

(i) Est-il possible que $z={\bar z}$ ? 

\soln Oui, dès lors que $z\in \R$. En effet, si l'on écrit $z=a + b\, i$, alors $z={\bar z}$ ssi $a + b\, i = a - b\, i $, ssi $b=-b$ ssi $z=a\in \R$.

\medskip
(ii) Est-il possible que $|{\bar z}|>|z|$ ? 

\soln Non, car on a toujours $|{\bar z}|=|z|$. En effet,  si $z=a + b\, i$, alors $$|z|=\sqrt{a^2+b^2}=\sqrt{a^2+(-b)^2} =|{\bar z}|\,.$$

\medskip
(iii) Est-il possible que ${\bar z}=2z$ ? 

\soln Oui, mais seulement si $z=0$. En effet, si ${\bar z}=2z$, alors $|z|=2 |z|$ et donc $|z|=0$. D'o\`u $z=0$ est la seule possibilité.

\medskip
\end{sol}