\chapter{Sous-espaces, ensembles g\'en\'erateurs}\label{chapter:Fr_05-subspaces}

Dans le chapitre pr\'ec\'edent, nous avons introduit la notion d'\emph{espace
vectoriel}: c'est un ensemble $V$ muni de deux
lois (addition et multiplication par scalaire) satisfaisant 10 axiomes. 
La moralité est que, algébriquement, un espace vectoriel n'est au final pas
si diff\'erent que ça de $\R^n$ (du moins, ils ont des comportements similaires avec l'addition
et avec la multiplication par scalaire).

Nous avons aussi présenté des exemples d'espaces vectoriels particuliers:

\begin{itemize}
\item
  $\R^n$ pour $n\geq 1$;
\item
  $\mathcal{E}$, l'espace des équations linéaires en 3 variables;
\item
  $\F([a,b])$, l'espace des fonctions d\'efinies sur l'intervalle $[a,b]$;
\item
  $\F(\R)$, l'espace des fonctions d\'efinies sur la droite réelle;
\item
  $\M_{m\times n}(\R)$, l'espace des matrices $m\times n$ avec entr\'ees réelles,
\end{itemize}

\noindent pour lesquels nous avons dû vérifier les 10 axiomes. Mais nous avons
remarqué que parfois axiomes d'arithm\'etique (5)-(10) sont « automatiquement » vérifiés, sous prétexte que nous avons un sous-ensemble d'un espace vectoriel pour les mêmes lois.

Examinons cela plus attentivement.

\section{Sous-espaces d'un espace vectoriel}\label{subsets-of-vectors-spaces}

Dans ce qui suit, considérons $V$ un espace vectoriel et $W \subseteq V$
un \emph{sous-ensemble} de $V$.

\begin{myexample}
Soit $V = \R^2$, qui est bien un espace vectoriel, et
soit $W = \{(x,2x) | x\in \R\}$, qui est bien un sous-ensemble de $V$. Avec les \emph{mêmes} opérations
d'addition et de multiplication par scalaire que celles sur  $\R^2$, que faudrait-il  vraiment savoir pour décider si  $W$ est
un espace vectoriel? 

{\textbf{Fermeture:}} (1) Fermeture pour l'addition: si $(x,2x)$ et
$(y,2y)$ sont dans $W$, alors $(x,2x) + (y,2y) = (x+y, 2(x+y))$ qui
est dans $W$ car c'est de la forme $(z,2z)\in W$ avec $z=x+y$.
Le premier axiome est donc vrai.

(2) Fermeture pour la multiplication par scalaire: si $(x,2x) \in W$ et
$c\in \R$, alors $c(x,2x) = (cx,2(cx))\in W$, ce qui montre que ce crit\`ere est aussi v\'erifi\'e.

{\textbf{Existence:}}~(3) Le vecteur nul est $\zero=(0,0)$, qui est bien dans
$W$. Cependant nous n'avons pas besoin de vérifier que
$\mathbf 0 + \uu = \uu$ pour tout $\uu \in W$, car nous
savons déjà que c'est vrai pour tout $\uu \in V$
(en effet $V$ est un espace vectoriel donc satisfait l'axiome (3)).
Donc il suffit de v\'erifier que $\mathbf 0 \in W$ !

(4) Si $\uu \in W$, on a que
$-\uu = (-1)\uu$ appartient à $W$ grâce au point (2). Remarquez que cette égalité découle du fait que $V$ est un espace vectoriel.

{\textbf{Crit\`eres d'arithm\'etique :}}~(5) - (10) Tous ceux-ci sont
vrais pour tous les vecteurs de $V$, donc ils le sont aussi en
particulier pour tous les vecteurs du sous-ensemble
$W \subset V$. Nous n'avons donc pas besoin de les vérifier à nouveau non plus !

En conclusion, on obtient assez rapidement que $W$ est un espace vectoriel !
\end{myexample}

\begin{definition}
Un sous-ensemble $W$ d'un espace vectoriel $V$ est appelé un \emph{sous-espace (vectoriel) de $V$} si $W$ est un espace vectoriel muni des même lois d'addition et de multiplication que scalaire sur $V$.
\end{definition}


\begin{myexample}
$W = \{ (x,2x) | x\in \R\}$ est un sous-espace de $\R^2$.

\end{myexample}


\begin{theorem}[Test du sous-espace] 
Soit $V$ un espace vectoriel et soit $W \subseteq V$ un sous-ensemble. Alors $W$ est
un sous-espace de $V$ si et seulement si les 3 conditions suivantes sont v\'erifi\'ees:
\begin{enumerate}
\item
  $\mathbf 0 \in W$;
\item $W$ est fermé pour l'addition; c'est-\`a-dire, pour tous $\uu, \vv \in W$, $\uu+\vv\in W$;
\item $W$ est fermé pour la multiplication par scalaire; c'est-\`a-dire, pour tous
  $\uu \in W$ et $r\in \R$, $r\uu\in W$. 
\end{enumerate}
\end{theorem}

Considérez ce théorème comme un raccourci pour vérifier si
un ensemble donné est un espace vectoriel ou non. Nous avons vu dans
l'exemple ci-dessus pourquoi cela se résume à ces trois
axiomes.\footnote{Vous vous demandez peut-être pourquoi nous devons
  inclure le premier point, surtout que «~$0 \uu = \mathbf 0$~» d\'ecoule de la fermeture pour la multiplication par scalaire.
  On peut en fait ignorer le point 1, mais seulement si l'ensemble $W$ n'est pas vide.
  Autrement dit, si $W$ n'est pas vide, il suffit en fait de vérifier les
  points 2 et 3. MAIS de manière générale, le point 1 est très facile à
  vérifier, et lorsque ce point 1 est faux, on peut conclure directement que $W$ n'est pas un
  sous-espace. C'est donc un moyen assez rapide d'exclure certains ensembles
  et nous l'utiliserons assez régulièrement.}

\section{Exemples} \label{many-examples}

Dans cette section, nous appliquons le test du sous-espace (cf. théorème précédent) pour donner davantage d'exemples d'espaces vectoriels. 
Les exercices suggérés incluent de nombreux exemples de sous-ensembles, dont certains pour lesquels on ne peut pas appliquer le test du sous-espace (car les
opérations ne sont pas les mêmes) ou pour lesquels le test du sous-espace échoue.
Il est recommand\'e de résoudre le maximum d'exercices pour développer une intuition sur ce qu'est vraiment un sous-espace (et sur ce que n'est pas un sous-espace) !

\begin{myprob}
D\'eterminez si l'ensemble $T = \{ \uu \in \R^3 | \uu \cdot (1,2,3) = 0\}$, muni des
opérations standards sur $\R^3$,  est un sous-espace de $\R^3$.

\begin{mysol}
Notez d'abord que $T \subset \R^3$ et que $T \neq \R^3$. En fait, $T$est le
plan d'équation \[x + 2y + 3z= 0\,,\] puisque $(1,2,3)$ est un vecteur normal à $T$.
Comme nous utilisons les opérations standards sur $\R^3$, on peut appliquer le test du
sous-espace:

\begin{enumerate}
\item
  Est-ce que $\mathbf 0 \in T$? Oui, puisque $\mathbf 0 = (0,0,0)$
  satisfait la condition $\uu \cdot (1,2,3) = 0$.
\item
  $T$ est-il fermé pour l'addition?\\
  Oui. En effet, pour tout $\uu,\ \vv\in T$, on a $\uu \cdot (1,2,3) = 0$ et $\vv \cdot (1,2,3) = 0$, et donc
  $$(\uu + \vv) \cdot (1,2,3) = (\uu \cdot (1,2,3)) + (\vv \cdot (1,2,3)) = 0 + 0 = 0\,.$$
  Il suit que $\uu + \vv \in T$ et donc que $T$ est fermé pour
  l'addition.
\item
  $T$ est-il fermé pour la multiplication par scalaire?\\
 Oui. En effet, soient $k \in \R$ et $\uu \in T$. Par définition on a $\uu \cdot (1,2,3) = 0$).\\
  Pour voir si $k\uu \in T$ on regarde si $(k\uu)\cdot (1,2,3) = 0$.\\
  On a bien
  $(k\uu)\cdot (1,2,3) = k(\uu \cdot (1,2,3)) = k(0) = 0$.\\
  Donc $k\uu \in T$ et $T$ est fermé pour la multiplication
 par scalaire.
\end{enumerate}

D'o\`u,  $T$ est un sous-espace de $\R^3$.

\end{mysol}

\end{myprob}

Notez que nous aurions pu en fait remplacer le vecteur $(1,2,3)$ par n'importe
quel autre vecteur $\nn$ et le résultat aurait été le même. Plus généralement, on a le r\'esultat suivant:

\standout{ Tout plan dans $\R^3$  passant par l'origine est un sous-espace de $\R^3$.}   

En fait, si un plan \emph{ne passe pas} par l'origine, alors il ne satisfait pas la première condition du test du sous-espace test. Ceci nous amène à dire que :

\standout{ Tout plan dans $\R^3$  qui NE PASSE PAS par l'origine N'EST PAS un sous-espace de $\R^3$.}   

Remarque : on appelle d'ailleurs \stress{plan affine} un plan qui ne passe pas par $\zero$.


\begin{myprob}
Pour $\vv \in \R^n$ on définit $L = \{ t\vv \,|\, t\in \R\}$ comme
étant la droite de $\R^n$ passant par l'origine et ayant pour vecteur
directeur le vecteur $\vv$. On munit l'ensemble $L$ des opérations standards sur $\R^n$. Est-il alors un sous-espace de $\R^n$?

\begin{mysol}
Comme $T$ est muni des mêmes opérations que $\R^n$, on peut appliquer le test du sous-espace :

\begin{enumerate}
\item
    On prenant $t=0$ on observe que $\mathbf 0 \in L$.
\item
  Si $t\vv$ et $s\vv$ sont deux vecteurs de $L$, alors
  $t\vv + s\vv = (t+s)\vv$ qui est encore un multiple de
  $\vv$ et donc appartient \`a $L$. Donc $L$ est fermé pour l'addition.
\item
  Si $t\vv \in L$ et $k\in\R$, alors $k(t\vv) = (kt)\vv$, qui est aussi
  un multiple de $\vv$ et est donc un vecteur de $L$. Ainsi $L$ est
  fermé pour la multiplication par scalaire.
\end{enumerate}

En conclusion $L$ est bien un sous-espace de $\R^n$.
\end{mysol}
\end{myprob}


De même que ci-dessus, nous avons plus généralement que:

\standout{ Toute droite de $\R^n$  passant par l'origine est un sous-espace de $\R^n$. Réciproquement, toute droite de $\R^n$  qui NE PASSE PAS par l'origine N'EST PAS un sous-espace de $\R^n$.}   



\begin{myprob}
Soit $V = \F(\R)$ l'espace vectoriel de toutes les fonctions d\'efinies sur $\R$ tout entier et à valeurs dans $\R$, et soit $\mathbb{P}$ l'ensemble de tous les \emph{fonctions
polynomiales} qui peuvent être écrites comme :
\[p(x) = a_0 + a_1 x + a_2 x^2 + \cdots + a_n x^n\] pour certains
$n \geq 0$ et $a_i \in \R$. L'ensemble $\PP$ est-il un espace vectoriel?

\begin{mysol}
Nous notons d'abord (cela n'est pas forcément évident) que les
opérations habituelles d'addition de deux polynômes et de multiplication d'un polynôme
par un scalaire sont exactement les mêmes que les opérations standards sur l'espace  $\F(\R)$ des 
fonctions. Donc on a bien l'inclusion $\PP \subset \F(\R)$ avec les mêmes
opérations dans les deux ensembles, et on peut appliquer le test du sous-espace :

\begin{enumerate}
\item
  Est-ce que $\mathbf 0 \in \PP$? Rappelez-vous que $\mathbf 0$ est la
  fonction nulle. En fait, c'est aussi le polynôme nul
  $p(x) = 0 + 0x + 0x^2$ (ou simplement $p(x) = 0$.)
  Donc c'est un polynôme et alors $\mathbf 0 \in \PP$.
\item
  L'ensemble $\PP$ est-il fermé pour l'addition? Oui, car la somme de deux
  polynômes est à nouveau un polynôme (on ne peut pas obtenir une fonction quelconque comme l'exponentielle par exemple).
\item
  L'ensemble $\PP$ est-il fermé pour la multiplication par scalaire? Oui, car multiplier un
  polynôme par un scalaire donne encore un polynôme.
\end{enumerate}

D'o\`u $\PP$ est un sous-espace de $\F(\R)$, et est en particulier un espace vectoriel.

\end{mysol}
\end{myprob}

\begin{definition}\label{transpose}  
La \defn{transposée} d'une matrice
 $A$ de taille $m\times n$ est la matrice de taille $n \times m$, not\'ee $A^T$, dont les
lignes sont exactement les colonnes de $A$. Par example,

\[\mat{a&b\\c&d}^T = \mat{a&c\\b&d}.\]
\end{definition}


\begin{myprob}
Soit $S = \{ A \in \M_{22}(\R) | A^T = A\}$ l'ensemble
des matrices \emph{symétriques} $2 \times 2$ (avec les opérations
standards sur $\M_{mn}(\R)$). Est-ce que $S$ est un espace vectoriel ?

\begin{mysol}
Comme $S \subset \M_{22}(\R)$ et qu'il est muni des même opérations que $\M_{22}(\R)$, 
nous pouvons appliquer le test du sous-espace.

Toutefois, il est utile de se familiariser avec cet ensemble $S$.
Réécrivez-le : \[\begin{aligned}
S &= \left\{ \mat{a&b\\c&d} \Big|  \mat{a&b\\c&d}^T = \mat{a&b\\c&d}\right\}\\
&= \left\{ \mat{a&b\\c&d} \Big|  \mat{a&c\\b&d} = \mat{a&b\\c&d}\right\}\\
&= \left\{ \mat{a&b\\b&d} \Big| a,b,d \in \R\right\}\,.\end{aligned}\] 
Tout à coup, on y voit plus clair !

\begin{enumerate}
\item
  En prenant $a=0$, $b=0$ et $d=0$ on voit que la matrice nulle $\zero$ appartient bien \`a $S$.
\item
  On prend deux matrices arbitraires dans $S$ et on les additionne comme suit
  \[\mat{a&b\\b&d} + \mat{a'&b' \\ b'&d'} = \mat{a+a' & b+b' \\ b+b' & d+d'}\,.\]
  C'est à nouveau dans $S$, car c'est bien de la forme typique d'un vecteur de $S$ (c'est-à-dire les entrées (1,2) et (2,1) de la matrice sont bien
  égales). Donc $S$ est fermé pour l'addition.
\item
  Soit $k\in \R$. Alors
  \[k\mat{a&b\\b&d} = \mat{ka & kb \\ kb& kd}\] est à nouveau un vecteur de
  $S$, et donc $S$ est fermé pour multiplication par scalaire.
\end{enumerate}

En conclusion, $S$ est un sous-espace de $\M_{22}(\R)$, et $S$ est donc bien un espace vectoriel.

\end{mysol}
\end{myprob}


\section*{Exercices}
\addcontentsline{toc}{section}{Exercices}
