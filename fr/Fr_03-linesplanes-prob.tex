
\section*{Exercices}
\addcontentsline{toc}{section}{Exercices}


    
 
\begin{prob}\label{prob03.1}  Répondez aux questions suivantes en n'utilisant que le produit vectoriel et/ou le produit scalaire.
\begin{enumerate}[a)]
\item Pour $\uu=(3,\ -1,\ 4)$ et $\vv=(-1,\ 6,\ -5)$, calculez
$\uu\times \vv$. \medskip
% $(-19,\ 11,\ 17)$
\item\sov~Trouvez tous les vecteurs de $\R^3$ qui sont orthogonaux à la fois à $(-1, 1, 5)$ et à $(2, 1, 2)$.  \medskip
% $\{(t,\ -4t,\ t)|\ t\ dans \R\}$.
\item Si  $\uu=(4,\ -1,\ 7),\ \vv=(2,\ 1,\ 2)$ et 
$\ww=(-1,\ -2,\ 3)$, déterminez $(\uu\times \vv)\times \ww$. \medskip
% $(30,\ 21,\ 24)$
\item\sov~Si $\uu=(-4,\ 2,\ 7),\ \vv=(2,\ 1,\ 2)$ et 
$\ww=(1,\ 2,\ 3)$, déterminez $\uu\cdot (\vv\times \ww)$. \medskip
% 17
\end{enumerate}

\end{prob}
\begin{prob}
\label{prob03.2}  Même consigne qu'à l'exercice précédent. \medskip
\begin{enumerate}[a)]

\item Trouvez l'aire du parallélogramme défini par les vecteurs
$\uu=(1,\ -1,\ 0)$ et $\vv=(2,\ -3,\ 1)$.    \medskip
\item\sov~Trouvez l'aire du triangle dont les trois sommets sont $ A=(-1,\ 5,\ 0) $, $B=(1,\ 0,\ 4)$ et $C=(1,\ 4,\ 0)$.  \medskip
%6
\item Trouvez l'aire du triangle dont les trois sommets sont $P=(1,\ 1,\ -1),\ Q=(2,\ 0,\ 1)$ et\\ $R=(1,\ -1,\ 3)$.  \medskip
%\sqrt{5}


\item\sov~Trouvez le volume du parallélépipède formé par les vecteurs $\uu=(1,\ 1,\ 0),\ \vv=(1,\ 0,\ -1)$\\ et $\ww=(1,\ 1,\ 1)$.\medskip
% 1

\item Trouvez le volume du parallélépipède formé par $\uu=(1, -2, 3),\ \vv=(1,3,1)$ et $\ww=(2,1,2)$.\medskip
% 10

\end{enumerate}



\end{prob} \begin{prob} \label{prob03.3}  Résolvez les problèmes suivants dans $\R^3$. \medskip
\begin{enumerate}[a)]
\item\sov~Trouvez le point d'intersection du plan dont l'équation cartésienne est $2x+2y-z=5$ 
et de la droite d'équations paramétriques $x=4-t,\ y=13-6t,\ z=-7+4t$.\medskip %$(2,\ 1,\ 1)$
\item\sov~Si $\mathcal L$ est la droite passant par $(1,\ 1,\ 0)$ et $(2,\ 3,\ 1)$, trouvez le point d'intersection de $\mathcal L$ avec le plan d'équation cartésienne $x+y-z=1$.\medskip % $(0,\ 1/2,\ -1/2)$
\item Trouvez le point d'intersection de la droite d'équations paramétriques $x=t-1$, $y=6-t,\\ z=-4+3t$ avec la droite d'équations $x=-3- 4t,\ y=6-2t,\ z=-5+3t$. \medskip%(1,\ 4,\ 2)
\item Déterminez si les plans d'équations cartésiennes $2x-3y+4z=6$ et $4x-6y+8z=11$ se croisent ou non. \medskip% Non
\item Trouvez la droite à l'intersection des plans d'équation cartésienne $5x+7y-4z=8$ et\\ $x-y=-8$. \medskip%$(-4,\ 4,\ 0)+ t (1,\ 1,\ 3),\, t\in \R$
\item\sov~Trouvez la droite à l'intersection des plans d'équations cartésiennes $x+11y-4z=40$ et $x -y=-8$. \medskip
%$(-4,\ 4,\ 0)+ t (1,\ 1,\ 3),\, t\in \R$

\end{enumerate}



\end{prob} \begin{prob} \label{prob03.4}  Résolvez les problèmes suivants dans $\R^3$. \medskip
\begin{enumerate}[a)]

\item Trouvez la distance du point $(0,\ -5,\ 2)$ au plan d'équation cartésienne
$2x+3y+5z=2$. \medskip
%7$/\sqrt{38}$.

\item\sov~Trouvez la distance du point $(-2,\ 5,\ 9)$ au plan d'équation cartésienne
$6x+2y-3z=-8$. \medskip
%3
\item Trouvez la distance entre le point $(5,\ 4,\ 7)$ et la droite passant par les points  $(3,\ -1,\ 2)$ et $(3,\ 1,\ 1)$.
\medskip



\item\sov~Trouvez la distance entre le point $(8,\ 6,\ 11)$ et la droite passant par les points $(0,\ 1,\ 3)$ et $(3,\ 5,\ 4)$.  \medskip
% 7
 
\item Trouvez l'angle entre les plans d'équations cartésiennes $x-z=7$ et $y-z=234$.
\medskip
%$\pi/3




\end{enumerate}

\end{prob} \begin{prob} \label{prob03.5}  Trouvez les équations paramétriques {\it et} la forme vectorielle de chacune des droites suivantes :
\medskip
\begin{enumerate}[a)]

\item La droite passant par les points $(3,\ -1,\ 4)$ et $(-1,\ 5,\ 1)$. \medskip
\item\sov~La droite passant par $(-5,\ 0,\ 1)$ et parallèle aux deux plans d'équations cartésiennes\\ $2x-4y+z=0$ et $x-3y-2z=1$. 
\medskip
%$x=-5+11t,\, y=5t,\, z=1-2t,\, t\in \R$
\item La droite passant par $(1,\ 1,\ -1)$ et perpendiculaire au plan d'équations cartésiennes\\ $2x-y+3z=4$. \medskip
%$x=1+2t,\, y=1-t,\, z=-1+3t,\, t\in \R$
\end{enumerate}

\end{prob} \begin{prob} \label{prob03.6}  Trouvez une équation cartésienne pour chacun des plans suivants :

\medskip
\begin{enumerate}[a)]
\item Le plan contenant $(3, -1, 4)$, $(-1, 5, 1)$ et $(0, 2, -2)$.\medskip% $2y - z = 3
\item\sov~Le plan parallèle au vecteur $(1, 1, -2)$ et contenant les points $(1, 5, 18)$ et $(4, 2, -6)$.\medskip %5x - 3y + z = 8
\item Le plan passant par les points $(2, 1, -1)$ et $(3, 2, 1)$ et parallèle à l'axe $x$. \medskip
\item\sov~Le plan contenant les deux droites 
$\set{(t-1,6-t,-4+3t)\st t \in \R}$ et $ \set{(-3 -4t, 6+ 2t, 7+5t)\st t\in \R}$.\medskip  %11x + 17y + 2z = 83
\item Le plan contenant à la fois le point $(-1, 0, 2)$ et la droite à l'intersection des deux plans $3x + 2y - z = 5$ et $2x + y + 2z = 1$.  \medskip %$23x + 12y + 19z = 15$.
\item\sov~Le plan contenant le point $(1, -1, 2)$ et la droite $\set{(4, -1 + 2t,2 + t)\st t \in \R}$. \medskip% $y - 2z + 5 = 0$.
\item Le plan passant par l'origine et parallèle aux deux vecteurs $(1, 1, -1)$ et $(2, 3, 5)$. \medskip%8x - 7y + z = 0$.
\item\sov~Le plan passant par le point $(1,\ -7,\ 8)$ et perpendiculaire à la droite\\ $\set{(2+2t,7-4t,-3+t) \st t\in \R}.$ \smallskip  
%$2x-4y+z=38$
\item Le plan passant par le point $(2,\ 4,\ 3)$ et perpendiculaire aux plans d'équations cartésiennes $x+2y-z=1$ et $3x-4y=2$.  
%$4x+3y+10z=50$.


\end{enumerate}
 

\end{prob} \begin{prob} \label{prob03.7}   Trouvez la forme vectorielle pour les plans $\mathcal W$ dont les équations cartésiennes sont les suivantes 
(c'est-à-dire, trouvez un point $a \in \mathcal W$ et deux vecteurs $\uu, v \in \R^3$ non-nuls et non-parallèles entre eux mais parallèles au plan $\mathcal W$, tels que $\mathcal W=\set{a+ s u + t v\st s,t \in \R}$) :\medskip
\begin{enumerate}[a)]

\item $x - y - z = 3$.\medskip 
\item\sov~$x - y - 2z = 4$. \medskip
\item $2x - y + z = 5$. \medskip
\item $y + 2x = -3$. \medskip
\item $x - y + 2z = 0$. \medskip
\item $x + y + z = -1$.\medskip
\end{enumerate}


\end{prob} \begin{prob} \label{prob03.8}\sov~Soient $\uu, v$ et $\ww$ des vecteurs quelconques de $\R^3$.  Déterminez lesquels des énoncés suivants pourrait être faux et donnez un exemple pour justifier chacune de vos réponses.
 \medskip

(1) $\uu\cdot \vv=\vv\cdot \uu$.

(2) $\uu\times \vv=\vv\times \uu$.

(3) $\uu\cdot(\vv+\ww)=\vv\cdot \uu+\ww\cdot \uu$.

(4) $(\uu+2\vv)\times \vv=\uu\times \vv$.

(5) $(\uu\times \vv)\times \ww=\uu\times(\vv\times \ww)$.

% 2 \& 5

\end{prob} \begin{prob} \label{prob03.9}\sov~Soient $\uu , \vv $ et $\ww $ des vecteurs de $\R^3$.  Lesquels des énoncés suivants sont (toujours) vrais ? Justifiez vos réponses quand c'est VRAI et donnez un contre-exemple lorsque c'est FAUX.
\medskip

(i) $(\uu\times \vv)\cdot \vv=0$.
 

(ii) $(\vv \times \uu)\cdot \vv=-1$.
 

(iii) $(\uu\times \vv)\cdot \ww$ est le volume du parallélépipède formé par $\uu$, $\vv$ et $\ww$.
 

(iv) $||\uu\times \vv||=||\uu||\,||\vv||\,\cos\theta$,
où $\theta$ est l'angle entre $\uu$ et $\vv$.
 

(v) $|\uu\cdot \vv|=||\uu||\,||\vv||\,\cos\theta$,
où $\theta$ est l'angle entre $\uu$ et $\vv$.



\end{prob} 


\begin{prob} \label{prob03.10} Prouvez l'identité $\uu \times (\vv\times \ww)= (\uu\cdot \ww) \vv -(\uu\cdot \vv) \ww$ pour tous $\uu,\vv,\ww \in \R^3$ comme suit. Dénotez  $D(\uu,\vv,\ww)$ la soustraction entre le côté gauche de l'identité et le côté droit.

Tout d'abord, par les propriétés des produits scalaires et vectoriels, il est facile de montrer les identités suivantes pour tout $k\in \R$ et tout $\uu,\vv,\ww,\uu',\vv',\ww' \in \R^3$ :
 
\medskip
\begin{enumerate}[(i)]
 
\item $D(k \uu +\uu', \vv ,\ww) =k\, D(\uu,\vv,\ww) +D(\uu',\vv,\ww)$,
\medskip

\item  $D(\uu , k \vv +\vv',\ww) =k\,D(\uu,\vv,\ww)  + D(\uu,\vv',\ww)$,
\medskip
\item $D(\uu, \vv,k \ww +\ww') =k\, D(\uu , \vv,\ww)+ D(\uu, \vv,\ww')$,
\medskip
\item $D(\uu, \vv,\ww) =-D(\uu,\ww,\vv) $.
\medskip
\end{enumerate}

(On résume les propriétés (i)--(iii) en disant que  «~{\it $D$ est linéaire en tous les arguments}~». Nous en reparlerons lorsque nous aborderons les transformations linéaires).
\medskip 

Ensuite, puisque chaque vecteur de $\R^3$ est une combinaison linéaire de $\hat i= (1,0,0), \hat j =(0,1,0)$ et $\hat k=(0,0,1)$, il suffit de vérifier qu'on a toujours 
$$D(\uu, \vv,\ww)=0$$ 
lorsque $\uu\in\{\hat i, \hat j, \hat k\}$ et que $(\vv, \ww)$ est une des 6 paires de deux éléments distincts de $\set{\hat i, \hat j, \hat k}$. 
En se limitant à ces $3\times 6= 18$ possibilités, il est facile de voir que $D(\uu,\vv,\ww)$ est égal à zéro à moins que $\uu = \vv$ ou $\uu = \ww$. Enfin, il reste à vérifier $D(\uu,\uu,\ww)=0$ que dans les 6 cas où $\uu\in \set{\hat i, \hat j, \hat k}$ et $\ww \in \set{\hat i, \hat j, \hat k}\setminus\set{\uu}$, ce qui conclut la preuve.  
  
\end{prob} 