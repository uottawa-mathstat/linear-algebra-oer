
\begin{sol}{prob13.1} 

 

(b) Trouvez le rang des matrices coefficients et des matrices augmentées correspondantes aux systèmes linéaires de l'Exercice \ref{prob12.2}.

\soln

 [\ref{prob12.2}(b)] Le rang de la matrice coefficients est 2 et celui de la matrice augmentée est 3. (C'est pourquoi le système linéaire correspondant est incompatible).
\medskip

[\ref{prob12.2}(d)] Le rang des deux matrices est 3.

\medskip 
 
\end{sol}

\bigskip
\begin{sol}{prob13.2} Supposez que $a,c\in \R$ et considérez le système linéaire suivant en les variables
$x,\, y$ et $z$ : 
$$\begin{matrix}  x&+&y &+&az&=&2\\
2x&+&y&+&2a z  &=&3\\ 
3x&+&y &+&3az&=&c\,. \end{matrix} $$

Notez que la solution générale de ce système peut dépendre des
paramètres $a$ et $c$. \\
Soit $[\,A\,|\, \bb\,]$ la matrice augmentée de ce système.

\medskip 
(b)  Trouvez \underbar{toutes} les valeurs de
$a$ et $c$ pour lesquelles ce syst\`eme: \smallskip
\begin{enumerate}[(i)]
 
\item admet une solution unique;\smallskip
\item  admet une infinit\'e de solutions;  \smallskip
\item  n'admet aucune solution. 
\end{enumerate}
\soln  

$$[\,A\,|\, \bb\,] =\bmatrix 1&1&a&|&2\\ 2&1&2a&|&3\\ 3&1&3a&|&c\endbmatrix \sim
\bmatrix 1&1&a&|&2\\ 0&1&0&|&1\\ 0&0&0&|&c-4\endbmatrix\,. $$

Puisque $\rank A=2<3=\#$ variables, pour {\it toutes} les valeurs de $a$ et $c$, ce système n'admet {\it jamais} une solution unique.
\smallskip

Pour $c=4$, on a $\rank A=2=\rank[\,A\,|\, \bb\,]$, donc ce système sera compatible et aura une infinité de solutions.
Notez qu'il y aura exactement $1$ paramètre dans la solution générale (car $\#\text{variables}-\rank A=3-2=1$).
\smallskip

Pour $c\neq 4$, on a $\rank A=2<3=\rank[\,A\,|\, \bb\,]$, donc ce système sera incompatible.
\medskip



\end{sol}


\bigskip
\begin{sol}{prob13.3}  Considérez le réseau routier avec
intersections A, B, C, D et E ci-dessous.   Les flèches
indiquent le sens de la circulation le long des routes, la circulation se faisant à sens unique. Les chiffres indiquent le nombre exact de voitures observées entrant ou sortant des intersections A, B, C, D et E pendant une minute.  Chaque
$x_i$ désigne le nombre inconnu de voitures qui sont passées par les routes correspondantes en une minute.
$$\xymatrix@=3pt{
&&&&&&&&&&&&\\ 
&&&&&&&&&&&&\\
&&&&&&&&&&&&\\
&&&&&&&&B\ar[rrdd]^{x_4}\ar[uuu]_{50}&&&&\\ 
&&&&&&&&&&&&\\
&&&&&&A\ar[lll]^{60}\ar[rruu]^{x_5}&&&&C\ar[lddd]_{x_3}\ar[llll]_{x_6}&&&&\ar[llll]^{70}\\
&&&&&&&&&&&&\\
&&&&&&&&&&&&\\
&&&&&&&E\ar[luuu]^{x_1}&&D\ar[ll]_{x_2}\ar[dddr]^{60}&&\\  
&&&&&&&&&&&&\\ 
&&&&&&&&&&&&\\
&&&&&&\ar[uuur]^{100}&&&&&&&&\\} 
$$

(b)~La MER de la matrice augmentée de la question (a) est la suivante: 
$$ \bmatrix 
  1 & 0 & 0 & 0 & -1 & 1 &|& 60 \\
 0 & 1 & 0 & 0 & -1 & 1 &|& -40 \\
 0 & 0 & 1 & 0 & -1 & 1 &|& 20 \\
 0 & 0 & 0 & 1 & -1 & 0 &|&  -50 \\
 0 & 0 & 0 & 0 & 0 & 0 &|& 0
\endbmatrix\,.$$ Donnez la solution générale. (Ignorez les contraintes à
à ce stade.)
\smallskip


\soln La solution g\'en\'erale est: 
 
 
\begin{align*} 
x_1&=60 +s-t&\\
x_2&=-40 +s-t&\\
x_3&=20 +s-t& s,t \in \R\\
x_4&=-50 +s&\\
x_5&= s &\\
x_6&=t &
\end{align*}
 

ou bien encore $$\set{(60 +s-t,-40 +s-t,20 +s-t,-50 +s,s,t)\st st,t \in \R}.$$

 


\end{sol}

\bigskip
\begin{sol}{prob13.4}  Pour chacun des énoncés suivants, indiquez s'il est (toujours) vrai
ou s'il est (possiblement) faux.    
   \smallskip    
\begin{enumerate}[$\bullet$]
\item Si vous dites que l'affirmation est vraie, vous devez donner une explication claire en r\'ef\'erant \`a un théorème, ou en donnant une {\it preuve valable pour tous les cas}.
\item Si vous dites que l'affirmation est fausse, vous devez donner un contre-exemple explicite.  
\end{enumerate}
\medskip

(b) Tout système non-homogène de 3 équations à 2 inconnues est compatible.


\soln Faux, car par exemple, le syst\`eme dont la matrice augment\'ee est 
$\scriptsize\bmatrix  
1 & 0 &|& 1 \\
0 & 1 &|& 0\\
0 & 0 &|& 1 \endbmatrix$ est incompatible.
\medskip
 

(d) Tout système de 2 équations à 2 inconnues admet une unique solution.

\soln  Faux, car par exemple, le syst\`eme dont la matrice augment\'ee est 
$\scriptsize\bmatrix  
1 & 0 &|& 0 \\
0 & 0 &|& 0\\
 \endbmatrix$ admet une infinit\'e de solutions, \`a savoir $ \{(0,s) \,|\, s\in \R\}$.
\medskip

(f) Si la matrice coefficients d'un système linéaire compatible comporte une colonne de zéros, alors le système admet une infinité de solutions.

\soln Ceci est vrai, car la MER de la matrice augmentée aura toujours une colonne de zéros localis\'ee dans la matrice coefficients. Donc (comme on le suppose compatible), il y aura au moins un paramètre dans la solution générale et d'o\`u une infinit\'e de solutions.
\medskip

(h) Si un système linéaire compatible admet une infinité de solutions, alors il doit y avoir une colonne de zéros dans la MER de la matrice coefficients.

\soln Faux! Pour un contre-exemple, voir question 4 (d).
\medskip
 

(j) Si un système linéaire homogène admet une unique solution, alors ce système comprend le même nombre d'équations que d'inconnues.

\soln Faux, car par exemple, le syst\`eme dont la matrice augment\'ee est  
$\scriptsize\bmatrix  
1 & 0 &|& 0 \\
0 & 1 &|& 0\\
0 & 0 &|& 0 \endbmatrix$ a une solution unique, la solution triviale $(0,0)$, et pourtant il y a $3$ équations et seulement $2$ inconnues.
\medskip
 

\end{sol}

