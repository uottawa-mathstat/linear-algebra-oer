\newpage
\section*{Exercices}
\addcontentsline{toc}{section}{Exercices}




\begin{prob} \label{prob13.1} 

 

\begin{enumerate}[a)]

\item Trouvez le rang de chaque matrice de l'Exercice \ref{prob12.1}.
\medskip
 
\item\sov~Trouver le rang des matrices coefficients ainsi que le rang des matrices augmentées correspondantes aux systèmes linéaires de l'Exercice \ref{prob12.2}.

\medskip 
 

\end{enumerate}
\end{prob} \begin{prob} \label{prob13.2} Soit $a,c\in \R$. Considérez le système linéaire suivant en les variables
$x,\, y,\,z$ : 
$$\begin{matrix}  x&+&y &+&az&=&2\\
2x&+&y&+&2a z  &=&3\\ 
3x&+&y &+&3az&=&c \end{matrix}. $$

Notez que la solution générale de ce système peut dépendre des
valeurs des paramètres $a$ et $c$.
Soit $[\,A\,|\, b\,]$ la matrice augmentée de ce système.

\begin{enumerate}[a)]
\item  Pour 
\underbar{toutes} les valeurs de
$a$ et $c$, trouvez:\smallskip
\begin{enumerate}[(i)]
\item 
$\rank (A)$,
\smallskip
\item  $\rank[\,A\,|\, b\,]$.
\end{enumerate}

\item\sov~ Trouvez \underbar{toutes} les valeurs de
$a$ et $c$ pour lesquelles ce syst\`eme \smallskip
\begin{enumerate}[(i)]
 
\item admet une solution unique,\smallskip
\item  admet une infinit\'e de solutions,  \smallskip
\item  n'admet aucune solution. 
\end{enumerate}

\medskip
\item Dans le cas b (ii) ci-dessus, donnez la solution générale et sa description géométrique complète. 
\end{enumerate}
\end{prob} \begin{prob} \label{prob13.7} 
Considérons le réseau routier ci-dessous avec
intersections A, B, C et D. Les flèches
indiquent le sens de la circulation sur les routes, elle se fait à
sens unique. Les chiffres indiquent le nombre exact de voitures qui traversent les intersections A, B, C et D par minute.
Chaque $x_i$ représente le nombre \emph{variable} de voitures qui passent par les routes en question chaque minute.

 $$\xymatrix@=5pt{&&&&&&&&\\&&&&&&&&\\&&&&B\ar[uu]_{10}\ar[rrdd]^{x_4}&&&&\\&&&&&&&&\\
{}\ar[rr]^{11}&&A\ar[rrrr]^{x_5}\ar[rrdd]_{x_2}\ar[rruu]^{x_1}&&&&C\ar[rr]^4
&&\\&&&&&&&&\\&&&&D\ar[rruu]_{x_3}&&&&\\&&&&&&&&\\&&&&\ar[uu]^3&&&&\\}
$$
\begin{enumerate}[a)]
\item \'Ecrivez le système linéaire qui décrit le flux du trafic {\bf
ainsi que toutes les contraintes} sur les variables $x_i$, pour $i=1,\dots,5$. (Ne pas r\'esoudre ce syst\`eme pour le moment, cela sera fait à votre place dans  (b).)\smallskip

\item La MER de la matrice augmentée de la partie (a) est la suivante: 
$$ \bmatrix 
1& 0& 0& -1& 0&|& 10\\  
0& 1& 0& 1&1&|& 1\\ 
0& 0& 1& 1& 1&|& 4\\ 
0& 0& 0& 0& 0&|& 0\endbmatrix\,.$$ Donnez la solution générale. (Ignorez les contraintes à ce stade.)\smallskip

\item Si la route $\overline{AC}$ est fermée en raison de travaux, trouvez tous les flux de circulation possibles, 
 \underbar{en utilisant les résultats que vous avez obtenus à la question  (b)}. 

\end{enumerate}

\end{prob} \begin{prob} \label{prob13.3} Considérez le réseau routier avec
intersections A, B, C, D et E ci-dessous.   Les flèches
indiquent le sens de la circulation le long des routes, la circulation se faisant à sens unique. Les chiffres indiquent le nombre exact de voitures observées entrant ou sortant des intersections A, B, C, D et E pendant une minute.  Chaque
$x_i$ désigne le nombre inconnu de voitures qui sont passées par les routes correspondantes en une minute.
$$\xymatrix@=5pt{
&&&&&&&&&&&&\\ 
&&&&&&&&&&&&\\
&&&&&&&&&&&&\\
&&&&&&&&B\ar[rrdd]^{x_4}\ar[uuu]_{50}&&&&\\ 
&&&&&&&&&&&&\\
&&&&&&A\ar[lll]^{60}\ar[rruu]^{x_5}&&&&C\ar[lddd]_{x_3}\ar[llll]_{x_6}&&&&\ar[llll]^{70}\\
&&&&&&&&&&&&\\
&&&&&&&&&&&&\\
&&&&&&&E\ar[luuu]^{x_1}&&D\ar[ll]_{x_2}\ar[dddr]^{60}&&\\  
&&&&&&&&&&&&\\ 
&&&&&&&&&&&&\\
&&&&&&\ar[uuur]^{100}&&&&&&&&\\} 
$$
\smallskip     
\begin{enumerate}[a)]
\item \'Ecrivez le système linéaire qui décrit le flux du trafic {\bf
ainsi que toutes les contraintes} sur les variables $x_i$, pour $i=1,\dots,6$. (Ne pas r\'esoudre se syst\`eme pour le moment car cela sera fait à votre place dans  (b).)\smallskip

\item\sov~La MER de la matrice augmentée de la question (a) est la suivante: 
$$ \bmatrix 
  1 & 0 & 0 & 0 & -1 & 1 &|& 60 \\
 0 & 1 & 0 & 0 & -1 & 1 &|& -40 \\
 0 & 0 & 1 & 0 & -1 & 1 &|& 20 \\
 0 & 0 & 0 & 1 & -1 & 0 &|&  -50 \\
 0 & 0 & 0 & 0 & 0 & 0 &|& 0
\endbmatrix\,.$$ Donnez la solution générale. (Ignorez les contraintes à
à ce stade.)
\smallskip
 

\item Si la route $\overline{ED}$ est fermée en raison de travaux, trouvez le flux minimum le long de la route  $\overline{AC}$  \underbar{en utilisant les résultats que vous avez obtenus à la question (b)}. 
  
\end{enumerate}

\end{prob} \begin{prob} \label{prob13.4}   Pour chacun des énoncés suivants, indiquez s'il est (toujours) vrai
ou s'il est (possiblement) faux.    
   \smallskip    
\begin{enumerate}[$\bullet$]
\item Si vous dites que l'affirmation est vraie, vous devez donner une explication claire en r\'ef\'erant \`a un théorème, ou en donnant une {\it preuve valable pour tous les cas}.
\item Si vous dites que l'affirmation est fausse, vous devez donner un contre-exemple explicite.  
\end{enumerate}
\medskip
\begin{enumerate}[a)]
\item Tout système non-homogène de 3 équations à 2 inconnues est incompatible.
\medskip
 
\item\sov~Tout système non-homogène de 3 équations à 2 inconnues est compatible.

\medskip
 
\item Tout système non-homogène de 2 équations à 3 inconnues a une infinité de solutions.
\medskip
 
\item\sov~Tout système de 2 équations à 2 inconnues admet une unique solution.
\medskip
 
\item S'il existe une ligne de zéros dans la matrice augmentée d'un système linéaire, alors le système admet une infinité de solutions.
\medskip
 
\item\sov~Si la matrice coefficients d'un système linéaire compatible comporte une colonne de zéros, alors le système admet une infinité de solutions.
\medskip
 
\item Si un système linéaire compatible admet une infinité de solutions, alors il doit y avoir une ligne de zéros dans la MER de la matrice augmentée.
\medskip
 
\item\sov~Si un système linéaire compatible admet une infinité de solutions, alors il doit y avoir une colonne de zéros dans la MER de la matrice coefficients.
\medskip
 
\item Si la matrice augmentée d'un système linéaire homogène de $3$ équations à $5$ inconnues est de rang $2$, alors il y a $4$ paramètres dans la solution générale.
\medskip
 
\item\sov~Si un système linéaire homogène admet une unique solution, alors ce système comprend le même nombre d'équations que d'inconnues.
\medskip
 
\end{enumerate}

\end{prob}

