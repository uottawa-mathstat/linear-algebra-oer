\section*{Exercices}
\addcontentsline{toc}{section}{Exercices}
La solution des exercices marqués par un astérisque $\star$ se trouvent à la fin du livre. Essayez-les d'abord avant de consulter la solution.

\begin{prob}
\label{prob01.1}\sov

 Écrire les nombres complexes suivants sous forme cartésienne   $a + b \,i$ avec $a,\, b \in \R$.
\medskip
\begin{enumerate}[a)]
\item $(2+i)(2+2 i)$. \medskip  
\item $ \dfrac 1{1+i}$.\medskip 
\item $ \dfrac{8+3i}{5-3i}$.\medskip 
\item $\dfrac{5+5\sqrt{3}\,i}{\sqrt{2}-\sqrt{2}\,i}$.\medskip 
\item $\dfrac{(1+2i)(2+5i)}{3+4i}$.\medskip  
%$\dfrac{ 12}{25} +\dfrac{59}{25}i$
\item $\dfrac{1-i}{2-i}+\dfrac{2+i}{1-i}$.\smallskip
\item $ \dfrac 1{(1-i)(3-2i)}$.
\end{enumerate}

\end{prob}
\begin{prob}
\label{prob01.2}\sov~Trouver la forme polaire des nombres complexes suivants (c'est-à-dire les écrire soit sous la forme $r e^{i\theta}$ soit sous la forme $r(\cos \theta + i \sin \theta)$, avec $r\ge 0$ et $-\pi <\theta \le \pi$) :\medskip
\begin{enumerate}[a)]

\item ${3\sqrt{3}-3i}$.\medskip
%$6(\cos (-\pi /6)+i \sin (-\pi /6)) $$6(\nbsp)
\item $\dfrac{3\sqrt{3}-3i} {\sqrt{2}+i\sqrt{2}}$.\medskip
%$3(\cos (-5\pi /12)+i \sin (-5\pi /12)) $(3)

\item $\dfrac{1-\sqrt{3}i}{-1+i}$. \medskip
%$\sqrt{2}(\cos (11\pi /12)+i\sin (11\pi /12))$$.
\item $\dfrac{5+5\sqrt{3}i}{\sqrt{2}-\sqrt{2}i}$. \medskip
%$5(\cos (7\pi /12)+i\sin (7\pi/12))$.
\item $\dfrac{3+3\sqrt{3}i} {-2+2i}$. \medskip
% $\dfrac{3}{\sqrt{2}}(\cos (5\pi /12)-i\sin (5\pi/12))$
\end{enumerate}

\end{prob}

\begin{prob}
\label{prob01.3} Trouvez le module de chacun des nombres complexes des exercices 1 et 2. (Rappelez-vous que $|z w|=|z|\, |w|$ et que si $w\not=0$, alors $\big|\frac{z}{w}\big|=\frac{|z|}{|w|}$.)

\end{prob}
\begin{prob}
\label{prob01.4}\sov~Si $z$ est un nombre complexe,

\begin{enumerate}[(i)]
	\item Est-il possible que $z={\bar z}$ ?
	\item Est-il possible que $|{\bar z}|>|z|$ ?
	\item Est-il possible que ${\bar z}=2z$ ? 
\end{enumerate}

\medskip Si oui, donnez des exemples pour illustrer la réponse. Si non, expliquez pourquoi l'énoncé est toujours faux.
 \end{prob}






\begin{prob}
\label{prob01.5}
\textbf{Pour les algébristes  ...}\\
Montrez qu'il n'existe pas de relation d'ordre sur les nombres complexes, c'est-à-dire qu'il n'existe pas de relation binaire $>$ ayant la propriété que si $x>0$ et $y>z$ alors $xy>xz$.
 \end{prob}
