\section*{Exercices}
\addcontentsline{toc}{section}{Exercices}


 \begin{prob} \label{prob08.1} Soit $V$ un espace vectoriel quelconque. Justifiez vos réponses aux questions suivantes : 
\begin{enumerate}[a)] 

\item Supposons que $\uu \in \sp{\vv,\ww}$. Montrez soigneusement que $\set{\uu,\vv,\ww}$ est LD.
\medskip
 
\item\sov~Supposons que $\uu \in \sp{\vv,\ww}$. Montrez soigneusement que $\sp{\vv,\ww}=\sp{\uu,\vv,\ww}$.
\medskip
 

\item Réciproquement, supposons que $\sp{\vv,\ww}=\sp{\uu,\vv,\ww}$. Montrez soigneusement que $\uu \in \sp{\vv,\ww}$. 
\medskip
 

\item\sov~Supposons que $\sp{\vv,\ww}=\sp{\uu,\vv,\ww}$. Montrez soigneusement que $\set{\uu,\vv,\ww}$ est LD. 
\medskip
 

\item Supposons que $\sp{\uu,\vv}=\sp{\ww,\xx}$. Montrez soigneusement que $\set{\uu,\vv,\ww}$ et $\set{\uu,\vv,\xx}$ sont tous les deux linéairement dépendants. 
\medskip
 
 
\item\sov~Supposons que $\set{\vv,\ww}$ soit LI et que $\uu \notin \sp{\vv,\ww}$. Montrez soigneusement que $\set{\uu,\vv,\ww}$ est aussi LI.
\medskip
  
\item Supposons que $\set{\vv,\ww}$ soit LI et que $\uu \notin \sp{\vv,\ww}$. Montrez soigneusement que $\sp{\vv,\ww}\not=\sp{\uu,\vv,\ww}$.
\medskip
 

\item\sov~Réciproquement, supposons que $\sp{\vv,\ww}\not=\sp{\uu,\vv,\ww}$. Montrez soigneusement que $\uu \notin \sp{\vv,\ww}$. 
\medskip
  
\end{enumerate}

\end{prob} \begin{prob} \label{prob08.2} Justifiez vos réponses aux questions suivantes :\medskip
\begin{enumerate}[a)] 


\item Supposons que $\uu,\vv,\ww$ sont des vecteurs non-nuls de $\R^4$ tels que
$\uu\cdot \vv= \uu\cdot \ww= \vv\cdot \ww=0$. Prouvez que $\set{\uu,\vv,\ww}$ est linéairement indépendant. \medskip



\item\sov~Supposons que deux polynômes $p$ et $q$ satisfassent $p\not=0$ et $\deg(p) <\deg(q)$. Montrez soigneusement que l'ensemble $\set{p,q}$ est linéairement indépendant. \medskip
 



\item Supposons qu'un ensemble de polynômes $\set{p_1, \dots, p_k}$ satisfasse $0\not=p_1$ et $\deg(p_1) <\deg(p_2)< \cdots < \deg (p_k)$. Montrer soigneusement que l'ensemble $\set{p_1, \cdots, p_k}$ est LI. \medskip
 
\item$^\ast$ Supposons que $f,g \in \F(\R)$ sont des fonctions dérivables et que $fg'-f'g $ n'est pas identiquement nulle (c'est-\`a-dire n'est pas la fonction nulle). Prouvez soigneusement que $\set{f,g}$ est linéairement indépendant.\footnote{ Indiccation : raisonnez par contradiction.}\medskip     
 
\item Utilisez la question précédente de cet exercice pour donner une autre preuve (c'est-à-dire, autre que celle vue dans un des exemples de ce chapitre) que l'ensemble $\set{\sin x, \cos x}$ est linéairement indépendant dans $\F(\R)$.\medskip 
 
\item\sov$^\ast$ Supposons que $\set{\uu,\vv,\ww}$ est un ensemble de vecteurs de $\R^3$ tels que $\uu\cdot (\vv\times \ww) \not=0$. Prouvez soigneusement que $\set{\uu,\vv,\ww}$ est linéairement indépendant.\footnote{ {\it Un argument géométrique impliquant un \og volume \fg\ n'est pas suffisant.} [Indication : rappelez-vous d'abord que l'ensemble $\{\uu,\vv,\ww\}$ est linéairement indépendant ssi aucun des vecteurs n'est une combinaison linéaire des autres. Ensuite, on raisonne par l'absurde et l'on réduit le nombre de cas à vérifier de 3 à 1 en gardant en t\^ete que, pour trois vecteurs quelconques $\vv_1, \vv_2, \vv_3 \in \R^3$ on a toujours la relation $\vv_1\cdot (\vv_2\times \vv_3)=\vv_3\cdot (\vv_1\times \vv_2)=\vv_2\cdot (\vv_3\times \vv_1)$.]} 
\end{enumerate}
 \end{prob} 

