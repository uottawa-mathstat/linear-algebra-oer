%%%%%%%%%%%%%%%%%%%%%%preface.tex%%%%%%%%%%%%%%%%%%%%%%%%%%%%%%%%%%%%%%%%%
% sample preface
%
% Use this file as a template for your own input.
%
%%%%%%%%%%%%%%%%%%%%%%%% Springer %%%%%%%%%%%%%%%%%%%%%%%%%%

\preface

\addcontentsline{toc}{chapter}{Pr\'eface}

Ce volume constitue la traduction fran\c{c}aise de la quatri\`eme \'edition du livre \og Vector Spaces First~\fg\ de Thierry Giordano, Barry Jessup et Monica Nevins, lequel est n\'e de la mise en commun de notes du cours \emph{Introduction à l'algèbre linéaire} enseigné à l'Université d'Ottawa. Ce livre est destiné à servir de manuel ou de compagnon pour compléter le cours.



L'approche que nous adoptons dans ce livre n'est pas standard : contrairement aux approches usuelles, nous introduisons les espaces vectoriels très tôt et nous ne traitons les systèmes linéaires qu'après une telle introduction approfondie des espaces vectoriels.

Nous agissons ainsi pour au moins deux raisons. D'une part, après avoir enseigné ce cours de diverses manières à plusieurs milliers d'étudiants sur près de vingt-cinq ans, nous sommes maintenant en mesure d'observer que les outils relatifs aux espaces vectoriels sont généralement mal vécus par les étudiants car ils représentent la partie la plus difficile du cours et trouvent pourtant traditionnellement leur place à la fin. Dans un cours dispensé en seulement douze semaines, le fait que la matière la plus difficile se trouve à la fin ne laisse pas à la plupart des étudiants suffisamment de temps pour s'attaquer aux concepts (apparemment) nouveaux que l'on peut rencontrer lorsque l'on s'attaque aux espaces vectoriels pour la première fois.

À l'opposé, notre expérience nous montre que le fait d'aborder les espaces vectoriels dès les deux premières semaines permet aux étudiants de bien mieux appréhender les deux \og grandes~\fg\ idées de l'algèbre linéaire: la notion de combinaison linéaire d'une famille de vecteurs, qui donne lieu à ce que l'on appelle \og enveloppe lin\'eaire~\fg, et la notion d'\og indépendance linéaire~\fg\ d'une famille de vecteurs. Ces deux notions sont au cœur de l'algèbre linéaire et sont généralement ressenties comme nouvelles, abstraites et difficiles par les étudiants lorsqu'elles sont rencontrées pour la première fois. Le mieux est donc sans doute d'introduire ces notions le plus tôt possible pour pouvoir ensuite les utiliser et ainsi définir par exemple les notions de {\it base} et de {\it dimension}, lesquelles sont utiles à la fois dans le reste du cours et dans différents autres contextes. Aussi, toujours d'après notre expérience, la plupart des étudiants semblent préférer aborder les concepts difficiles le plus tôt possible afin d'avoir le temps de se familiariser avec des idées vraisemblablement nouvelles et différentes de ce qu'ils avaient vu dans l'enseignement secondaire.

D'autre part, une autre raison justifiant notre organisation est d'alerter les étudiants sur le fait qu'il y a réellement des concepts à la fois nouveaux et différents dans ce cours ! En effet, si l'on commençait le cours avec les systèmes linéaires que beaucoup d'étudiants ont déjà étudiés auparavant (en petite dimension du moins), il serait alors facile de se reposer sur l'idée qu'au final peu de nouvelles notions seront abordées dans ce cours, si bien que les étudiants seraient pris au dépourvu plus tard lorsque les espaces vectoriels sont introduits... Au contraire, si l'on commence avec des concepts nouveaux mais abordables, les étudiants sont stimulés pour aller de l'avant. D'où notre choix d'organisation pour cet ouvrage.






