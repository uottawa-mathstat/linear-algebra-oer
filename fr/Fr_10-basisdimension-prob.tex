\section*{Exercices}
\addcontentsline{toc}{section}{Exercices} 


 
\begin{enumerate}
\item Les questions avec un astérisque $ ^\ast$ (ou deux !) signifie que le niveau est difficile, c'est du bonus.
\item Vous devez justifier toutes vos réponses.
\end{enumerate}
\smallskip



 \begin{prob} \label{prob09.1} Donnez \underbar{deux} bases diff\'erentes pour chacun des sous-espaces suivants. En d\'eduire leur dimension.
\medskip

\begin{enumerate}[(a)]

\item  $\set{(2x, x) \in \R^2\st x\in \R}$.\medskip
% no

\item\sov~ $\set{(x, y) \in \R^2\st 3x - y=0 }$. \medskip
 


 
\item  $\set{(x, y, z) \in \R^3\st x+y-2z=0 }$.   \medskip
 

\item\sov~ $\set{(x, y, z, w) \in \R^4\st x-y+z-w=0 }$.   \medskip
 

\item  $\Bigg\{  \bmatrix a&b\\ c&d\endbmatrix \in \M_{2 \,2}(\R) \;\Bigg|\; b=c\Bigg\}$.\medskip \medskip
 

\item\sov~ $\Bigg\{  \bmatrix a&b\\ c&d\endbmatrix \in \M_{2 \,2}(\R) \;\Bigg|\; a+d=0\Bigg\}$. \medskip
 

\item  $\Bigg\{  \bmatrix a&0\\ 0&b\endbmatrix \in \M_{2 \,2}(\R) \;\Bigg|\;  a, b \in \R\Bigg\}$. \medskip
 


\item\sov~ $\Bigg\{  \bmatrix 0&-b\\ -b&0\endbmatrix \in \M_{2 \,2}(\R) \;\Bigg|\; b \in \R\Bigg\}$.      \medskip
 

\item  $\Bigg\{  \bmatrix a&b\\ c&d\endbmatrix \in \M_{2 \,2}(\R) \;\Bigg|\; a+b+c+d=0\Bigg\}$.      \medskip


\item\sov~ $ \PP_n $.  \medskip

\item  $ \set{p \in \PP_2 \st  p(2)=0}$.  \medskip


\item\sov~ $ \set{p \in \PP_3 \st  p(2)=p(3)=0}$.  \medskip



\item  $ \set{p \in \PP_2 \st  p(1)+p(-1)=0}$.      \medskip

\item\sov~ $ \sp{\sin x, \cos x}$.      \medskip

\item  $ \sp{1, \sin x, \cos x}$.      \medskip

 \item\sov~ $ \sp{1, \sin^2 x, \cos^2 x}$.      \medskip


\item$^\ast$  Soit l'ensemble $\set{(x, x-3) \in \R^2\st x\in \R}$ muni des \underbar{opérations non-standards suivantes} :\\ 
Addition : $(x,y) \tilde+ (x',y')=(x+x', y+y +3).$\\ 
Multiplication par scalaire: pour $k\in \R$ , $k \odot (x,y)=(kx, ky+3k-3).$

\item$^\ast$ Soit le sous-ensemble $\set{(x, y, z) \in \R^3\st x+2y+z=2 }$ de $V=\R^3$ muni des \underbar{\it opérations non-}  \underbar{\it standards suivantes} :\\
Addition : $(x,y,z) \tilde+ (x',y',z')=(x+x', y+y,z+z'-2)$.\\ 
Multiplication par scalaire: pour $k\in \R$, $k\odot (x,y,z)=(kx, ky, kz-2k+2)$.\medskip 

\end{enumerate}
{\it [ Indication pour les questions (k) et (l) : rappelez-vous que si $p$ est un polynôme en $x$ de degré $\geq1$ et que $p(a)=0$ pour un certain $a \in \R$, alors $p$ a un facteur de la forme $x-a$; c'est-à-dire que $p$ s'écrit $p(x)=(x-a)\, q(x)$ pour un certain polynôme $q$ tel que $\deg(q)=\deg(p)-1$.]}



\end{prob} \begin{prob} \label{prob09.2} Déterminez si les ensembles suivants sont des bases des espaces vectoriels indiqués.  
  
\begin{enumerate}[(a)]
\medskip
\item  $\set{(1,2)} $; ($\R^2$). \medskip
% yes
\item\sov~$\set{(1,2), (-2,-4)} $; ($\R^2$).\medskip
% no

\item $\set{(1,2), (3,4)} $; ($\R^2$).\medskip
% no

\item\sov~$\set{(1,2), (3,4), (0,0)} $; ($\R^2$).\medskip
% no

\item  $\set{(1,2,3), (4,8,6)}$; ($\R^3$). \medskip


\item\sov~$\set{(1,2,3), (4,8,7)}$; ($\R^3$).\medskip


\item $\set{(1,2,3), (4,8,7), (3,6,4)}$; ($\R^3$).\medskip

\item\sov~$\set{(1,0,1,0), (0,1,0,1)}$; ($\R^4$).\medskip


\item  $\Big\{\bmatrix 1&0\\1&2\endbmatrix, \bmatrix 0&1\\0&1\endbmatrix \Big\}$; ($\M_{2 \,2}$).\medskip


\item\sov~ $\Big\{\bmatrix 1&0\\1&2\endbmatrix, \bmatrix 0&1\\0&1\endbmatrix , \bmatrix 1&-2\\1&0\endbmatrix \Big\}$; ($\M_{2 \,2}$).\medskip


\item  $\Big\{\bmatrix 1&0\\0&0\endbmatrix, \bmatrix 0&1\\0&0\endbmatrix,\bmatrix 0&0\\1&0\endbmatrix,\bmatrix 0&0\\0&1\endbmatrix \Big\}$; ($\M_{2 \,2}$).\medskip

 
\item  $\set{1, 1-x, 1-2x}$; ($\PP_2$). \medskip

\item\sov~ $\set{1, 1+x, x^2}$; ($\PP_2$). \medskip 

\item  $\set{ \sin x,  \cos x}$; ($\F(\R)$). \medskip  

\item\sov~ $\set{1, \sin x, 2 \cos x}$; ($\F(\R)$). \medskip  

\item  $\set{2, 2\sin^2 x,  3\cos^2 x}$; ($\F(\R)$). \medskip  

\item$^{\ast\ast}$ $\set{(1,0), (0,0)}$; $ V=\R^2$  muni des \underbar{opérations non-standards suivantes} :\\ 
Addition : $(x,y) \tilde+ (x',y')=(x+x', y+y-2)$.\\
Multiplication par scalaire: pour $k\in \R$, $k\circledast (x,y)=(kx, ky-2k+2)$.     \medskip

\item$^{\ast\ast}$ $\set{(1,3), (2,4)}$; $ V=\R^2$ muni des \underbar{opérations non-standards suivantes} :\\ 
Addition : $(x,y) \tilde+ (x',y')=(x+x', y+y-2)$.\\ 
Multiplication par scalaire: pour $k\in \R$, $k\circledast (x,y)=(kx, ky-2k+2)$.     \medskip

\end{enumerate}
 

\end{prob} \begin{prob} \label{prob09.3}
Soient $f(x) =1+x$, $g(x) = x+ x^2$ et $h(x) = x+ x^2 + x^3$ des polynômes de $\PP$, et soit 
$W=\sp{f,g,h}$.
\smallskip
\begin{enumerate}[(a)]
\item Montrez que $f$, $g$ et $h$ sont linéairement indépendants. 
\item Trouvez une base pour $W$. En d\'eduire sa dimension.
\smallskip
\item Si $j(x) = 1-x^2 +x^3$, montrez que $j \in W$.
\item Que vaut $\dim \sp{f,g,h,j}$ ?
\end{enumerate}
\end{prob} 

\begin{prob} \label{prob09.4} Soit $U=\set{(x,y,z,w) \in \R^4 \st x-y+z-w=0}$.\medskip
\begin{enumerate}[(a)]
\item Trouvez une base pour $U$. En d\'eduire la valeur de $\dim U$.


\item Étendre la base trouvée en (a) pour obtenir une base de $\R^4$. 
\end{enumerate}
 

\end{prob} \begin{prob} \label{prob09.5} Considérons l'espace vectoriel
$ \F\big([0,2]\big)=\set{f \st f : [0,2] \to \R}$ (muni des mêmes opérations que $\F(\R)$.) Supposons que $f(x)= x$,  
que $g(x)=\frac1{x+1}$ et que $W=\sp{f,g}.$

\begin{enumerate}[a)]
\item Montrez que $\set{f,g}$ est linéairement indépendant. 
\item En d\'eduire $\dim W$.
\item Si $h(x)=x^2$, montrez que $h\notin W$. 
\item Quelle est la dimension de $\sp{f,g,h}$ ?
\end{enumerate}


\end{prob} \begin{prob} \label{prob09.6} Soit $\E=\set{\text{«} \ ax+by+ cz=d\  \text{»}\st a,b,c, d\in\R}$ l'ensemble des équations linéaires à coefficients réels en les variables $x$, $y$ et $z$. On muni $\E$ des opérations habituelles: l'addition d'équations, notée ici par \og $\dsum$ \fg\ et la multiplication par scalaire, notée ici par \og $\circledast$ \fg, définies comme suit : 

$$\text{\text{«}} \ ax+by+ cz=d\  \text{»}\, \dsum\, \text{\text{«}}\ ex+fy+gz=h\ \text{»}\,=\,\text{\text{«}} \ (a+e)x + (b+f)y + (c+g)z=d+h\  \text{»} $$
et
$$ \forall k\in \R,\quad k\circledast \, \text{\text{«}}\  ax+by+ cz=d\ \text{»}\, = \,\text{\text{«}}\  ka\, x+ kb \,y+ kc \,z = k\,d\ \text{»}.$$

Dans un exercice précédent, on a montré qu'avec ces opérations $\E$ est un espace vectoriel. Trouvez une base pour $\E$ et en d\'eduire $ \dim \E$.
\end{prob}
