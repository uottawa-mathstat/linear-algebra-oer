\begin{sol}{prob17.1} Dans chaque question, étendez l'ensemble LI donné en une base de $\R^n$.  (Utilisez vos résultats obtenus à l'Exercice \ref{prob16.1} lorsque cela est utile.)
\medskip

(b) $\set{(1, 0,1), (0,1,1)}$ dans $\R^3$.

\soln Puisque $\dim \R^3=3$ et que nous avons déjà 2 vecteurs linéairement indépendants, nous devons seulement trouver $\ww\in \R^3$ tel que $\rank \scriptsize\bmatrix 1& 0&1\\
0&1&1\\&\ww\endbmatrix=3.$ Clairement $\ww=(0,0,1)$ convient, puisqu'alors $\scriptsize\bmatrix 1& 0&1\\
0&1&1\\&\ww\endbmatrix$ est une ME dont le rang est  $3$. 
\medskip

(d) $\set{(1, 0, 1, 3)}$ dans $\R^4$.

\soln Puisque $\dim \R^4=4$ et que nous n'avons qu'un seul vecteur linéairement indépendant, nous devons trouver $\uu,\vv,\ww$ dans $\R^4$ tels que $\rank \scriptsize\bmatrix 1&0&1&3\\
&&\hspace{-0.5cm}\uu\\ &&\hspace{-0.5cm}\vv\\&&\hspace{-0.5cm}\ww\endbmatrix=4.$ Clairement, les vecteurs $\uu=(0,1,0,0),\ \vv=(0,0,1,0)$ et $\ww=(0,0,0,1)$ sont convenables, puisqu'alors $\scriptsize\bmatrix 1&0&1&3\\
&&\hspace{-0.5cm}\uu\\ &&\hspace{-0.5cm}\vv\\&&\hspace{-0.5cm}\ww\endbmatrix$ est une ME dont le rang est $4$.
\medskip


\end{sol}