\chapter{ Th\'eor\`emes sur la dimension}
\label{chapter:Fr_11-dimensiontheorems}

Dans le chapitre précédent, nous avons vu une \stress{inégalité intéressante} :

\standout{ Taille de tout ensemble LI de $V$ \; $\leq$ 
\; Taille de tout ensemble qui engendre $V$\,.} 

De cet énoncé découlaient les propriétés et définitions suivantes : 
\begin{itemize}
	\item  \stress{Tous les ensembles linéairement indépendants qui engendrent}  $V$
sont appelés \emph{bases} de $V$ et ils ont tous le même nombre de vecteurs. Ce nombre constant $n$ de vecteurs dans
les bases est appelé la \stress{dimension} de $V$ et est noté $\dim(V)$.
	\item Nous avons trouvé des bases pour un grand nombre d'\emph{espaces vectoriels} typiques, et nous avons également cherché des bases pour des \emph{sous-espaces vectoriels}.\\
\end{itemize} 






À partir de la définition de dimension, nous pouvons maintenant améliorer l'inégalité ci-dessus pour obtenir une \red{inégalité importante} :

\standout{Taille de tout ensemble LI dans $V$ \; $\leq\,\dim V\,\leq\,$  
 Taille de tout ensemble g\'en\'erateur de $V$\,.} 


\section{Tout sous-espace d'un espace de dimension finie admet une base finie}

Dans les chapitres précédents, nous avons montré les r\'esultats suivants :
\begin{itemize}
\item Si $\{\vv_1, \cdots, \vv_m\}$ est un sous-ensemble LI de $V$, alors il existe
des vecteurs $\vv_{m+1}, \cdots, \vv_n$ dans $V$ tels que $\{\vv_1, \cdots, \vv_m, \vv_{m+1}, \cdots, \vv_n\}$ est une base pour $V$; \stress{c'est-à-dire que tout sous-ensemble linéairement indépendant de $V$ peut être étendu en une base de $V$.}
\item Réciproquement, nous avons vu que si $S = \{\vv_1, \cdots, \vv_l\}$ est un ensemble gén\'erateur de $V$, alors il existe un sous-ensemble de $S$ qui est une base de $V$. Autrement dit, \stress{tout ensemble gén\'erateur peut être réduit en une base de $V$.}
\end{itemize}

\begin{remark}
Dans un espace vectoriel de dimension \stress{infinie}, le processus d'ajout de vecteurs à votre ensemble LI pourrait ne jamais s'arrêter, puisqu'il pourrait arriver que vous
ne puissiez pas engendrer cet espace par un nombre fini de vecteurs.  D'où notre
restriction aux espaces vectoriels de dimensions \stress{finie}.  \end{remark}

En théorie, cela signifie qu'on peut toujours trouver une base pour
n'importe quel sous-espace $W$ de dimension finie : soit vous commencez avec un ensemble g\'en\'erateur (par exemple $W$ lui-même) qu'on réduit en une base, soit
vous d\'ebutez par un ensemble de vecteurs non-nuls quelconques LI (par exemple $\{\ww\}$ avec $\ww\neq\zero$) qu'on \'etend en une base.  

Mais en pratique, ce n'est pas si facile : cette méthode requiert des calculs intensifs...  (Nous y reviendrons
bientôt !)



\section{Un raccourci pour vérifier si un ensemble est une base}

\begin{theorem}[Raccourci pour savoir si un ensemble est une base]\index{raccourci pour décider si un ensemble est une base}\label{Raccourci_bases}
Soit $V$ un espace vectoriel de dimension finie $\dim(V) = n < \infty$.
On a :
\begin{enumerate}
\item Tout ensemble LI $\{\vv_1, \cdots, \vv_n\}$ avec précisément $ n $ vecteurs de $V$ est une base de $V$ \stress{(c'est-à-dire, qu'il engendre aussi $V$ !);}
\item Tout ensemble g\'en\'erateur $\{\vv_1, \cdots, \vv_n\}$ de $V$ constitué d'exactement $n $ vecteurs est une base de $V$ \stress{(c'est-à-dire, qu'il est aussi LI !);}
\end{enumerate}
\end{theorem}

Ce théorème est très important ! Il indique que si vous connaissez déjà la dimension de $V$, vous disposez alors d'un raccourci pour trouver une base: il suffit de trouver un ensemble LI \emph{ou} un ensemble gén\'erateur avec le nombre EXACT de vecteurs. 

\begin{proof}
\begin{enumerate}
\item Supposons que $\{\vv_1, \cdots, \vv_n\} \subseteq V$ soit LI avec exactement $n$ vecteurs distincts, et supposons 
par l'absurde que $\{\vv_1, \cdots, \vv_n\}$ n'engendre pas $V$.
Nous pouvons alors trouver un vecteur $\vv \in V$ tel que $\{\vv, \vv_1, \cdots, \vv_n\}$ soit LI (voir Th\'eor\`eme \ref{EnlargingLI}).
C'est un ensemble LI avec $n+1$ vecteurs de $V$, et on obtient l'inégalité $\dim(V)=n < n+1$ qui contredit notre \red{inégalité importante} du début du chapitre...  Donc la supposition par l'absurde est fausse,
notre ensemble engendre bien $V$ et il est donc une base.

\item Supposons maintenant que $\spn\{\vv_1, \cdots, \vv_n\} = V$ mais que
 $\{\vv_1, \cdots, \vv_n\}$ n'est \emph{pas} LI.  Ne pas être LI signifie être LD, donc on peut enlever au moins un vecteur sans affecter l'enveloppe lin\'eaire :
il existe un sous-ensemble $S \subset \{\vv_1, \cdots, \vv_n\}$ avec au plus $n-1$ vecteurs tel que $\spn(S) = \spn\{\vv_1, \cdots, \vv_n\}=V$.  Mais alors l'inégalité $\dim(V)=n > n-1$ qui contredit notre \red{inégalité importante} du début du chapitre...  D'o\`u l'ensemble $\{\vv_1, \cdots, \vv_n\}$ est nécessairement LI et  c'est donc une base de $V$.
\end{enumerate}
\end{proof}

\begin{myprob} Montrez que $\{(2,2,2), (7,1,-11)\}$ forme une base de $U = \{(x,y,z) | 2x-3y+z = 0\}$.  

\begin{mysol}  Nous voyons que $U$ est un plan passant par l'origine, donc c'est un espace vectoriel de dimension $2$.  On peut facilement vérifier que les deux vecteurs donnés appartiennent bien \`a $U$ en les substituant dans l'\'equation du plan :
$$
2(2) -3(2) + 2 = 0 \qquad \text{et} \qquad 2(7)-3(1)+(-11) = 0.
$$
De plus, puisqu'il y a exactement deux vecteurs et qu'ils ne sont pas des multiples scalaires 
l'un de l'autre, ils sont linéairement indépendants.  Au final :
\begin{itemize}
\item nous avons exactement deux vecteurs;
\item ils appartiennent \`a un espace de dimension $2$;
\item ils sont linéairement indépendants.
\end{itemize}
Donc, d'après le Théorème \ref{Raccourci_bases}, l'ensemble $\{(2,2,2), (7,1,-11)\}$ est une bien base de $U$.
\end{mysol}\end{myprob}

\begin{myprob} Étendre $\{(2,2,2), (7,1,-11)\}$ en une base pour $\R^3$.

\begin{mysol} Nous savons que $\dim(\R^3) = 3$, et nous remarquons rapidement que ces deux vecteurs appartiennent bien à $\R^3$. Étant donné que l'ensemble $\{(2,2,2), (7,1,-11)\}$ est LI, 
il suffit de trouver un vecteur $\vv\notin \sp{(2,2,2), (7,1,-11)}=U$, et on aura que $\{\vv, (2,2,2), (7,1,-11)\}$ sera LI et qu'ainsi que ce nouvel ensemble sera une base de $\R^3$ par le Théorème \ref{Raccourci_bases}.

On cherche donc un vecteur $\vv=(x,y,z)$ qui ne satisfait pas la condition définissant $U$, par exemple $\vv = (1,0,0)$, et on obtient que $\{(1,0,0), (2,2,2), (7,1,-11)\}$ est une base de $\R^3$.
\end{mysol}\end{myprob}

\standout{Attention : on peut utiliser ce «~raccourci~»  \emph{uniquement} lorsque l'on connaît la dimension de $V$.}

D'o\`u la question: comment faire pour connaître la dimension d'un espace vectoriel?


\section{Dimension des sous-espaces de $V$}

\begin{theorem} [Dimensions d'un sous-espace] \label{dimsubspaces}
 Supposons que $\dim(V) = n$ et soit 
$W$ un sous-espace de $V$.  Alors :
\begin{enumerate}
\item $0 \leq \dim(W) \leq \dim(V)$;
\item $\dim(W) = \dim(V)$ si et seulement si $W=V$;
\item $\dim(W)=0$ si et seulement si $W = \{\zero\}$.
\end{enumerate}
\end{theorem}

\begin{proof}
\begin{enumerate}
\item Prenez une base de $W$. Le nombre de vecteurs dans cette base est donc précisément $\dim(W)$ (qui est donc $\geq 0$ !), et ils sont bien sûr LI par définition. Donc par notre \red{inégalité importante}, on a $\dim(W) \leq \dim(V)$, ce qui conclut la preuve de ce point.

\item Supposons que $\dim(W) = \dim(V)=n$.  Par hypothèse, on sait que $W\subseteq V$. Si par l'absurde on avait $W\neq V$,
alors il existerait un vecteur $\vv \in V$ tel que $\vv \notin W$.
Mais alors, en notant $\{\vv_1,\cdots,\vv_n\}$ une base de $W$, 
l'ensemble $\{\vv, \vv_1,\cdots,\vv_n\}$ serait LI dans $V$
avec exactement $n+1$ vecteurs, ce qui contredit l'\red{inégalité importante} pour $V$...
On conclut alors que $W=V$. Réciproquement, si $W=V$, on a naturellement que leur dimension sont égales.

\item Si $\dim (W)=0$ et que par l'absurde $W$ contenait un vecteur non-nul $\vv$, alors $\{\vv\}$ serait LI et par l'\red{inégalité importante} nous aurions une contradiction : $0 = \dim (W) \ge 1$... Ainsi $\dim (W)=0 \implies W=\set{\zero}$. Réciproquement, si $W=\set{\zero}$, alors nous avons par convention que $\dim(W)=0$.\footnote{Il existe deux autres façons de définir la dimension d'un espace vectoriel : (a) la considérer comme la taille du plus grand ensemble linéairement indépendant, ou (b) la taille du plus petit ensemble gén\'erateur. Non seulement ces définitions sont équivalentes à notre définition de dimension pour les valeurs $\geq1$, mais elles ont aussi le bon goût de bien se généraliser à la valeur $0$ : par exemple pour la définition (a), il est clair que le plus grand sous-ensemble de $\{\zero\}$ qui est LI est l'ensemble vide $\emptyset$, lequel est bien sûr constitué de $0$ vecteur. D'où $\dim\{\zero\}=0$. On peut aussi concilier cela avec la définition (b) si nous sommes tous d'accord sur le fait que l'enveloppe lin\'eaire engendr\'e par l'ensemble vide est $\set{\zero}$... Si cela vous semble trop bizarre, restez sur la définition (a) !}
\end{enumerate}
\end{proof}

Ce théorème a d'importantes conséquences !

\begin{myexample} En utilisant la partie (1) du théorème, on a que tout sous-espace de $\R^3$ a dimension $0$ ou $1$ ou $2$ ou $3$.  
Cela correspond respectivement \`a: l'espace nul (partie (3) du théorème),
les droites, les plans, $\R^3$ tout entier (partie (2) du théorème). \end{myexample}

\begin{myexample} Le sous-espace de $\M_{2\times 2}(\R)$ qui a dimension $4$ est
 $\M_{2\times 2}(\R)$ lui-même.   \end{myexample}

\begin{myexample}
Le seul sous-espace bi-dimensionnel de $U = \{(x,y,z) | 2x-3y+z = 0\}$ est $U$ lui-même.  \end{myexample}

Ces deux derniers exemples illustrent une idée importante :

\standout{Si $U$ est un sous-espace de $V$ tel que $\dim(U) = m$,
alors pour tout sous-espace $W$ de $V$ tel que $W\subseteq U$, on a que $W$ est un sous-espace de $U$
et qu'il a donc une dimension $\leq m$.}

Autrement dit, nous n'avons pas besoin d'appliquer le théorème uniquement à nos « grands » espaces vectoriels $V$, mais aussi \`a nos sous-espaces.

\section{Avantages de la dimension des sous-espaces}

Au début de ce livre, nous avons parlé des droites et des plans
dans $\R^2$ et $\R^3$ et nous nous sommes demandés quelles seraient leurs généralisations à
$\R^n$.

Nous avons proposé le concept général d'espace vectoriel, dont les sous-espaces
de $\R^n$ en sont un exemple, et nous en avons déduit que les droites et les plans passant par
l'origine sont des sous-espaces de $\R^2$ et $\R^3$. Aussi, en plus de l'espace nul $\{\zero\}$ et de l'espace tout entier, ces espaces sont \emph{les seuls} sous-espaces 
de $\R^2$ et $\R^3$.

Nous sommes donc d'accord sur le fait que les « sous-espaces » fournissent l'équivalent des « droites » et des « plans » en dimensions sup\'erieures
 (du moins, les droites et plans qui passent par l'origine).

\`A ce stade, nous savons qu'en dimension finie tout sous-espace admet une base et qu'en particulier, si $S$
est une base de $U$, alors $U$ s'écrit $U = \spn(S)$.  En d'autres termes, nous pouvons
décrire chaque sous-espace en utilisant des équations paramétriques qui peuvent être vues comme des «~analogues en dimensions sup\'erieures~» des
équations paramétriques que nous avons l'habitude de manipuler pour une droite ou pour un plan.

De plus, nous avons vu comment mesurer la «~taille~» d'un sous-espace (à savoir, par sa \stress{dimension}), et nous pouvons ainsi classifier tous les sous-espaces possibles
d'un espace vectoriel en utilisant leur dimension. Par exemple:
\begin{itemize}
	\item dans $\R^n$, il existe des sous-espaces de dimension $0$, $1$, ... jusqu'à $n$ compris;
	\item dans $\PP_n$, il existe des sous-espaces de dimension $0$, $1$, ... jusqu'à $n+1$ compris;
	\item dans $M_{m\times n}(\R)$, il existe des sous-espaces de dimension $0$, $1$, ... jusqu'à $m\,n$ compris.\\
\end{itemize}

Ainsi, la première étape pour passer de la géométrie aux dimensions supérieures est déjà \'etablie. Passons aux autres étapes.

\section{Avantages des bases d'un sous-espace : partie I}

Alors, à quoi sert une base concrètement ?  Dans le chapitre précédent, nous avons \'evoqu\'e une 
application très importante, à savoir qu'on a l'intuition que tout
plan passant par l'origine dans $\R^3$ peut \^etre «~identifi\'e~» avec $\R^2$ : ils ont la même apparence et géométriquement ils correspondent au même type d'objet. Mais pour avoir une explication algébrique, c'est tout un défi !

\begin{myprob} Considérons le plan $U = \{ (x,y,z) | x-4y+z = 0\}$ dans $\R^3$.
Montrez que $$\{(2/3,1/3,2/3), (1/\sqrt{2},0,-1/\sqrt{2})\}$$ est
une base de $U$ et que ces deux vecteurs sont
orthogonaux et de norme 1.  (Nous appellerons une telle base
une \stress{base orthonormale}, cf. Chapitre ~\ref{chapter:Fr_22-orthogproj}).   

\begin{mysol}  Comme précédemment, il est facile de vérifier que ces deux
vecteurs appartiennent \`a $U$ et sont LI, donc comme $\dim(U)=2$, ceci qui implique qu'ils forment une base de $U$. Ensuite, on calcule leur produit scalaire, on voit qu'il donne $0$ et on en déduit qu'ils sont donc orthogonaux.  Enfin, on calcule leur
norme et on voit qu'elles sont bien égales à $1$. D'où le résultat.

(Rappelons que $\Vert (x,y,z) \Vert = \sqrt{x^2+y^2+z^2}$.)\end{mysol}\end{myprob}


Une base \emph{orthonormale} est un peu spéciale car elle a des propriétés en commun la base standard de $\R^n$.  Donc on peut voir une base orthonormale un peu «~comme~» une base standard de $U$.

En guise d'application, supposons que l'image d'un chat dessinée dans le plan $\R^2$,
et disons que le coin inférieur gauche est à l'origine et que le coin supérieur droit a pour coordonnées $(640,480)$, de sorte que
que chaque paire de coordonnées corresponde à un pixel.
On peut alors projeter notre image sur le plan $U$ en envoyant le point $(a,b)$ de $\R^2$ sur le point suivant de $U$ :
$$
  a\mat{2/3\\1/3\\2/3} + b\mat{1/\sqrt{2}\\0\\-1/\sqrt{2}}\,.
$$
Si vous voulez changer la couleur d'un pixel sur la nouvelle image dans $U$, donnez la nouvelle couleur
du pixel de coordonnées $(a,b)$ puis coloriez le point correspondant
$(x,y,z)$ de $\R^3$ décrit par la relation ci-dessus.
\qed

Mais cela soulève davantage de questions: comment obtient-on une telle 
base standard pour $U$ ?  Est-ce possible de l'obtenir dans tous les espaces espaces vectoriels, même lorsqu'on n'est pas dans le cas de $\R^n$? Et
qu'en est-il des dimensions supérieures ?  Nous verrons une partie des réponses plus loin dans ce livre.

\section{Avantage des bases d'un sous-espace: partie II}

Nous allons voir une application encore plus merveilleuse des bases. Mais tout d'abord, nous avons besoin d'une définition.

\begin{definition} Une {\it base ordonnée} $\set{\vv_1,\cdots, \vv_n}$ est l'ensemble $\set{\vv_1,\cdots, \vv_n}$ avec l'ordre pr\'ecisé des vecteurs.

\end{definition}
\begin{myexample} La base ordonnée $\set{(1,0), (0,1)}$ de $\R^2$ n'est pas la même que la base ordonnée. $\set{ (0,1), (1,0)}$\footnote{Même si, {\it en tant qu'ensembles}, $\set{(1,0), (0,1)}= \set{ (0,1), (1,0)}$, puisque les deux côtés ont exactement les mêmes vecteurs en eux. } L'ordre des vecteurs compte dans cette définition!
\end{myexample}

\begin{theorem}[Coordonnées]\index{coordonn\'ees}
Soit $\mathcal B = \{\vv_1,\cdots, \vv_n\}$ une base {\it ordonnée} d'un espace vectoriel $V$.  Alors, pour tout vecteur $\vv \in V$, il existe des scalaires  \emph{uniques} $x_1, \cdots, x_n \in \R$ tels que 
$$
\vv = x_1\vv_1 + \cdots + x_n\vv_n.
$$
Le $n$-uplet $(x_1,x_2, \ldots, x_n)$ est appelé \defn{coordonnées de
$\vv$ dans la base ordonnée $\mathcal B$}.
\end{theorem}

\begin{proof}
Ce n'est pas difficile à prouver!  
\begin{itemize}
	\item Existence : puisque $\mathcal B$ engendre $V$, vous pouvez 
écrire tout vecteur de $V$ comme une combinaison linéaire des vecteurs de $\mathcal B$. 
	\item Unicité : supposons qu'on a deux expressions de ce type, disons $\vv = x_1\vv_1 + \cdots + x_n\vv_n$ et $\vv = y_1\vv_1 + \cdots + y_n\vv_n$, et prouvons qu'elles sont en fait identiques.
Prenons la diff\'erence entre ces deux expressions pour avoir $\vv-\vv = \zero$ :
$$
(x_1-y_1)\vv_1 + \cdots + (x_n-y_n)\vv_n = \zero\,.
$$
Mais $\mathcal B$ est LI, donc cette relation est triviale et chacun des coefficients doit être nul, ce qui
donne $x_i = y_i$ pour tout $i$.  D'o\`u l'unicité.
\end{itemize}
\end{proof}

Nous pouvons utiliser ce théorème pour \emph{identifier} des espaces vectoriels de dimension $m$ à $\R^m$, 
tout comme nous avions identifié l'espace vectoriel bi-dimensionnel $U$ avec $\R^2$.

L'idée est que nous avons la correspondance suivante :

\begin{center}
\begin{tabular}{ccc}
$V$ &$\leftarrow \kern-.1em\rightarrow$ &$\R^n$ \\ 
$\vv = x_1\vv_1 + \cdots + x_n\vv_n$ & $\leftarrow \kern-.1em\rightarrow$ & $(x_1,x_2, \ldots, x_n)$ \,.
\end{tabular}
\end{center}

\begin{example}
En choisissant la base ordonnée $B=\Big\{{\scriptsize\bmatrix 1&0\\0&0\endbmatrix, \bmatrix 0&1\\0&0\endbmatrix,\bmatrix 0&0\\1&0\endbmatrix,\bmatrix 0&0\\0&1\endbmatrix }\Big\}$ (la base dite «~standard ordonnée~») de $\M_{22}(\R)$, 
nous avons la correspondance suivante :

\begin{center}
\begin{tabular}{ccc}
$\M_{22}(\R)$  &$\leftarrow \kern-.1em\rightarrow$&  $\R^4$ \\ 
$\mat{a&b\\c&d}$&$\leftarrow \kern-.1em\rightarrow$ & $(a,b,c,d)$\,.
\end{tabular}
\end{center}
\end{example}

\begin{example}
En choisissant $B = \left\{\mat{1&0\\0 & -1}, \mat{0&1\\0&0}, \mat{0&0\\1&0}\right\}$ comme base ordonnée pour $\sl_2$, nous avons la correspondance suivante :

\begin{center}
\begin{tabular}{ccc}
$\sl_2 $ &$\leftarrow \kern-.1em\rightarrow$& $\R^3$ \\ 
$\mat{a&b\\c&-a}$&$\leftarrow \kern-.1em\rightarrow$ & $(a,b,c)$\,.
\end{tabular}
\end{center}
\end{example}

\begin{example}
En choisissant la base standard ordonnée $\set{1, x, x^2}$ pour $\PP_2$, nous avons la correspondance suivante :

\begin{center}
\begin{tabular}{ccc}
$\PP_2$ &$\leftarrow \kern-.1em\rightarrow$& $\R^3$ \\ 
$a+bx+cx^2$&$\leftarrow \kern-.1em\rightarrow$ & $(a,b,c)$\,.
\end{tabular}
\end{center}
\end{example}

\standout{Attention : notez que l'ordre des vecteurs dans les bases ordonnées est important !  Si nous avions plutôt choisi $B = \{ x^2,x,1\}$ comme base ordonnée pour $\PP_2$, nous aurions obtenu les coordonnées $(c,b,a)$ au lieu de $(a,b,c)$ dans $\R^3$, ce qui pourrait prêter à confusion...  D'où l'importance de toujours garder l'ordre à l'esprit!}


