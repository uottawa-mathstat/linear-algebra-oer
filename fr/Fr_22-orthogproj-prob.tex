\section*{Exercices}
\addcontentsline{toc}{section}{Exercices}
 

\begin{prob} \label{prob19.1} Dans chaque cas, trouvez les coefficients de Fourier (coordonn\'ees) du vecteur $\vv$ par rapport à la base orthogonale $\mathcal B$ donnée de l'espace vectoriel $W$ indiqué.
\medskip

\begin{enumerate}[a)]
\item  $\vv=(1,2,3)$, $\mathcal B = \set{(1,0,1),(-1,0,1), (0,1,0) }$, $W=\R^3$.
\medskip
 
\item\sov~ $\vv=(1,2,3)$, $\mathcal B = \set{(1, 2 , 3 ),(-5, 4, -1),(1, 1, -1)}$, $W=\R^3$.
\medskip
 
\item  $\vv=(1,2,3)$, $\mathcal B = \set{(1, 0, 1),(-1, 2, 1)}$, $$W=\set{(x,y,z)\in\R^3 \st x+y-z=0}.$$
 
 
\item\sov~$\vv=(4,-5,0)$, $\mathcal B = \set{(-1, 0, 5),(10, 13, 2)}$, $$W=\set{(x,y,z)\in\R^3 \st 5x-4y+z=0}.$$
 
 
\item $\vv=(1,1,1,1)$,  $\mathcal B =\set{(1, 0, 1, 1), (0, 1, 0, 0), (0, 0, 1, -1)}$, $$W=\set{(x,y,z, w)\in\R^4 \st x-w=0}.$$
\medskip
 
\item\sov~$\vv=(1,0,1,2)$,  $\mathcal B =\set{(1, 0, 1, 1), (0, 1, 0, 0), (0, 0, 1, -1),(1, 0, 0, -1)}$, $W= \R^4$.
\medskip
 
\end{enumerate}

\end{prob} \begin{prob} \label{prob19.2} Trouvez la formule de la projection orthogonale sur les sous-espaces des questions c), d)~\sov~et e) ci-dessus.

\end{prob} \begin{prob} \label{prob19.3} Appliquez l'algorithme de Gram-Schmidt à chacun des ensembles LI suivants (pour obtenir un ensemble orthogonal), et vérifiez que l'ensemble de vecteurs que vous obtenez est bien orthogonal.
\medskip
\begin{enumerate}[a)]
\item $\set{(1,1,0),(2,0,3)}$.
\medskip
 
\item\sov~$\set{(1, 0, 0, 1),(0, 1, 0, -1),(0, 0, 1, -1)}$.
\medskip
 
\item $\set{(1, 1, 1, 1),(0, 1, 0, 0),(0, 0, 1, -1)}$.
\medskip
 
\item\sov~$\set{(1, 1, 0),(1, 0, 2),(1, 2, 1)}$.
\medskip
 
\end{enumerate}
\end{prob} \begin{prob} \label{prob19.4} Trouvez une base orthogonale pour chacun des sous-espaces suivants et vérifiez que votre base est bien orthogonale. (Astuce : tout d'abord, trouvez une base quelconque de manière standard, puis appliquez l'algorithme de Gram-Schmidt pour en trouver un orthogonale.)
\medskip
\begin{enumerate}[a)]
\item $W=\set{(x,y,z)\in\R^3 \st x+y+z=0}$.
\medskip
 
\item\sov~
 $U=\set{(x,y,z, w)\in\R^4 \st x+y-w=0}$.\medskip
 
\item $X=\set{(x,y,z, w)\in\R^4 \st x+y-w=0 \, \text{ et } \, z+y=0}$.
\medskip
 
\item\sov~$V=\ker \bmatrix 1 & 2 & -1 & -1 \\
 2 & 4 & -1 & 3 \\
 -3 & -6 & 1 & -7 \endbmatrix$.
\medskip
 
\end{enumerate}

\end{prob} \begin{prob} \label{prob19.5} Dans chaque cas, trouvez la meilleure approximation du vecteur $\vv$ donné par un vecteur $\ww$ du sous-espace $W$ donné. \medskip
\begin{enumerate}[a)]
\item  $\vv=(1,1,1)$,  $W=\set{(x,y,z)\in\R^3 \st x+y-z=0}$.
\medskip
 
\item\sov~$\vv=(1,1,1)$,   $W=\set{(x,y,z)\in\R^3 \st 5x-4y+z=0}$.
\medskip
 
\item  $\vv=(1,1,1,2)$,   $W=\set{(x,y,z, w)\in\R^4 \st x-w=0}$.
\medskip
 
\end{enumerate}
\end{prob} \begin{prob} \label{prob19.6}  Pour chacun des énoncés suivants, indiquez s'il est (toujours) vrai ou s'il est (possiblement) faux.   
   \smallskip    
\begin{enumerate}[$\bullet$]
\item Si vous dites que l'\'enonc\'e est faux, donnez un contre-exemple.   
\item Si vous dites que l'\'enonc\'e est vrai, donnez une explication claire - en citant un théorème ou en donnant une {\it preuve valide dans tous les cas}. 
\end{enumerate}

\medskip
\begin{enumerate}[a)]
\item Tout ensemble orthogonal est linéairement indépendant.
\medskip
 
\item\sov~Tout ensemble linéairement indépendant est orthogonal.
\medskip
 
\item Pour trouver la projection orthogonale d'un vecteur $\vv$ sur un sous-espace $W$, il suffit d'utiliser une base quelconque de $W$ dans la formule de la Définition\ref{projdef} (page~\pageref{projdef}).  
\medskip
 
\item\sov~Lorsque l'on cherche la projection orthogonale d'un vecteur $\vv$ sur un sous-espace $W$, une fois la réponse obtenue, on peut la multiplier la réponse par un  scalaire convenable pour éliminer les fractions.
\medskip
 
\item Lorsque l'on applique l'algorithme de Gram-Schmidt à une base, à chaque étape, il est bon de simplifier le vecteur obtenu en multipliant par un scalaire pour éliminer les fractions.
\medskip
  
\item\sov~Lorsque l'on recherche la projection orthogonale d'un vecteur $\vv$ sur un sous-espace $W$, si l'on utilise différentes bases orthogonales de $W$ dans la formule de la Définition~\ref{projdef} (page~\pageref{projdef}), alors on aura possiblement des réponses différentes.
\medskip
 
\item Si $\proj_W(\vv)$ désigne la projection orthogonale d'un vecteur $\vv$ sur un sous-espace $W$, alors le vecteur $\vv-\proj_W(\vv)$ est toujours orthogonal à tout vecteur de $W$.
\medskip
 
\item\sov~Pour vérifier qu'un vecteur, disons $\uu$, est orthogonal à tout vecteur dans $W$, il suffit de vérifier que $\uu$ est orthogonal à tout vecteur d'une base de $W$.
\medskip
 
\item$^{\ast \ast}$ Si l'on applique l'algorithme de Gram-Schmidt à un {\it ensemble g\'en\'erateur} $\set{\ww_1, \dots , \ww_p}$ d'un sous-espace $W$ plutôt qu'à une base de $W$, et si l'on obtient un vecteur nul \`a l'\'etape $k$, cela signifie que $\ww_k \in \sp{\ww_1, \dots , \ww_{k-1}}$.
\medskip
 
\item $^{\ast \ast}$ Si vous appliquez l'algorithme de Gram-Schmidt à un {\it ensemble g\'en\'erateur} $\set{\ww_1, \dots , \ww_m}$ d'un sous-espace $W$ plutôt qu'à une base de $W$, et que vous écartez tous les vecteurs nuls qui apparaissent au cours du processus, alors vous obtiendrez bien au final une base orthogonale pour $W$.
\medskip
 
\end{enumerate}
\end{prob}
  
