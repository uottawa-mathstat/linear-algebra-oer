\section*{Exercices}
\addcontentsline{toc}{section}{Exercices}



 \begin{prob} \label{prob06.1} Justifiez vos réponses aux questions suivantes :
\medskip
\begin{enumerate}[a)]
\item Le vecteur $(1,2)$ est-il une combinaison linéaire de $(1,0)$ et $(1,1)$ ?
\medskip

\item\sov~Le vecteur $(1,2)$ est-il une combinaison linéaire de $(1,1)$ et $(2,2)$ ?
\medskip

\item Le vecteur $(1,2)$ est-il une combinaison linéaire de $(1,0)$ et $(2,2)$ ?
\medskip

\item\sov~Le vecteur $(1,2,2,3)$ est-il une combinaison linéaire de $(1,0,1,2)$ et $(0,0,1,1)$ ?
\medskip

\item Le vecteur $(1,2,2,4)$ est-il une combinaison linéaire de $(1,0,1,2)$ et $(0,0,1,1)$ ?
\medskip

\item\sov~La matrice $\bmatrix 1&2\\2&3\endbmatrix $ est-elle une combinaison linéaire de $\bmatrix 1&0\\1&2\endbmatrix $ et de $\bmatrix 0&1\\0&1\endbmatrix $ ?
\medskip

\item La matrice $\bmatrix 1&2\\2&4\endbmatrix $ est-elle une combinaison linéaire de  $\bmatrix 1&0\\ 1&2\endbmatrix$ et de $ \bmatrix 0&1\\ 0&1\endbmatrix $ ?
\medskip

\item\sov~Le polynôme $1+x^2$ est-il une combinaison linéaire de $1+x-x^2$ et $x$ ?
\medskip

\item Le polyn\^ome $1+x^2$ est-il une combinaison linéaire de  $1+x$ et $1-x$ ?
\medskip

\item\sov~La fonction $\sin x$ est-elle une combinaison linéaire de la fonction constante $1$ et $\cos x$ ?
\medskip

\item La fonction $\sin^2 x$ est-elle une combinaison linéaire de la fonction constante $1$ et $\cos^2 x$ ?
\medskip

\item\sov~Si $\uu$, $\vv$ et $\ww$ sont des vecteurs quelconques d'un espace vectoriel $V$, le vecteur $\uu-\vv$ est-il une combinaison linéaire des vecteurs $\uu$, $\vv$ et $\ww$ ?
\medskip

\item Si $\uu$, $\vv$ et $\ww$ sont des vecteurs quelconques d'un espace vectoriel $V$, le vecteur $\ww$ est-il toujours une combinaison linéaire des vecteurs $\uu$ et $\vv$ ?
\medskip


\end{enumerate}

\end{prob} \begin{prob} \label{prob06.2}  Justifiez vos réponses aux questions suivantes.
\medskip
\begin{enumerate}[a)]
\item Le vecteur $(3,4)$ appartient-il à $\sp{(1,2)}$ ? \medskip

\item\sov~Est-ce que l'énoncé «  $(3,4) \in\sp{(1,2)} $  » est vrai ? (Notez que c'est la même question que dans la question précédente, mais écrite en utilisant la notation mathématique).
\medskip


\item Est-ce que l'énoncé «  $(2,4) \in \sp{(1,2)}$  » est vrai ?
\medskip


\item\sov~Combien de vecteurs appartiennent à $\sp{(1,2)}$ ?
\medskip

\item Les sous-ensembles $\set{(1,2)}$ et $\sp{(1,2)}$ sont-ils égaux ?
\medskip

\item\sov~Est-ce que $\set{(1,2)}$ est un sous-ensemble de $\sp{(1,2)}$ ?
\medskip

\item
\medskip Si $\vv$ est un vecteur non nul d'un espace vectoriel, montrez qu'il existe une infinité de vecteurs dans $\sp{\vv}$. (Est-ce aussi vrai si $\vv=\zero$ ?)\medskip

\item\sov~Considérons $S$ un sous-ensemble d'un espace vectoriel $V$.  Si $S= \text{Vect}\, S$, expliquez pourquoi $S$ doit alors nécessairement être un sous-espace de $V$.
\medskip


\item\sov~ Si $S$ est un {\it sous-espace} de $V$, montrez que $S= \text{Vect}\, S$ (c'est la réciproque de la question précédente).\\
Notez que pour montrer que $S= \text{Vect}\, S$, vous devez établir deux choses :
\begin{enumerate}[(i)]
\item Si $\ww\in S$, alors $\ww\in \text{Vect}\, S$,

\item Si $\ww\in \text{Vect}\, S$, alors $\ww\in S$.
\end{enumerate}
\medskip


\end{enumerate}

\end{prob} \begin{prob} \label{prob06.3} Donnez \underbar{deux} ensembles distincts {\it finis} g\'en\'erateurs pour chacun des sous-espaces suivants.
\medskip

\begin{enumerate}[(a)]

\item  $\set{(2x, x) \in \R^2\st x\in \R}$.\medskip
% no

\item\sov~ $\set{(x, y) \in \R^2\st 3x - y=0 }$. \medskip




\item  $\set{(x, y, z) \in \R^3\st x+y-2z=0 }$.   \medskip


\item\sov~ $\set{(x, y, z, w) \in \R^4\st x-y+z-w=0 }$.   \medskip


\item  $\Bigg\{  \bmatrix a&b\\ c&d\endbmatrix \in \M_{2 \,2}(\R) \;\Bigg|\; b=c\Bigg\}$.\medskip \medskip

\item\sov~ $\Bigg\{  \bmatrix a&b\\ c&d\endbmatrix \in \M_{2 \,2}(\R) \;\Bigg|\;a=d=0\quad \&\quad b=-c  \Bigg\}$.\medskip \medskip

\item  $\Bigg\{  \bmatrix a&b\\ c&d\endbmatrix \in \M_{2 \,2}(\R) \;\Bigg|\; a+d=0\Bigg\}$. \medskip


\item\sov~ $\Bigg\{  \bmatrix a&0\\ 0&b\endbmatrix \in \M_{2 \,2}(\R) \;\Bigg|\;  a, b \in \R\Bigg\}$. \medskip



\item  $\Bigg\{  \bmatrix 0&b\\ -b&0\endbmatrix \in \M_{2 \,2}(\R) \;\Bigg|\; b \in \R\Bigg\}$.      \medskip


\item\sov~ $\Bigg\{  \bmatrix a&b\\ c&d\endbmatrix \in \M_{2 \,2}(\R) \;\Bigg|\; a+b+c+d=0\Bigg\}$.      \medskip

\item  $ \PP_2=\set{p\st p \text{ est un polyn\^ome avec } \deg(p)\le 2} $.  \medskip

\item\sov~ $ \PP_n=\set{p\st p \text{ est un polyn\^ome avec } \deg(p)\le n} $.  \medskip

\item  $ \set{p \in \PP_2 \st  p(2)=0}$.  \medskip


\item\sov~ $ \set{p \in \PP_3 \st  p(2)=p(3)=0}$.  \medskip



\item  $ \set{p \in \PP_2 \st  p(1)+p(-1)=0}$.      \medskip

\item\sov~ $ \sp{\sin x, \cos x}$.      \medskip

\item  $ \sp{1, \sin x, \cos x}$.      \medskip

 \item\sov~ $ \sp{1, \sin^2 x, \cos^2 x}$.      \medskip


\item$^\ast$  $\set{(x, x-3) \in \R^2\st x\in \R}$ muni des \underbar{\it opérations non-standards suivantes:} \\
Addition : $$(x,y) \tilde+ (x',y')=(x+x', y+y +3).$$
Multiplication par scalaire: pour $k\in \R$,
$$k \odot (x,y)=(kx, ky+3k-3).$$

\item$^\ast$ $\set{(x, y, z) \in \R^3\st x+2y+z=2 }$ ; $V=\R^3$ muni des \underbar{\it opérations non-standards suivantes:} \\
Addition : $$(x,y,z) \tilde+ (x',y',z')=(x+x', y+y,z+z'-2).$$
Multiplication par scalaire: pour $k\in \R$,  $k\odot (x,y,z)=(kx, ky, kz-2k+2)$.\medskip

\end{enumerate}
{\it Indication pour les parties (m) et (n) :  si $p$ est un polynôme en la variable $x$ de degré au moins 1 et que $p(a)=0$ pour un certain $a \in \R$, alors nécessairement $x-a$  est un facteur de $p$, c'est-à-dire que $p$ s'écrit $p(x)=(x-a)q(x)$ où $q$ est un polynôme tel que $\deg(q)=\deg(p)-1$.}


\end{prob} \begin{prob} \label{prob06.4} Justifiez vos réponses aux questions suivantes :
\medskip
\begin{enumerate}[a)]
\item Soient $\uu$ et $\vv$ des vecteurs appartenant à un espace vectoriel $V$. Montrez soigneusement que $\sp{\uu,\vv}=\sp{\uu, \uu+\vv}$. Autrement dit, vous devez montrer deux choses :
\begin{enumerate}[(i)]
\item si $\ww\in \sp{\uu,\vv}$, alors $\ww \in \sp{\uu, \uu+\vv}$;
\item si $\ww\in \sp{\uu, \uu+\vv}$, alors $\ww\in \sp{\uu,\vv}$.
\end{enumerate}
\medskip

\item\sov~Soient $\uu$ et $\vv$ des vecteurs appartenant à un espace vectoriel $V$. Montrez soigneusement que $\sp{\uu,\vv}=\sp{\uu-\vv, \uu+\vv}$. Autrement dit, vous devez montrer que :
\begin{enumerate}[(i)]
\item si $\ww\in \sp{\uu,\vv}$, alors $\ww \in \sp{\uu-\vv, \uu+\vv}$;
\item si $\ww\in \sp{\uu-\vv, \uu+\vv}$, alors $\ww\in \sp{\uu,\vv}$.
\end{enumerate}
\medskip


\item Considérons $\uu  \in \sp{\vv,\ww}$. Montrez soigneusement que $\sp{\vv,\ww}=\sp{\uu,\vv,\ww}$.
\medskip


\item\sov~Réciproquement, supposons que $\sp{\vv,\ww}=\sp{\uu,\vv,\ww}$. Montrez soigneusement que $\uu \in \sp{\vv,\ww}$.
\medskip

\item Montrez que $x^2 \notin \sp{1,x}$.\footnote{\it Indication : raisonnez par l'absurde. Supposez que $x^2 \in \sp{1,x}$ puis \'ecrivez explicitement ce que cela signifie et ensuite ré-écrivez l'équation sous la forme $q(x)=0$, où $q(x)$ est un polynôme quadratique quelconque. Maintenant, rappelez-vous : tout polynôme non nul de degré 2 a au plus 2 racines distinctes. Revenez de nouveau à l'équation $q(x)=0$ et regardez combien de racines cette équation indique que $q$ possède. Trouvez maintenant votre contradiction et concluez.}
\medskip

\item\sov~Montrez que $x^{n+1} \notin \PP_n$. \footnote{\it Indication : généraliser l'idée de l'indice précédent. Rappelez-vous que tout polynôme non nul de degré $n+1$ a au plus $n+1$ racines distinctes.}
\medskip

\item$^\ast$ Montrez que $\PP$ n'admet aucun ensemble fini de g\'en\'erateurs. \footnote{\it Indication : raisonnez par l'absurde à nouveau. Supposez qu'il y ait un tel ensemble fini générateur, alors $\PP$ s'écrirait $\PP=\sp{p_1, \dots, p_k}$ pour certains polynômes $p_1, \dots, p_k$. Définissez ensuite $n=\max\set{\deg(p_1), \dots, \deg(p_k)}$ puis montrez que $x^{n+1}\notin\sp{p_1, \dots, p_k}$ en utilisant le même argument que dans l'indice précédent. Mais bien sûr $x^{n+1} \in \PP$, et donc la contradiction. }
\medskip

\item\sov~Supposons pour le moment (nous le prouverons plus tard) que si $W$ est un sous-espace d'un espace vectoriel $V$ et que $V$ est engendr\'e par un ensemble fini de vecteurs, alors $W$ est aussi engendré par un nombre fini de vecteurs.

Utilisez ce fait et la question précédente pour montrer que l'espace vectoriel $\F(\R)$ n'admet aucun ensemble fini de g\'en\'erateurs.
\medskip

\end{enumerate}


\end{prob}
