   
\section*{Exercices}
\addcontentsline{toc}{section}{Exercices}

La solution des exercices marqués par un astérisque $\star$ se trouvent à la fin du livre. Essayez-les d'abord avant de consulter la solution.


\begin{prob}
\label{prob02.1}
Donnez l'expression du vecteur nul de $\R^2$, de $\R^3$ et de $\R^5$.
\end{prob}

\begin{prob}
\label{prob02.2}
Montrez que $\uu + (\vv + \ww) = (\uu + \vv) + \ww$ pour tous $\uu,\vv,\ww\in\R^3$.
\end{prob}
 

\begin{prob}
\label{prob02.3}\sov~Soient $A=(1,\ 2,\ 3),\ B=(-5,\ -2,\ 5)$ et $C=(-2,\ 8,\ -10)$,
et soit $D$ le milieu du segment $\overline{AB}$. Trouvez les coordonnées
du milieu du segment $\overline{CD}$.
% $(-2,\ 4,\ -3)$

\end{prob}
\begin{prob}
\label{prob02.4} Résoudre les questions suivantes en utilisant le produit scalaire.\medskip
\begin{enumerate} [a)]
\item Trouvez toutes les valeurs $k\in\R$ telles que les vecteurs $(k,\ k,\ 1)$ et $(k,\ 5,\ 6)$ sont orthogonaux.\medskip % -3 et -2
\item\sov ~Trouvez l'angle entre les vecteurs $ (0,\ 3,\ -3)$ et $ (-2,\ 2,\ -1)$.\medskip % $\pi/4 
\item Si $A=(2,\ 4,\ 1),\ B=(3,\ 0,\ 9)$ et $C=(1,\ 4,\ 0)$, trouver l'angle $ \angle BAC$.  \medskip
% $3\pi/4

\end{enumerate}

\end{prob}

\begin{prob}
\label{prob02.5}\sov~Résoudre les problèmes suivants. \medskip
\begin{enumerate}[a)]

\item Si $\uu=(2,\ 1,\ 3)$ et $\vv=(3,\ 3,\ 3)$, trouvez
 $\proj_{\vv}{\uu}$.  \medskip
%${{2}\over{3}}(3,\ 3,\ 3)$.
\item Si $\uu=(3,\ 3,\ 6)$ et $\vv=(2,\ -1,\ 1)$, trouver
$\|\proj_{\vv}{\uu}\|$. \medskip
%$(3\sqrt6)/2
\item Trouver l'angle entre le plan d'équation cartésienne $x-z=7$ et le plan d'équation $y-z=234$.
\medskip

\end{enumerate}

\end{prob}



 

 
 