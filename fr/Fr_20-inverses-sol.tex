

\begin{sol}{prob18.1} Pour chacune des matrices suivantes, si elle est inversible trouvez son inverse, sinon justifiez qu'elle n'est pas inversible.
\medskip

(b) $A=\bmatrix1&1&0\\
0&-1&-2\\
0&2&3\endbmatrix$.

\soln  $A^{-1}=\bmatrix 1 & -3 & -2 \\0 & 3 & 2 \\0 & -2 & -1 \endbmatrix$.
\medskip

(d) $B=\bmatrix  1&x\\ -x&1\endbmatrix$.
\medskip

\soln  $B^{-1}=\bmatrix\frac{1}{x^2+1} & -\frac{x}{x^2+1} \\\frac{x}{x^2+1} & \frac{1}{x^2+1} \endbmatrix$.

 
\end{sol}

\bigskip
\begin{sol}{prob18.2}  Pour chacun des énoncés suivants, indiquez s'il est (toujours) vrai ou s'il est (possiblement) faux.   
   \smallskip    
\begin{enumerate}[$\bullet$]
\item Si vous dites que l'\'enonc\'e est faux, donnez un contre-exemple.   
\item Si vous dites que l'\'enonc\'e est vrai, donnez une explication claire, en citant un théorème ou en donnant une {\it preuve valide dans tous les cas}. 
\end{enumerate}


\medskip
(b)    Si $A^2=0$ pour une matrice $A$ de taille $n\times n$, alors $A$ n'est pas inversible.

\soln Vrai! Supposons au contraire que $A$ soit inversible, d'inverse $A^{-1}$. En multipliant les deux côtés de l'équation $A^2=0$ par la matrice $A^{-2}$, on obtient l'équation $I_n=0$, qui est évidemment absurde. Donc $A$ n'est pas inversible.
\medskip
 

(d)  Si $A$ est inversible, alors la MER de $A$ admet une ligne nulle.

\soln Faux. Par exemple, la matrice $I_2 =\scriptsize\bmatrix 1&0\\0&1\endbmatrix$ est inversible et elle est sous forme MER, mais elle n'a pas de ligne de zéros. (Plus généralement, on sait qu'une matrice $A$  de taille $n \times n$ est inversible ssi sa MER est $I_n$, laquelle n'a pas de ligne de zéros, donc l'énoncé est toujours faux.)
\medskip
 


(f) Si $ A $ est une matrice non-inversible $ n\times n$, alors $A\xx=\bb$ est incompatible pour tout $\bb \in \R^n$.

\soln Faux! Par exemple, pour la matrice $A=\scriptsize\bmatrix 1&0\\0&0
\endbmatrix$ et le vecteur $\bb=\scriptsize\bmatrix 1\\0
\endbmatrix$, on a que $A$ n'est pas inversible et que cependant $A\xx=\bb$ est compatible. En fait, cette équation admet même une  infinité de solutions, \`a savoir $\xx=\scriptsize\bmatrix 1\\s
\endbmatrix$ pour $s\in \R$.
\medskip
 

(h) Si $A$ est une matrice $ n\times n$  satisfaisant $A^{3}-3A^{2}+I_{n}=0$, alors $A$ est inversible et on a  $A^{-1}=3A-A^{2}$.

\soln Vrai! R\'e-\'ecrivez l'\'equation $A^{3}-3A^{2}+I_{n}=0$ en $3A^{2}-A^3=I_{n}$ puis factorisez par $A$ le côté gauche pour avoir $$A(3A-A^2)=I_{n}\,.$$  
Cette derni\`ere \'equation montre que $A$ est inversible et que $A^{-1}=3A-A^{2}$.
\medskip

\end{sol}

