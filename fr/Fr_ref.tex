
\centerline{\bf R\'ef\'erences}
  \bigskip
Parmi les nombreuses bonnes références sur l'algèbre linéaire, citons les suivantes :

\begin{enumerate}[]
\item {\it Elementary Linear Algebra}, Howard Anton.
 \smallskip

\item {\it  Linear Algebra}, Tom M. Apostol.
 \smallskip
 \item {\it Linear Algebra with Applications}, Otto Bretscher. 
\smallskip
\item {\it Alg\'ebre lin\'eaire et applications}, David Lay, Steven R.~Lay, Judi J.~McDonald ($5^e$ édition, Pearson.)
 \smallskip
 \item {\it Elementary Linear Algebra}, W.~Keith Nicholson (Second Edition, McGraw Hill, 2003.)
\smallskip
\item {\it Linear Algebra (Schaum's outlines)}, Seymour Lipschutz and Marc Lipson.
\smallskip
\item {\it Introduction à l'algèbre linéaire}, Paul A.~Martel, Ginette Ouellette. (Modulo)
\smallskip 
\item {\it Linear Algebra with Applications}, W.~Keith Nicholson. (Sixth Edition, McGraw Hill.)
\smallskip
 \item {\it Linear Algebra and Its Applications}, Gilbert Strang. (Fourth Edition, Wellesley-Cambridge Press.)
\smallskip
\item {\it Introduction to Linear Algebra}, Gilbert Strang.


\end{enumerate}

\vspace{1cm}
Pour ceux qui souhaitent une approche plus théorique, vous pouvez essayer ce qui suit :

\begin{enumerate}[]
\item {\it Linear Algebra}, Kenneth M. Hoffman and Ray Kunze. (Second edition, Pearson)
 \smallskip
 \item {\it Linear Algebra}, Sterling K. Berberian. 
\smallskip
 \item {\it Introduction to Linear Algebra}, Serge Lang.


\end{enumerate}
