\section*{Exercices}
\addcontentsline{toc}{section}{Exercices}


\begin{prob} \label{prob18.1} Pour chacune des matrices suivantes, si elle est inversible trouvez son inverse, sinon justifiez qu'elle n'est pas inversible.
\medskip
\begin{enumerate}[a)]
\item $\bmatrix 1&1&-1\\
0&1&1\\
0&0&1\endbmatrix$.
\medskip
% $\bmatrix 1 & -1 & 2 \\0 & 1 & -1 \\0 & 0 & 1 \endbmatrix$
\item\sov~$\bmatrix1&1&0\\
0&-1&-2\\
0&2&3\endbmatrix$.
\medskip
% $\bmatrix 1 & 0 & 0 \\0 & 3 & 2 \\0 & -2 & -1 \endbmatrix$
\item $\bmatrix 1 & 2 & 2\\ 1 & 3 & 1\\ 1 & 3 & 2\endbmatrix$.
\medskip
%
\item\sov~ $\bmatrix  1&x\\ -x&1\endbmatrix$.
\medskip
% $\bmatrix\frac{1}{x^2+1} & -\frac{x}{x^2+1} \\\frac{x}{x^2+1} & \frac{1}{x^2+1} \\endbmatrix$
\item 
\medskip $\bmatrix
1&1&1\\ 1&2&1\\2&3&2 \endbmatrix$.

\end{enumerate}

 
\end{prob} \begin{prob} \label{prob18.2}  Pour chacun des énoncés suivants, indiquez s'il est (toujours) vrai ou s'il est (possiblement) faux.   
   \smallskip    
\begin{enumerate}[$\bullet$]
\item Si vous dîtes que l'\'enonc\'e est faux, donnez un contre-exemple.   
\item Si vous dîtes que l'\'enonc\'e est vrai, donnez une explication claire, en citant un théorème ou en donnant une {\it preuve valide dans tous les cas}. 
\end{enumerate}

\begin{enumerate}[a)]
\medskip
\item Si $A$ est inversible et que $AB=0$, alors $B=0$.
\medskip


\item\sov~  Si $A^2=0$ pour une matrice $A$ de taille $n\times n$, alors $A$ n'est pas inversible.
\medskip

\item Si $A^2=0$ pour une matrice $A$ de taille $n\times n$, alors $\rank A<n$.
\medskip

\item\sov~ Si $A$ est inversible, alors la MER de $A$ admet une ligne nulle.
\medskip

\item Si $ A $ est une matrice inversible $ n\times n$, alors $A\xx=\bb$ est compatible pour tout $\bb\in\R^n$.
\medskip

\item\sov~ Si $ A $ est une matrice non-inversible $ n\times n$, alors $A\xx=\bb$ est incompatible pour tout $\bb \in \R^n$.
\medskip

\item Si $ A $ est une matrice non-inversible $ n\times n$, alors $A\xx=\zero$ admet une solution unique.
\medskip

\item\sov~ Si $A$ est une matrice $ n\times n$  satisfaisant $A^{3}-3A^{2}+I_{n}=0$, alors $A$ est inversible et on a  $A^{-1}=3A-A^{2}$.
\medskip

\end{enumerate}

\end{prob}

