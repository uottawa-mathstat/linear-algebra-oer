

\begin{sol}{prob02.3} Si $A=(1,\ 2,\ 3),\ B=(-5,\ -2,\ 5),\ C=(-2,\ 8,\ -10)$ 
et si $D$ est le milieu du segment $\overline{AB}$, alors trouvez les coordonnées du milieu du segment $\overline{CD}$.

\soln Le vecteur $\vv$ qui point vers le milieu du segment $\overline{CD}$ est $\vv=(C+D)/2$. Mais comme $D=(A+B)/2$, on a 
$\vv=(C+(A+B)/2)/2=(-2,\ 4,\ -3)$.
\medskip

\end{sol} 

\bigskip
\begin{sol}{prob02.4} Résolvez les problèmes suivants en utilisant le produit scalaire.


\medskip
(b) Trouvez l'angle entre les vecteurs $ (0,\ 3,\ -3)$ et $ (-2,\ 2,\ -1)$.  \medskip

\soln Si $\theta$ est l'angle entre ces deux vecteurs, alors $$\cos \theta=\dfrac{(0,\ 3,\ -3)\cdot(-2,\ 2,\ -1)}{\| (0,\ 3,\ -3)\| \| (-2,\ 2,\ -1)\|}=\dfrac{9}{\sqrt{18}\sqrt{9}}=\dfrac{\sqrt{2}}{2}.$$ D'o\`u $\theta= \dfrac{\pi}{4}$.
% $\pi/4$ 
\medskip


\end{sol} 

\bigskip
\begin{sol}{prob02.5}  Résoudre les problèmes suivants. \medskip


(a) Si $\uu=(2,\ 1,\ 3)$ et $\vv=(3,\ 3,\ 3)$, trouvez
 $\proj_{\vv}{\uu}$. 

\soln $$\proj_{\vv}{\uu}=\dfrac{\uu\cdot \vv}{\|\vv\|^2}\vv=\dfrac{(2,\ 1,\ 3)\cdot (3,\ 3,\ 3)}{\|(3,\ 3,\ 3)\|^2}(3,\ 3,\ 3)=\dfrac{18}{27}(3,\ 3,\ 3)=(2,2,2).$$ \medskip
%${{2}\N-over{3}}(3,\3,\3)$
\\


(b) Si $\uu=(3,\ 3,\ 6)$ et $\vv=(2,\ -1,\ 1)$, trouvez 
la longueur de la projection de $\uu$ le long de $\vv$. 

\soln
$$\|\proj_{\vv}{\uu}\|=\dfrac{|\uu\cdot \vv|}{\|\vv\||^2}\|\vv\|=\dfrac{|\uu\cdot \vv|}{\|\vv\|}=\dfrac{(3,\ 3,\ 6)\cdot (2,\ -1,\ 1)}{\|(2,\ -1,\ 1)\|}=\dfrac{ 3}{ 2}\sqrt{6}.$$
 
 \medskip
(c) Trouver l'angle entre les plans d'équations cartésiennes $x-z=7$ et $y-z=234$.

\soln L'angle entre deux plans est défini comme l'angle aigu entre leurs vecteurs normaux. On trouve donc l'angle $\varphi$ entre (leurs vecteurs normaux) ($1,0,1$) et $(0,1,-1)$ et on ajuste l'angle si nécessaire pour bien avoir un angle aigu (et pas obtus). L'angle $\varphi$ est $\dfrac{2\pi}{3}$ et donc la réponse est $\pi -\dfrac{2\pi}{3}=\dfrac{\pi}{3}$ après ajustement. 
%$\pi/3$.

\medskip
%$(3\sqrt6)/2$
 
  
\end{sol}