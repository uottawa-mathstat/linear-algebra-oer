
\begin{sol}{prob22.1} Trouvez les valeurs propres réelles des matrices suivantes. 
\medskip

(b)  $\bmatrix 
1&1&1\\1&1&-1\\1&1&2 \endbmatrix $.

\soln 

\begin{equation*}
\begin{split}
  \det \bmatrix 
1-\lam&1&1\\1&1-\lam&-1\\1&1&2-\lam \endbmatrix &= \det \bmatrix 
1-\lam&1&1\\2-\lam&2-\lam&0\\1&1&2-\lam \endbmatrix \quad (L_1+L_2\to L_2) \\
  &=(2-\lam)\det \bmatrix 
1-\lam&1&1\\1&1&0\\1&1&2-\lam \endbmatrix \quad (\text{Facteur de  }(2-\lam) \text{ in } L_2)\\
  &=(2-\lam)\det \bmatrix 
1-\lam&1&1\\1&1&0\\0&0&2-\lam \endbmatrix \quad (-L_2+L_3\to L_3)\\
  &=(2-\lam)^2 \det \bmatrix 
1-\lam&1\\1&1\endbmatrix \quad (\text{d\'eveloppement de Laplace selon }L_3)\\
  &= -\lam(2-\lam)^2\\
\end{split}\end{equation*}
Les valeurs propres sont donc $0$ et $2$. (La valeur propre $2$ a une multiplicité algébrique de 2.)

\medskip

(d) $\bmatrix 
1&0&1\\0&1&0\\1&1&1 \endbmatrix $.

\soln 
\begin{equation*}
\begin{split}
 \det  \bmatrix 
1-\lam&1&1\\0&1-\lam&0\\1&1&1-\lam \endbmatrix   &= (1-\lam)\det  \bmatrix 
1-\lam&1\\1&1-\lam \endbmatrix\quad (\text{d\'eveloppement de Laplace selon }L_3)\\
  &=(1-\lam)\{(1-\lam)^2-1\} \\
  &=(1-\lam)\lam(2-\lam)
\end{split}\end{equation*}
Donc les valeurs propres sont $0$, $1$ et $2$, et chacune a multiplicité algébrique $1$.
\medskip


(f) $\bmatrix 
2&1&0\\0&2&1\\0&0&2 \endbmatrix $.

\soln Comme cette matrice est une matrice triangulaire supérieure, ses valeurs propres sont les entrées de la diagonale (principale). Par conséquent, sa seule valeur propre est $2$, avec une multiplicité algébrique $3$.
\medskip

(h) $\bmatrix 
2&1&0\\0&2&0\\0&0&2 \endbmatrix$.

\soln Comme cette matrice est une matrice triangulaire supérieure, ses valeurs propres sont les entrées de la diagonale (principale). Par conséquent, sa seule valeur propre est $2$, avec une multiplicité algébrique $3$.
\medskip


\end{sol}

\bigskip
\begin{sol}{prob22.2}
Pour chacune des matrices de l'exercice précédent, trouvez une base pour chaque espace propre.

\soln 

\begin{enumerate}[]
\item[(b)] On a vu que les valeurs propres de $A=\bmatrix 
1&1&1\\1&1&-1\\1&1&2 \endbmatrix $ sont $0$ et $2$.  On a :

\begin{align*}E_0&=\ker(A-0I_3)\\
&=\ker A\\
&=\ker\bmatrix 
1&1&1\\1&1&-1\\1&1&2 \endbmatrix =\ker \bmatrix 
 1 & 1 & 0 \\
 0 & 0 & 1 \\
 0 & 0 & 0 \endbmatrix\\
& =\set{(-s,s,0)\st s\in \R}\\
&=\sp{(-1,1,0)}\,, \end{align*}
 donc $\set{(-1,1,0)} $ est une base de $E_0$, et :
\begin{align*}
E_2&=\ker(A-2I_3)\\
& =\ker\bmatrix 
-1&1&1\\1&-1&-1\\1&1&0 \endbmatrix \\
&=\ker \bmatrix 
 1 & 0 & -\frac{1}{2} \\
 0 & 1 & \frac{1}{2} \\
 0 & 0 & 0 \endbmatrix \\
&=\set{(\frac{s}2,-\frac{s}2,s)\st s\in \R}\\
&=\sp{(1,-1,2)}\,,\end{align*}
donc $\set{(1,-1,2)} $ est une base de $E_2$.

\medskip
\item[(d)] On a vu que les valeurs propres de $\bmatrix 
1&0&1\\0&1&0\\1&1&1 \endbmatrix $ sont $0$, $1$ et $2$. On a:

\begin{align*}
E_0&=\ker(A-0I_3)\\
&=\ker A=\ker\bmatrix 
1&0&1\\0&1&0\\1&1&1 \endbmatrix\\
& =\ker \bmatrix 
 1 & 0 & 1 \\
 0 & 1 & 0 \\
 0 & 0 & 0 \endbmatrix \\
&=\set{(-s,0,s)\st s\in \R}\\
&=\sp{(-1,0,1)}\, \end{align*} 
donc $\set{(-1,0,1)} $ est une base de $E_0$; et :
\begin{align*}
E_1&=\ker(A-I_3)\\ 
&=\ker\bmatrix 
0&0&1\\0&0&0\\1&1&0 \endbmatrix \\
&=\ker \bmatrix 
 1 & 1 & 0 \\
 0 & 0 & 1 \\
 0 & 0 & 0 \endbmatrix \\
&=\set{(-s,s,0)\st s\in \R}\\
&=\sp{(-1,1,0)}\,,\end{align*} 
donc $\set{(-1,1,0)} $ est une base de $E_1$; et :
\begin{align*}
	E_2&=\ker(A-2I_3) \\
	&=\ker\bmatrix
-1&0&1\\0&-1&0\\1&1&-1 \endbmatrix  \\
&=\ker \bmatrix 
 1 & 0 & -1 \\
 0 & 1 & 0 \\
 0 & 0 & 0  \endbmatrix \\
 &=\set{(s,0,s)\st s\in \R}\\
 &=\sp{(1,0,1)}\,,
 \end{align*}
donc $\set{(1,0,1)} $ est une base de $E_2$.
\medskip

\item[(f)]  On a vu que la seule valeur propre de $A=\bmatrix 
2&1&0\\0&2&1\\0&0&2 \endbmatrix $ est $2$. On a:
\begin{align*}
	E_2&=\ker(A-2I_3)\\
	&=\ker\bmatrix 
0&1&0\\0&0&1\\0&0&0\endbmatrix    \\
	&=\set{(s,0,0)\st s\in \R}\\
	& =\sp{(1,0,0)}\,,
\end{align*}
donc $\set{(1,0,0)} $ est une base de $E_2$.
\medskip
\item (h)  On a vu que la seule valeur propres de $A=\bmatrix 
2&1&0\\0&2&0\\0&0&2 \endbmatrix $ est $2$. On a:
\begin{align*}
	E_2 &= \ker(A-2I_3)\\
	&=\ker\bmatrix 
0&1&0\\0&0&0\\0&0&0\endbmatrix    \\
	& =\set{(s,0,t)\st s\in \R}\\
	& =\sp{(1,0,0), (0,0,1)}\,,
\end{align*} 
donc $\set{1,0,0), (0,0,1)} $ est une base de $E_2$.
\medskip
\end{enumerate}

\end{sol}

