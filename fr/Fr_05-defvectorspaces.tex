\chapter{Espaces vectoriels}
\label{chapter:Fr_04-vectorspaces}

Jusqu'à présent, nous avons vu qu'il n'est pas trop difficile de généraliser
certaines notions sur les \stress{vecteurs} de $\R^2$ ou $\R^3$ à l'espace plus général $\R^n$.
Avec cette idée en tête, nous appelons toujours \stress{vecteurs} les éléments de $\R^n$, même si nous ne pouvons plus vraiment
les dessiner avec une flèche dans l'espace.  

Dans le chapitre \ref{chapter:Fr_02-vectors} sur la géométrie vectorielle, nous avons aussi rencontrer quelques problèmes. Parmi les questions principales, on se demandait : quel est 
l'analogue des droites et des plans en dimensions supérieures ? Et comment les décrire ?

Dans ce chapitre, nous nous attaquons à ces questions, et pour ce faire, nous revenons un peu en arrière pour voir jusqu'à quel point
nous pouvons généraliser la notion d'\stress{espace vectoriel}. 
Est-ce que $\R^n$ est le seul objet qui se comporte comme $\R^2$ et $\R^3$, ou y a-t-il d'autres objets mathématiques
qui, si on les regarde de la bonne façon, se comportent également comme $\R^2$ et $\R^3$ ?

\section{Un premier exemple}
Jusqu'ici, on ne peut nier le fait que les éléments de $\R$, $\R^2$ et $\R^3$ sont des 
\defn{vecteurs géométriques}, puisqu'on peut les dessiner avec une flèche.
Nous sommes également d'accord que nous pouvons toujours appeler \defn{vecteurs} les éléments de $\R^n$ pour $n \geq 4$.

Cherchons maintenant des «~vecteurs~» qui ne sont pas dans $\R^n$ (et qui ne seront pas non plus géométriques).


\begin{myexample} {\bf L'espace des équations « linéaires »}
\label{exemple : l espace des equation lineaires}

Considérons les trois équations $E_1$, $E_2$ et $E_3$ suivantes :
\begin{alignat*}{2}
E_1 &: \quad  &x-y-z &= -1; \\
E_2 &: \quad &2x-y+z &= 1;\\
E_3 &: \quad &-x+2y+4z &=4.
\end{alignat*}
Nous pouvons obtenir une nouvelle équation à partir de ces trois équations, en calculant par exemple 
$$
E_4 = E_2-2E_1 \quad : \quad y+3z=3
$$
ou encore
$$
E_5=E_1 + E_3 \quad : \quad y+3z = 3\ .
$$
On peut même dire qu'alors 
$$
E_2-2E_1 = E_1 + E_3,
$$
et il est légitime de ré-écrire ceci comme 
$$
3E_1 -E_2 + E_3 = 0\, ,
$$
où «~$0$~» ici représente l'équation «~$0=0$~».

Mais que signifie réellement ceci ?
\begin{itemize}
\item On peut {\bf additionner} deux équations pour en obtenir une autre.
\item On peut {\bf multiplier} une équation {\bf par un scalaire} et avoir
une autre.
\item Il existe une {\bf équation nulle} donnée par «~$0=0$~».  Appelons-la disons
$E_0$.
\item Les équations ont des {\bf opposés} : l'opposé de $E_1$ est 
$$
-E_1 : -x+y+z=1.
$$
Pourquoi est-ce l'opposé ?  Simplement parce que $(E_1) + (-E_1) = E_0$.
\item Les « règles habituelles de l'arithmétique » sont respectées :
\begin{itemize}
\item $E_1 + E_2 = E_2+E_1$ («~commutativité~»);
\item $E_1 + (E_2+E_3) = (E_1 + E_2) + E_3$ («~associativité~»);
\item $k(E_1+E_2) = kE_1+kE_2$;
\item $(k+l)E_1 = kE_1 + lE_1$;
\item $(kl)E_1 = k(lE_1)$;
\item $1 E_1 = E_1$.
\end{itemize}
\end{itemize}


En d'autres termes, on retrouve {\bf exactement} les propriétés de $\R^n$ qu'on avait introduites dans le chapitre précédent!  
Donc, même si nous n'avons pas (encore) de moyen d'écrire les «~équations~» comme des $n$-uplets de nombres,
on peut voir que, algébriquement, elles se comportent vraiment comme des vecteurs.
\end{myexample}




\standout{Idée importante \#1 (de ce chapitre) : il existe en fait de nombreux espaces vectoriels autres que $\R^n$.}

Pour comprendre l'intérêt de notre méthode, donnons un peu plus de détails.

Considérons l'espace $\mathcal{E}$ de toutes les équations «~obtenues à partir~»   des  équations $E_1$, $E_2$ et $E_3$. Plus précisément, on dit que $\mathcal E$ est \stress{généré par} ou \stress{engendré par} ces trois équations, et il s'écrit :
$$
\mathcal{E} = \{ k_1E_1 + k_2 E_2 + k_3 E_3 \;| \; k_i \in \R\}.
$$
En fait, c'est l'ensemble des \emph{toutes} les combinaisons linéaires \index{combinaison linéaire} de $E_1$, $E_2$ et $E_3$, et on peut montrer que  $\mathcal{E}$ satisfait les crit\`eres d'un espace vectoriel dans le
sens où il vérifie tous les points de la liste ci-dessus (on reviendra sur une définition plus formelle dans la suite).

À présent, une question que l'on peut se poser est : étant donné $y_0\in\R$, est-ce que l'équation «~$y=y_0$~» est un élément de $\mathcal E$ ?
C'est-à-dire, est-ce que nous pouvons l'écrire comme combinaisons linéaire de $E_1$, $E_2$ et $E_3$ pour certains coefficients $k_1, k_2, k_3$ ?

\begin{myexample} Traitons une question plus facile :
pouvons-nous trouver $a,b,y_0\in\R$ de sorte que l'équation $y=y_0$ soit égale à $aE_1+bE_2$ ? \\ 
Réponse : NON, jamais.  Pourquoi ?\\
Dans l'équation $aE_1+bE_2$, le coefficient devant la variable $x$ est $a+2b$, et celui devant
$z$ est $-a+b$. On veut obtenir une équation de la forme «~$y=y_0$~» qui ne dépend que de la variable $y$, donc les coefficients devant $x$ et $z$ sont forcément nuls, c'est-à-dire $a+2b=0$ et $-a+b=0$. De ceci, on déduit que nécessairement $a=0$ et $b=0$ (vérifiez-le), ce qui donne
 $aE_1+bE_2 = E_0$, l'équation nulle. Donc on ne pourra jamais obtenir une équation de la forme $y=y_0$ à partir de $E_1$ et $E_2$ seulement... 
\end{myexample}

(Nous considérerons plus tard des questions plus difficiles,
mais la technique plus sophistiquée de l'\stress{élimination de Gauss-Jordan} qu'on introduira au chapitre~\ref{chapter:Fr_13-solvingsystems}
sera en fait entièrement basée sur cette idée de ne prendre que des combinaisons linéaires
d'équations).

Application : la \emph{stœchiométrie} est la science qui étudie comment
produire un certain composé chimique en faisant des réaction de composés chimiques 
 élémentaires. En d'autres termes, si l'on dénote $\mathcal{E}$
l'ensemble de toutes les réactions chimiques possibles, on peut dire que la stœchiométrie
cherche la meilleure combinaison (linéaire) de réactions de $\mathcal E$ qui donnera le résultat souhaité.

 
\standout {Idée importante \#2 (des deux prochains chapitres) : Pour répondre à certains problèmes, nous aurons besoin de
comprendre les notions de sous-espace vectoriel et d'ensemble générateur.}

\section{Alors, de quoi avons-nous vraiment besoin ?}
Idée :  Les \stress{vecteurs} n'ont pas besoin d'être géométriques ni même des $n$-upplets.
Cependant nous voulons avoir des propriétés similaires à celles connues sur les vecteurs.
(Notez que nous allons mettre en pause les propriétés géométriques (comme le produit scalaire) pour quelques temps et nous nous concentrons sur l'algèbre).

Nous avons donc besoin de : 
\begin{itemize}
\item un ensemble $V$, dont les éléments seront appelés «~vecteurs~»;
\item une loi pour l'« addition » de deux « vecteurs » (notée $+$);
\item une loi pour la « multiplication par un scalaire » d'un « vecteur » par un scalaire $c\in \R$ (notée $\cdot$ ou sans symbol),
\end{itemize}
tels que les 10 \defn{axiomes} suivants sont vérifiés :
\begin{description}
\item[\it Fermeture] (ces deux axiomes garantissent que $V$ est « suffisamment grand ») \\
(1) La somme de deux vecteurs doit être à nouveau un vecteur: si $\uu, \vv \in V$, alors $\uu + \vv \in V$.\\
(2) Le multiple scalaire d'un vecteur est un vecteur: si $\uu \in V$ et $c\in \R$, alors $c\,\uu \in V$.
\item[\it Existence] (ces deux axiomes garantissent que $V$ satisfait les crit\`eres élémentaires de l'algèbre)\\
(3) Il doit exister un vecteur nul $\zero$ dans $V$ qui satisfait
$\zero + \uu = \uu$ pour tout $\uu \in V$.\\
(4) Chaque vecteur de $V$ doit admettre un oppos\'e dans $V$: pour $\uu \in V$, il doit exister un vecteur $\vv \in V$  tel que $\uu + \vv = \zero$. On note habituellement $-\uu=\vv$ l'opposé de $\uu$.
\item[\it Crit\`eres d'arithmétique] (Ces axiomes garantissent que les opérations se comportent comme dans $\R^n$) \\
Pour tous $\uu, \vv, \ww \in V$ et tous $c,d\in\R$:\\
(5) $\uu + \vv = \vv + \uu$;\\
(6) $\uu + (\vv + \ww) = (\uu+\vv)+\ww$;\\
(7) $c(\uu + \vv) = c\uu + c\vv$;\\
(8) $(c+d)\uu = c\uu + d\uu$;\\
(9) $c(d\uu) = (cd)\uu$;\\
(10) $1\uu = \uu$.
\end{description}

\begin{definition}
Tout ensemble $V$ muni de deux lois $+$ et $\cdot$ satisfaisant ces 10 axiomes est
appelé \defn{espace vectoriel}.
\end{definition}

Notez qu'on peut aussi ajouter les deux formules suivantes pour tout vecteur $\uu\in V$ :
$$0\cdot\uu = \zero \quad\quad\text{et}\quad\quad -\uu = (-1)\cdot\uu,$$
mais ce ne sont pas des axiomes car on peut les obtenir à partir des 10 axiomes ci-dessus comme montré dans l'exercice \ref{prob04.14}.
 



\section{Exemples d'espaces vectoriels}

\begin{myexample} Munis des opérations habituelles $+$ et $\cdot$, les ensembles $\R$, $\R^2$, $\ldots$, $\R^n$ sont tous des espaces vectoriels. \end{myexample}

\begin{myexample} L'ensemble $\mathcal{E}$ de \stress{toutes} les équations linéaires en $n$ variables muni des opérations habituelles est un espace vectoriel. \end{myexample}

\begin{myexample} L'ensemble $V = \{\zero\}$ muni des lois suivantes 
$\zero + \zero = \zero$ et $c \cdot \zero = \zero$  est un espace vectoriel. C'est l'\emph{espace vectoriel nul}.
 (Attention : par contre, l'ensemble vide $\emptyset$ n'est pas un espace vectoriel car il ne contient pas $\zero$ et donc il contredit l'axiome $(3)$.) \end{myexample}

\begin{myexample} L'ensemble $V = \{(x,2x) | x \in \R\}$ muni des opérations standards
de $\R^2$ est un espace vectoriel car :
\begin{itemize}
\item FERMETURE (1) Soient $\uu, \vv \in V$.  On peut écrire $\uu = (x,2x)$ et $\vv = (y,2y)$ pour certains
$x\in \R$ et $y\in \R$. Il suit que $\uu + \vv = (x+y, 2x+2y) = (x+y, 2(x+y))$, qui est bein dans $V$ parce qu'il est de la forme $(z,2z)$ 
avec $z = x+y \in \R$.
\item FERMETURE (2) Soient $\uu = (x,2x)$ et $c\in \R$.  Alors $c\uu = (cx,c(2x))=(cx,2(cx))$ qui est à son tour dans $V$, car il a la forme $(z,2z)$ avec $z=cx\in \R$.
\item EXISTENCE (3) Le vecteur nul de $\R^2$ est $\zero = (0,0)$ et il est dans $V$ (prendre $x=0$). Il satisfait la condition $\zero + \uu = \uu$ pour tout $\uu \in\R^2$, donc a fortiori il vérifie la même condition pour tout $\uu \in V$ puisque $V\subseteq\R^2$.  D'où l'existence d'un vecteur nul dans $V$.
\item EXISTENCE (4) L'opposé de $\uu = (x,2x)$ est $-\uu=(-x,-2x) = (-x,2(-x))$, puisque $(x,2x)+(-x,2(-x))=(0,0)$, et on voit que $(-x,2(-x))$ est bien dans $V$.  D'où l'existence d'un oppos\'e dans $V$.
\item CRIT\`ERES D'ARITHM\'ETIQUE: comme on l'a vu dans les chapitres précédents, ces critères fonctionnent pour TOUS vecteurs $\uu, \vv, \ww$ de $\R^2$, donc ils fonctionnent encore  
pour TOUS vecteurs $\uu, \vv,\ww$ de $V$ puisque $V\subseteq \R^2$.
\end{itemize}
Ainsi $V$ est un bien espace vectoriel car il vérifie les 10 axiomes. \end{myexample}

\begin{myexample} L'ensemble $V = \{(x,x+2) | x\in \R\}$ muni des opérations habituelles dans $\R^2$ N'EST PAS un espace vectoriel. (Voir l'exercice \ref{prob04.1}, question (8), pour une variante avec une opération différente.)

Pour montrer que $V$ n'est pas un espace vectoriel, il suffit de donner UN SEUL contre-exemple parmi les 10 axiomes, m\^eme si l'axiome n'est faux que pour UN SEUL vecteur.
(Car dans un espace vectoriel, TOUS les axiomes doivent TOUJOURS être vrais et pour tout vecteurs.)

Ceci étant dit, analysons quand même tous les axiomes juste pour voir comment ceux-ci peuvent échouer de manières différentes.
 
\begin{itemize}
\item FERMETURE (1) Prenez $(x,x+2)$ et $(y,y+2)$ de $V$.  Alors leur somme 
$(x+y, x+y+4)$ n'a PAS la bonne forme $(z,z+2)$, par exemple prenez $z=0$.  En d'autres termes, lorsque vous ajoutez deux éléments de $V$, vous vous retrouvez à l'extérieur de $V$.  Cet ensemble n'est donc PAS FERMÉ pour l'addition, et l'axiome $(1)$ n'est pas vérifié.
\item FERMETURE (2) Prenez $c\in\R$ et $\uu = (x,x+2)\in V$.  Alors $c\,\uu = (cx, cx+2c)$, et pour $c\neq 1$ on a $c\uu \notin V$... Donc  $V$ n'est PAS FERMÉ pour la multiplication par scalaire, et l'axiome $(2)$ n'est pas vérifié.
\item EXISTENCE (3) Le vecteur nul $\zero=(0,0)$ de $\R^2$ n'est pas dans $V$, puisque 
on ne peut pas avoir $(0,0) = (z,z+2)$ quelque soit $z\in \R$. Plus généralement, on peut montrer que $V$ n'admet aucun vecteur nul, et donc l'axiome $(3)$ n'est pas vérifié.
\item EXISTENCE (4) L'oppos\'e de $\uu = (x,x+2)$ dans $\R^2$ est $-\uu = (-x,-x-2)$, mais il n'est pas un \'el\'ement de $V$. Plus généralement, on peut montrer qu'il n'y a pas l'existence de l'inverse pour tous les éléments de $V$, et donc l'axiome $(4)$ n'est pas vérifié.
\item CRIT\`ERES D'ARITHM\'ETIQUE (5)-(10) : 
ils sont tous vrais car $V$ est un sous-ensemble de $\R^2$.
\end{itemize} 
En fait, l'ensemble $V$ n'est pas un espace vectoriel car nous l'avons «~mal choisi~», il ne passe pas par le point $\zero\in\R^2$ et c'est en fait un \stress{sous-espace affine} de $\R^2$... (Voir sur internet pour plus d'informations).
\end{myexample}

\standout{Ce qu'il faut retenir : un sous-ensemble d'un espace vectoriel ne peut être un espace vectoriel
vectoriel que s'il contient le vecteur nul $\zero$. En particulier les droites et les plans qui ne passent pas
par l'origine \stress{ne sont pas des espaces vectoriels}.}

\begin{definition}
Une \defn{matrice} est un tableau de nombres. Elle est généralement notée entre crochets, et
on dit que sa taille est $m \times n$ si elle comporte $m$ lignes et
$n$ colonnes.  Par exemple, 
$$\mat{1 & 2 & 3 \\ 4 & 5 & 6}$$
est une matrice de $2 \times 3$.
Deux matrices de la même taille peuvent être additionnées composante par composante comme suit :
$$
\mat{1 & 2 \\ 3 & 4} + \mat{5 & 6 \\ 7 & 8} = \mat{6 & 8 \\ 10 & 12}\,,
$$
et nous pouvons aussi les multiplier par des scalaires en multipliant toutes ces composantes par ce scalaire :
$$
2\mat{1 & 2 \\ 3 & 4} = \mat{2 & 4 \\ 6 & 8}\,.
$$
Notez que deux matrices sont égales si, et seulement si, toutes leurs composantes sont deux-à-deux égales. Par exemple :
$$
\mat{1 & 2 \\ 3 & 4} \neq \mat{ 3 & 4 \\1 & 2} \,.
$$ 
\end{definition}









\begin{myexample}  L'ensemble $\displaystyle V = \mathcal M_{22}(\R) = \left\{ \mat{a & b \\ c & d} \Big| a,b,c,d \in \R \right\}$ muni des opérations ci-dessus est un espace vectoriel.
Vérifions les axiomes.
\begin{itemize}
	\item[(1),(2)]Les critères de \stress{fermeture} sont vérifiés car (1) la somme de deux matrices de taille $2\times 2$  est encore une matrice
$2\times 2$; et (2) les multiples scalaires d'une matrice $2\times 2$ sont encore des matrices  $2\times 2$.
	\item[(3),(4)]Les critères d'\stress{existence} sont aussi satisfaits: (3) le vecteur nul est $\zero = {\scriptsize\mat{0&0 \\ 0& 0}} \in M_{2 2}(\R)$ car $A+\zero=A$ pour toute matrice $A$; 
et (4) l'oppos\'e de $M={\scriptsize\mat{a&b\\c&d}}$ est $-M={\scriptsize\mat{-a&-b\\-c&-d}} \in M_{22}(\R)$ car lorsqu'on les additionne on obtient la matrice nulle.
\end{itemize}
Les critères d'arithmétique (5)-(10) découlent des propriétés déjà connues de
 $\R^4$. Il est donc facile de les vérifier, mais on inclut quand bien même les
preuves ci-dessous, avec parfois quelques techniques différentes.  
Considérons $\uu = {\scriptsize\mat{u_1 & u_2\\ u_3 & u_4}}$, $\vv = {\scriptsize\mat{v_1&v_2\\ v_3&v_4}}$ deux matrices, et soient $c,d\in \R$.  Alors :
\begin{itemize}
\item[(5)] $\uu + \vv = \mat{u_1+v_1 & u_2+v_2\\ u_3+v_3 & u_4+v_4}$ et
$\vv + \uu = \mat{v_1+u_1 & v_2+u_2\\ v_3+u_3 & v_4+u_4}$. D'où $\uu + \vv = \vv + \uu$.
\item[(6)]  $\uu + (\vv + \ww) = \mat{u_1 & u_2 \\ u_3 & u_4}+ \mat{v_1+w_1 & v_2+w_2\\ v_3+w_3 & v_4+w_4} = \mat{ u_1 + (v_1+w_1) & u_2+(v_2+w_2) \\ u_3 + (v_3+w_3) & u_4+(v_4+w_4)}$, et 
$(\uu + \vv) + \ww =  \mat{ (u_1 + v_1)+w_1 & (u_2+v_2)+w_2 \\ (u_3 + v_3)+w_3 & (u_4+v_4)+w_4}$. D'où $\uu + (\vv + \ww) = (\uu + \vv) + \ww$.
\item[(7)] On a bien $c(\uu + \vv) = c\uu + c\vv$:
\begin{align*}
c(\uu + \vv) &= c \mat{u_1+v_1 & u_2+v_2\\ u_3+v_3 & u_4+v_4} \\
&= \mat{c(u_1+v_1) & c(u_2+v_2)\\ c(u_3+v_3) & c(u_4+v_4)} \\
&= \mat{cu_1+cv_1 &  cu_2+cv_2\\ cu_3+cv_3 & cu_4+cv_4}\\
&= \mat{cu_1 & cu_2 \\ cu_3 & cu_4}+ \mat{cv_1&cv_2\\cv_3&cv_4}\\
&= c\uu + c\vv\,.
\end{align*} 
\item[(8)] On a bien $(c+d)\uu = c\uu + d\uu$ :
\begin{align*}
(c+d)\uu &= (c+d)\mat{u_1 & u_2 \\ u_3 & u_4}\\
&= \mat{(c+d)u_1 & (c+d)u_2 \\ (c+d)u_3 & (c+d)u_4}\\
&= \mat{cu_1+du_1 & cu_2+du_2 \\ cu_3+du_3 & cu_4+du_4}\\
&= \mat{cu_1 & cu_2 \\ cu_3 & cu_4}+\mat{du_1 & du_2 \\ du_3 & du_4}\\
&= c\uu + d\uu,
\end{align*}
\item[(9)] On veut montrer que $a(b\uu) = (ab)\uu$. Comparons la $i$-ème composante de chaque côté, pour $1 \leq i \leq 4$.  
À gauche, la $i$-ème composante est $a(b u_i)$, tandis qu'à droite elle est
$(ab)u_i$. Comme on compare ces valeurs dans $\R$ qui est un espace vectoriel, on en déduit qu'on a bien l'égalité des $i$-ièmes composantes, et par la suite on obtient l'égalité désirée $a(b\uu) = (ab)\uu$.
\item[(10)] La $i$-ième entrée de $1\uu$ est $1u_i$, qui est égal à la
$i$-ème entrée $u_i$ de $\uu$. D'où $1\uu = \uu$.
\end{itemize}
Ainsi $\mathcal M_{22}(\R)$ muni de ces opérations est un espace vectoriel.
\end{myexample}

\standout{Plus généralement, l'ensemble $\mathcal M_{m\,n}(\R)$ de toutes les matrices $m\times n$, muni d'opérations similaires à celles décrites ci-dessus, est également un \stress{espace vectoriel}, et ceci est vrai pour tout $m,n \geq 1$ !}



Notre prochain exemple est l'\stress{espace de fonctions}.  Nous établissons d'abord quelques notations.

\begin{definition}
Soit $[a,b]=\{x \in \R \,|\, a \leq x \leq b\}$ un intervalle.
On dénote par
$$
\F\big([a,b]\big) = \{ f \,|\, f \colon [a,b] \to \R\}
$$
 l'ensemble de toutes les fonctions définies sur $[a,b]$ et à valeurs dans $\R$.
Pour $f,g \in \F([a,b])$, on dit que $f=g$ si et seulement si $f(x)=g(x)$ 
pour tout $x\in [a,b]$.  Aussi, nous définissons $f+g$ comme la fonction
qui associe à $x$ la valeur $f(x)+g(x)$, c'est-à-dire :
$$
(f+g)(x) = f(x) + g(x)\,,
$$
et pour tout scalaire $c \in \R$, $cf$ est la fonction qui  à $x$ associe
$cf(x)$, c'est-à-dire :
$$
(cf)(x) = c(f(x)).
$$
Géométriquement, l'addition correspond à l'addition verticale des graphes de $f$ et de $g$, et la multiplication par scalaire correspond à la dilatation du graphe de $f$ par un coefficient $c$.
\end{definition}

\begin{myexample} {\bf Espace des fonctions}

Muni des ces deux lois, l'ensemble $\F([a,b])$ est un espace vectoriel.

La vérification est laiss\'ee au lecteur.   
(Les crit\`eres de fermeture sont vrais par définition.
Notez que le vecteur nul est donné par la fonction nulle $\zero$ qui associe à tout $x$ la valeur $0$ ; l'oppos\'e de $f$ est la fonction $-f$ qui à $x$ associe $-f(x)$ ; et que les axiomes d'arithmétique sont une conséquence de ceux sur $\R$.)
\end{myexample}

\begin{myexample} Nous pouvons également considérer $\F(\R)$, l'ensemble de toutes les fonctions
de $\R$ dans $\R$, muni des mêmes opérations. Il s'agit là encore d'un espace vectoriel !
Notez que par exemple $\cos(x)$ et $x+x^2$ sont dans $\F(\R)$, et toute fonction polynomiale est également dans $\F(\R)$. En revanche, des fonctions comme 
$\frac{1}{x}$ ou $\tan(x)$ ne sont pas dans $\F(\R)$ car elles ne sont pas
définies sur $\R$ tout entier...   
\end{myexample}

Dans plusieurs de nos exemples, les axiomes d'arithmétique (5)-(10) étaient automatiquement vérifiés 
car ils étaient vrais pour un «~plus grand~» ensemble de vecteurs. En d'autres termes, nous disposons d'un raccourci pour vérifier si
un sous-ensemble d'un espace vectoriel est aussi
un espace vectoriel (avec les \stress{mêmes opérations} !). Si c'est bien le cas, on l'appellera \stress{sous-espace vectoriel}.





