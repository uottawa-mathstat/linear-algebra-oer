
\section*{Problems}
\addcontentsline{toc}{section}{Problems}


    
 
\begin{prob}\label{prob03.1}  Solve the following problems using the cross and/or dot products.\medskip
\begin{enumerate}[a)]
\item If $\uu=(3,\ -1,\ 4)$ and $\uu=(-1,\ 6,\ -5)$, find
$\uu\times \vv$. \medskip
% $(-19,\ 11,\ 17)$
\item\sov Find all vectors in $\R^3$ which are orthogonal to both  $(-1, 1, 5)$ and $(2, 1, 2)$.  \medskip
% $\{(t,\ -4t,\ t)|\ t\in \R\}$
\item  If $\uu=(4,\ -1,\ 7),\ \vv=(2,\ 1,\ 2)$ and 
$\ww=(-1,\ -2,\ 3)$, find $(\uu\times \vv)\times \ww$. \medskip
% $(30,\ 21,\ 24)$
\item\sov If $\uu=(-4,\ 2,\ 7),\ \vv=(2,\ 1,\ 2)$ and 
$\ww=(1,\ 2,\ 3)$, find $\uu\cdot (\vv\times \ww)$. \medskip
% 17
\end{enumerate}

\end{prob}
\begin{prob}
\label{prob03.2}  Solve the following problems using the appropriate products.\medskip
\begin{enumerate}[a)]

\item Find the area of the parallelogram determined by the vectors
$\uu=(1,\ -1,\ 0)$ and $\vv=(2,\ -3,\ 1)$.    \medskip
%
\item\sov Find the area of the triangle with vertices $A=(-1,\ 5,\
0)$, $B=(1,\ 0,\ 4)$ and $C=(1,\ 4,\ 0)$.  \medskip
%6
\item Find the area of the triangle whose vertices are $P=(1,\
1,\ -1),\ Q=(2,\ 0,\ 1)$ and $R=(1,\ -1,\ 3)$.  \medskip
%\sqrt{5}


\item\sov Find the volume of the parallelepiped  determined by $\uu=(1,\ 1,\ 0),\ \vv=(1,\ 0,\ -1)$ and $\ww=(1,\ 1,\ 1)$.\medskip
% 1

\item Find the volume of the parallelepiped  determined by $\uu=(1, -2, 3),\ \vv=(1,3,1)$ and $\ww=(2,1,2)$.\medskip
% 10

\end{enumerate}



\end{prob} \begin{prob} \label{prob03.3}  Solve the following problems. \medskip
\begin{enumerate}[a)]
\item\sov Find the point of intersection of the plane with Cartesian equation $2x+2y-z=5$,
and the line with parametric equations $x=4-t,\ y=13-6t,\ z=-7+4t$.  \medskip
%$(2,\ 1,\ 1)$
\item\sov If $L$ is the line passing through $(1,\ 1,\ 0)$ and $(2,\
3,\ 1)$, find  the point of intersection of $L$ with the plane with Cartesian equation $x+y-z=1$. \medskip
% $(0,\ 1/2,\ -1/2)$
\item  Find the point where the lines with parametric equations $x=t-1$,   $y=6-t,\ z=-4+3t$ and $x=-3- 4t,\ y=6-2t,\ z=-5+3t$ intersect. \medskip
%(1,\ 4,\ 2)
\item\sov  Do the planes with Cartesian equations $2x-3y+4z=6$ and $4x-
6y+8z=11$ intersect? \medskip
% No
\item Find the line of intersection of the planes with Cartesian equations $5x+7y-4z=8$ and $x-y=-8$. \medskip
%$(-4,\ 4,\ 0)+ t (1,\ 1,\ 3),\, t\in \R$

\item\sov Find the line of intersection of the planes with Cartesian equations $x+11y-4z=40$ and $x -y=-8$. \medskip
%$(-4,\ 4,\ 0)+ t (1,\ 1,\ 3),\, t\in \R$

\end{enumerate}



\end{prob} \begin{prob} \label{prob03.4}  Solve the following problems. \medskip
\begin{enumerate}[a)]

\item  Find the distance from the point $(0,\ -5,\ 2)$ to the plane with Cartesian equation
$2x+3y+5z=2$. \medskip
%$7/\sqrt{38}$

\item\sov  Find the distance from the point $(-2,\ 5,\ 9)$ to the plane with Cartesian equation
$6x+2y-3z=-8$  \medskip
%3
\item Find the distance from the point $(5,\ 4,\ 7)$ to the
line containing the points $(3,\ -1,\ 2)$ and $(3,\ 1,\ 1)$.
\medskip
%



\item\sov  Find the distance from the point $(8,\ 6,\ 11)$ to the
line containing the points $(0,\ 1,\ 3)$ and $(3,\ 5,\ 4)$.  \medskip
% 7
 
\item Find angle between the planes with Cartesian equations $x-z=7$ and $y-z=234$
\medskip
%$\pi/3$




\end{enumerate}



\end{prob} \begin{prob} \label{prob03.5}  Find the scalar {\it and} vector parametric forms for  the following lines:
\medskip
\begin{enumerate}[a)]

\item The line containing $(3,\ -1,\ 4)$  and $(-1,\ 5,\ 1)$. \medskip
% 
\item\sov The line containing $(-5,\ 0,\ 1)$ and which is parallel to the two planes with Cartesian equations $2x-4y+z=0$ and  $x-3y-2z=1$ 
\medskip
%$x=-5+11t,\, y=5t,\, z=1-2t,\, t\in \R$
\item The line passing through  $(1,\ 1,\ -1)$ and which is perpendicular to the plane with Cartesian equation $2x-y+3z=4$ \medskip
%$x=1+2t,\, y=1-t,\, z=-1+3t,\, t\in \R$
\end{enumerate}

\end{prob} \begin{prob} \label{prob03.6}  Find a Cartesian equation for each of the following planes:

\medskip
\begin{enumerate}[a)]
\item The plane containing $(3, -1, 4)$, $(-1, 5, 1)$ and $(0, 2, -2)$.\medskip
% $2y - z = 3$

\item\sov  The plane parallel to the vector $(1, 1, -2)$  and containing  the points $(1, 5, 18)$ and $(4, 2, -6)$ \medskip 
%$5x - 3y + z = 8$

\item  The plane passing through the points  $(2, 1, -1)$ and $(3, 2, 1)$, and parallel to the $x$--axis. \medskip 
%
\item\sov  The plane containing the   two lines 
$ \set{(t-1,6-t,-4+3t)\st t \in \R}$ and 
 $ \set{(-3 -4t, 6+ 2t,  7+5t)\st \in \R} $    \medskip  
%$11x + 17y + 2z = 83$
\item The plane which contains the point (-1, 0, 2) and the line of intersection of the two planes $3x + 2y - z = 5$ and $2x + y + 2z = 1$.  \medskip  
%$23x + 12y + 19z = 15$
\item\sov The plane containing the point $(1, -1, 2)$  and the line \hfill \\$\set{(4, -1 + 2t,2 + t)\st t \in \R}$. \medskip
% $y - 2z + 5 = 0$
\item The plane through the origin parallel to the two vectors  $(1, 1, -1)$ and $(2, 3, 5)$.\medskip
%$8x - 7y + z = 0$
\item\sov The plane containing the point $(1,\ -7,\ 8)$ which is perpendicular to the line $\set{(2+2t,7-4t,-3+t\st  t\in \R}.$ \smallskip  
%$2x-4y+z=38$
\item The plane containing the point  $(2,\ 4,\ 3)$ and which is perpendicular to the planes with Cartesian equations $x+2y-z=1$  and $3x-4y=2$.  
% $4x+3y+10z=50$


\end{enumerate}
 

\end{prob} \begin{prob} \label{prob03.7}   Find a vector parametric form  for  the planes with Cartesian equations given as follows. (i.e. find some $a \in H$ and two non-zero, non-parallel vectors $\uu, \vv \in \R^3$, parallel to the plane $H$. Then  $H=\set{a+ s \uu + t \vv\st s,t \in \R}$.)\medskip
\begin{enumerate}[a)]

\item  $x - y - z = 3$\medskip 
%  
\item\sov $x - y - 2z = 4$ \medskip
% 
\item $2x - y + z = 5$ \medskip
% 
\item $y + 2x = -3$ \medskip
% 
\item $x - y + 2z = 0$ \medskip
% 
\item  $x + y + z = -1$\medskip
% 
\end{enumerate}


\end{prob} \begin{prob} \label{prob03.8}\sov  Let $\uu, \vv$ and $\ww$ be any vectors in $\R^3$.  Determine which  of the following statements could be false, and give an example to illustrate each of your answers.
 \medskip

(1)  $\uu\cdot \vv=\vv\cdot \uu$

(2)  $\uu\times\vv=\vv\times \uu$

(3)  $\uu\cdot(\vv+\ww)=\vv\cdot \uu+\ww\cdot \uu$

(4)  $(\uu+2\vv)\times \vv=\uu\times \vv$

(5)  $(\uu\times \vv)\times \ww=\uu\times(\vv\times \ww)$

% 2 \& 5

\end{prob} \begin{prob} \label{prob03.9}\sov  Let $\uu$, $\vv $ and $\ww $ be vectors in $\R^3$.  Which of the following
statements are (always) true? Explain your answers, including  giving examples to illustrate statements which could be false.
\medskip

(i)  $(\uu\times \vv)\cdot \vv=0$.
 

(ii)  $(\vv\times \uu)\cdot \vv=-1$.
 

(iii)  $(\uu\times\vv)\cdot \ww$ is the volume of the of the parallelepiped  determined by $\uu$, $\vv$ and $\ww$.
 

(iv)  $\Vert \uu\times \vv\Vert=\Vert \uu \Vert \, \Vert \vv \Vert\,\cos\theta$
where $\theta$ is the angle between $\uu$ and $\vv$.
 

(v)  $|\uu\cdot \vv|=\Vert \uu \Vert \, \Vert \vv \Vert\,\cos\theta$
where $\theta$ is the angle between $\uu$ and $\vv$.



\end{prob} \begin{prob} \label{prob03.10} Prove the identity $\uu \times (\vv\times \ww)= (\uu\cdot \ww) \vv -(\uu\cdot \vv) \ww$ for all $\uu,\vv,\ww \in \R^3$, as follows: Denote the difference between the left hand side of the identity  and the right hand side by $D(\uu,\vv,\ww)$.

First note that, by properties of the dot and cross products,  for all $k\in \R$, $\uu,\vv,\ww,\uu',\vv',\ww' \in \R^3$, it's easy to see that the following identities hold.
 
\medskip
\begin{enumerate}[(i)]
 
\item $D(k \uu +\uu', \vv,\ww) =k\, D(\uu,\vv,\ww) +D(\uu',\vv,\ww)$
\medskip
%

\item  $D(\uu , k \vv +\vv',\ww) =k\,D(\uu,\vv,\ww)  + D(\uu,\vv',\ww)$
\medskip
%
\item $D(\uu, \vv,k \ww +\ww') =k\, D(\uu , \vv,\ww)+ D(\uu, \vv,\ww')$
\medskip
%
\item Also: $D(\uu, \vv,\ww) =-D(\uu,\ww,\vv) $
\medskip
%
\end{enumerate}

(Properties (i)--(iii) are summarized by saying that ``{\it $D$ is  linear in each argument}''. More on this when we get to linear transformations.)
\medskip 

Now since every vector in $\R^3$ is a linear combination of $\hat i= (1,0,0), \hat j =(0,1,0)$ and $\hat k=(0,0,1)$, by the identities above, it suffices to check that  $D(\uu,\vv,\ww)=0$ when $\uu$ is  $\hat i, \hat j$ and $\hat k$, and $\vv$ and $\ww$ are one the 6 pairs of two distinct choices from $\set{\hat i, \hat j, \hat k}$. Restricted to these (18) choices, it's easy to see   $D(\uu,\vv,\ww)$ is zero unless $\uu$ is $\vv$ or $\ww$, so we really only need to check $D(\uu,\uu,\ww)=0$ in the 6 cases where  $\uu\in  \set{\hat i, \hat j, \hat k}$ and   $\ww \in\set {\hat i, \hat j, \hat k}\setminus\set{\uu}$.   This amounts to 6 easy computations. Having checked those, you're done!  
  
\end{prob} 