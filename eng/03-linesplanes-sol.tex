\begin{sol}{prob03.1}  Solve the following problems using the cross and/or dot products.\medskip
  
 
(b) Find all vectors in $\R^3$ which are orthogonal to both  $(-1, 1, 5)$ and $(2, 1, 2)$.  \medskip

\soln Such vectors will be parallel to the cross product of these two vectors, which is found to be $(-3, 12, -3)$. Hence $\{(t,\ -4t,\ t)|\ t\in \R\}$ is the solution.
 
\medskip

 

(d) If $u=(-4,\ 2,\ 7),\ v=(2,\ 1,\ 2)$ and 
$w=(1,\ 2,\ 3)$, find $u\cdot (v\times w)$. \medskip

\soln This is $(-4,\ 2,\ 7)\cdot (2,\ 1,\ 2)\times (1,\ 2,\ 3)= (-4,\ 2,\ 7)\cdot (-1, -4, 3)=17$.
% 17
 \end{sol}
 
\begin{sol}{prob03.2}  Solve the following problems using the appropriate products.\medskip


(b) Find the area of the triangle with vertices $A=(-1,\ 5,\
0)$, $B=(1,\ 0,\ 4)$ and $C=(1,\ 4,\ 0)$ 

\soln This will be $\dfrac12$ the length of the cross product of the two vectors $B-A$ and $C-A$, and so is  $\dfrac12\|(4, 8, 8)\|=6$.  
%6

\medskip
(d) Find the volume of the parallelepiped  determined by $u=(1,\ 1,\ 0),\ v=(1,\ 0,\ -1)$ and $w=(1,\ 1,\ 1)$.

\soln This is simply the absolute value of $u \cdot v \times w$ and so is $1$.
% 1


\end{sol} \begin{sol}{prob03.3}  Solve the following problems. \medskip

(a) Find the point of intersection of the plane with Cartesian equation $2x+2y-z=5$,
and the line with parametric equations $x=4-t,\ y=13-6t,\ z=-7+4t$.  

\soln Once we substitute $x=4-t,\ y=13-6t$ and $ z=-7+4t$ into $2x+2y-z=5$, we can solve for $t$, which we then substitute backinto $x=4-t,\ y=13-6t$ and $ z=-7+4t$ to obtain $(2,\ 1,\ 1)$.\medskip
%$(2,\ 1,\ 1)$

\medskip

(b) If $L$ is the line passing through $(1,\ 1,\ 0)$ and $(2,\
3,\ 1)$, find  the point of intersection of $L$ with the plane with Cartesian equation $x+y-z=1$. 

\soln We find the (scalar) parametric equations for the $L$, and then proceed as in part (a) to obtain $(1/2,0,\ -1/2)$


\medskip

(d) Do the planes with Cartesian equations $2x-3y+4z=6$ and $4x-
6y+8z=11$ intersect? 

\soln No, their normals are parallel, so the planes are, but their equations are not multiples of the other.\medskip

\medskip


(f) Find the line of intersection of the planes with Cartesian equations $x+11y-4z=40$ and $x -y=-8$. 

\soln A direction vector for this line will be perpendicular to both normal vectors, and so can be obtained as the cross product of these normals, namely $(-4, -4, -12)$. (We will choose $(1,1,3)$ instead.) Now all we need to do is find one point on this line. This is done by substituting $x= y-8$ into $x+11y-4z=40$, and simplifying to obtain $3y-z=12$. Now set $z=0$, to obtain the point $(-4,\ 4,\ 0)$. Hence the line is $\set{(-4,\ 4,\ 0)+ t (1,\ 1,\ 3),\st  t\in \R}$\medskip
%$(-4,\ 4,\ 0)+ t (1,\ 1,\ 3),\, t\in \R$

 

\end{sol} 
\begin{sol}{prob03.4}  Solve the following problems. \medskip



(b)  Find the distance from the point $Q=(-2,\ 5,\ 9)$ to the plane with Cartesian equation
$6x+2y-3z=-8$ 

\soln Choose any point $P$ on the plane - say $P=(0,-4,0)$. The the distance between $Q$ and the plane is the length of the projection of $QP$ in the direction of the normal $(6,2,-3)$. This length is $\Big|\dfrac{(Q-P)\cdot(6,2,-3)}{\|(6,2,-3)\|}\Big|=3.$ \medskip

\medskip


(d)  Find the distance from the point $P=(8,\ 6,\ 11)$ to the
line containing the points $Q=(0,\ 1,\ 3)$ and $R=(3,\ 5,\ 4)$. 

\soln Draw a picture: this will be the smaller of the lengths of the vectors $(Q-P)\pm \proj_{R-Q}(Q-P)$. Alternatively, we could solve for $s$ in $0=(R-Q)\cdot(P-(Q+ s (R-Q))$ to find the point $S=Q+ s (R-Q)$ on the line closest to $P$, and then compute $\|P-S \|$. In either case, the answer is $7$. \medskip
% 7





\end{sol} 


\begin{sol}{prob03.5}  Find the scalar {\it and} vector parametric forms for  the following lines:
\medskip


(b)  The line containing $(-5,\ 0,\ 1)$ and which is parallel to the two planes with Cartesian equations $2x-4y+z=0$ and  $x-3y-2z=1$. 

\soln This line will be perpendicular to both normals, and so parallel to the cross product of the normals. Hence the scalar parametric form is $x=-5+11t,\, y=5t,\, z=1-2t,\, t\in \R$, and the vector parametric form is $(-5,\ 0,\ 1) + t(11,5,-2)$.\medskip

\end{sol} 
\begin{sol}{prob03.6}  Find a Cartesian equation for each of the following planes:

\medskip

(b)   The plane parallel to the vector $(1, 1, -2)$  and containing  the points $P=(1, 5, 18)$ and $Q=(4, 2, -6)$  
$5x - 3y + z = 8$. 

\soln This will be the plane through $P$ with normal parallel to the cross product of $(1, 1, -2)$ and $P-Q$. An easy computation then yields the equation $5x - 3y + z = 8$ (after dividing by 6).
\medskip

 

(d)  The plane containing the   two lines 
$ \set{(t-1,6-t,-4+3t)\st t \in \R}$ and  $ \set{(-3 -4t, 6+ 2t,  7+5t)\st \in \R} $     

\soln Pick a point on either line, say $P=(-1,6,-4)$. Then this will be the plane through $P$ with normal parallel to the cross product of the direction vectors of each line. An easy computation then yields the equation $11x + 17y + 2z = 83$.
\medskip 

 

(f) The plane containing the point $P=(1, -1, 2)$  and the line $\set{(4, -1 + 2t,2 + t)\st t \in \R}$.\soln Pick a  point on the line, say $Q=(4,-1,2)$. This will be the plane through $P$ with normal parallel to the cross product of $P-Q$ and a direction vector for the line (say) $(0,2,1)$. One obtains (after division by $\pm3$) the equation $y - 2z =- 5$.\medskip
% 

 

(h) The plane containing the point $P=(1,\ -7,\ 8)$ which is perpendicular to the line $\set{(2+2t,7-4t,-3+t\st  t\in \R}.$ \soln This will be the plane through $P$ with normal paralle to a direction vector for the line. One obtains the equation $2x-4y+z=38$.\medskip  
%$2x-4y+z=38$ 

\end{sol}

 \begin{sol}{prob03.7} Find a vector parametric form  for  the planes with Cartesian equations given as follows. (i.e. find some $a \in H$ and two non-zero, non-parallel vectors $u, v \in \R^3$, parallel to the plane $H$. Then  $H=\set{a+ s u + t v\st s,t \in \R}$.)\medskip

(b)  $x - y - 2z = 4$. 

\soln Take $a=(4,0,0) \in H$. To pick $u$ and $v$, simply choose two (non-zero, non-parallel) vectors perpendicular to a normal vector $(1,-1,-2)$. So  $u= (1,1,0)$  and $v=(2,0,1)$ will do. Then $H=\set{(4,0,0)+ s (1,1,0) + t (2,0,1)\st s,t \in \R}$. (There are of course infinitely many correct answers.)\medskip


\end{sol} 

\begin{sol}{prob03.8}  Let $u, v$ and $w$ be any vectors in $\R^3$.  Which  of the following statements could be false, and give an example to illustrate each of your answers.
 \medskip

(1)  $u\cdot v=v\cdot u$. \soln This is always true.

\medskip

(2)  $u\times v=v\times u$. \soln Since $u\times v=- v\times u$ always holds, this is only true if $u$ and $v$ are parallel or either is zero. So for a counterexample, take $u=(1,0,0)$ and $v=(0,1,0)$. Then $u\times v=(0,0,1)\not=(0,0,-1)=v \times u$

\medskip

(3)  $u\cdot(v+w)=v\cdot u+w\cdot u$. \soln This is always true. 

\medskip

(4)  $(u+2v)\times v=u\times v$. \soln This is always true, since $v\times v=0$ always holds.

\medskip

(5)  $(u\times v)\times w=u\times(v\times w)$. \soln This is almost always false. Indeed by the last question in this set of exercises, it is true only if $(u \cdot w)v- (u\cdot v) w = (w\cdot u) v- (w\cdot v) u \iff (u\cdot v) w = (w\cdot v) u$. So for a counterexample, take $u=(1,0,0)=v$ and $w=(0,1,0)$. Then $(u\times v)\times w=(0,0,0) \not=(0,-1,0)=u\times(v\times w)$.

% 2 \& 5

\end{sol}


 \begin{sol}{prob03.9}  Let $u , v $ and $w $ be vectors in $\R^3$.  Which of the following 
statements are (always) true? Explain your answers, including  giving examples to illustrate statements which could be false. 
\medskip

(i)  $(u\times v)\cdot v=0$. \soln This is always true: it is a property of the cross product which is easliy checked.
 
\medskip

(ii)  $(v\times u)\cdot v=-1$ \soln This is always false, and the left hand side is always zero. So take $u=v=0$.
 
\medskip

(iii)  $(u\times v)\cdot w$ is the volume of the of the parallelepiped  determined by $u$, $v$ and $w$. \soln Volumes are always positive, so this is only true if $(u\times v)\cdot w= |(u\times v)\cdot w|$; so if $u=(1,0,0), v=(0,1,0)$ and $w=(0,0,-1)$, $(u\times v)\cdot w=-1$, which is not the volume of the unit cube.  
 
\medskip

(iv)  $||u\times v||=||u||\,||v||\,\cos\theta$
where $\theta$ is the angle between $u$ and $v$. \soln This is only true if $\cos\theta=\sin \theta$, since it is always true that $||u\times v||=||u||\,||v||\,\sin\theta$, where $\theta$ is the angle between $u$ and $v$. So take $u=(1,0,0)$ and $v=(0,1,0)$ for a counterexample.

\medskip 

(v)  $|u\cdot v|=||u||\,||v||\,\cos\theta$
where $\theta$ is the angle between $u$ and $v$. \soln This is always true.
 
\end{sol} 


