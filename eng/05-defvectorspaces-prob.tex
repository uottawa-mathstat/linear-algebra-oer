 
 
 
  

 
\medskip {\bf Remark:} In the following examples, {\it explain your answer}, which means: if you say a set closed under some operation, give an explanation (`proof') which works  in {\it all} cases: \underbar{don't} choose examples and simply verify it works for your chosen examples. On the other hand, if you say the set isn't closed under some operation, {\it you must} give an example to illustrate your answer. 

  \begin{prob} \label{prob04.1} Determine whether   the following sets are closed under the indicated rule for addition. 

  
\begin{enumerate}[a)]
\medskip
\item  $\set{(x, x+2) \in \R^2\st x\in \R}$; standard addition of vectors in $\R^2$ \medskip
% no

\item\sov  $\set{(x, y) \in \R^2\st x -3y=0 }$; standard addition of vectors in $\R^2$ \medskip
%

\item $\set{(x, y) \in \R^2\st x -3y=1 }$; standard addition of vectors in $\R^2$ \medskip
%

\item\sov  $\set{(x, y) \in \R^2\st xy \ge 0 }$ ; standard addition of vectors in $\R^2$\medskip
%
 
\item  $\set{(x, y, z) \in \R^3\st x+2y+z=0 }$ ; standard addition of vectors in $\R^3$ \medskip
%
\item\sov  $\set{(x, y, z) \in \R^3\st x+2y+z=1 }$ ; standard addition of vectors in $\R^3$\medskip 
%
\item  $\set{(x, y, z, w) \in \R^4\st x-y+z-w=0 }$; standard addition of vectors in $\R^4$.\medskip 
%
\item\sov  $\set{(x, x+2) \in \R^2\st x\in \R}$; {\it Non-standard addition}: $(x,y) \tilde+ (x',y')=(x+x', y+y'-2)$.    \medskip
%
\item  $\set{(x, y, z) \in \R^3\st x+2y+z=1 }$ ; {\it Non-standard addition}: $(x,y,z) \tilde+ (x',y',z')=(x+x', y+y', z+z'-1)$.\medskip 
%
\end{enumerate}

\end{prob} \begin{prob} \label{prob04.2} Determine whether each of the following sets is closed under the indicated rule for multiplication of vectors  by scalars.
\begin{enumerate}[a)]
\medskip
\item $\set{(x, x+2) \in \R^2\st x\in \R}$; standard rule for multiplication of vectors  in $\R^2$ by scalars. \medskip
% no

\item\sov  $\set{(x, y) \in \R^2\st x -3y=0 }$;  standard rule for multiplication of vectors  in $\R^2$ by scalars.   \medskip
%
\item$\set{(x, y) \in \R^2\st x -3y=1 }$; standard rule for multiplication of vectors  in $\R^2$ by scalars.\medskip
\item\sov  $\set{(x, y) \in \R^2\st xy \ge 0 }$; standard rule for multiplication of vectors  in $\R^2$ by scalars.\medskip
%
 
\item  $\set{(x, y, z) \in \R^3\st x+2y+z=0 }$; standard rule for multiplication of vectors  in $\R^3$ by scalars.\medskip \medskip
%
\item\sov  $\set{(x, y, z) \in \R^3\st x+2y+z=1 }$; standard rule for multiplication of vectors  in $\R^3$ by scalars.\medskip 
%
\item  $\set{(x, y, z, w) \in \R^4\st x-y+z-w=0 }$; standard rule for multiplication of vectors in  $\R^4$ by scalars.\medskip 
%
\item\sov  $\set{(x, x+2) \in \R^2\st x\in \R}$; {\it Non-standard  multiplication of vectors  by  scalars $k\in \R$}: $$k\circledast (x,y)=(kx, ky-2k+2).$$   
%
\item  $\set{(x, y, z) \in \R^3\st x+2y+z=1 }$ ; {\it Non-standard  multiplication of vectors  by  scalars $k\in \R$}: $$k\circledast (x,y,z)=(kx, ky, kz-k+1).$$
%
\end{enumerate} 


\end{prob} \begin{prob} \label{prob04.3}  Determine whether   the following subsets of $\F(\R)=\set{f \st f: \R \to \R}$ are closed under the standard addition of functions in  $\F(\R)$. (Recall that $\F(\R)$ consists of all real-valued functions of a real variable; i.e., all functions with domain $\R$, taking values in $\R$). 
\begin{enumerate}[a)]\medskip
\item  $\set{f \in \F(\R) \st f(2)=0 }$ ; \medskip \medskip
%
\item\sov  $\set{f \in \F(\R) \st f(2)=1 }$.\medskip \medskip
% 

\item  $\set{f \in \F(\R) \st f(1)=2 }$.\medskip \medskip
%
\item\sov  $\set{f \in \F(\R) \st \text{ for all } x\in \R,   \, f(x)\le 0}$.\medskip 
%
\item  $\set{f \in \F(\R) \st \text{ For all } x\in \R,   \, f(-x)= f(x)}$\medskip 
%
\item\sov  $\set{f \in \F(\R) \st \text{ For all } x\in \R,   \, f(-x)= -f(x)}$\medskip 
%
\item $\set{f \in \F(\R)   \st \text{$f$ is twice-differentiable, and  for all } x\in \R,   \, f''(x)+ f(x)=0}$ \medskip 
%
 
\end{enumerate}

\end{prob} \begin{prob} \label{prob04.4} Determine whether the following sets are closed under the standard rule  for multiplication of functions  by scalars in $\F(\R)$. 
\begin{enumerate}[a)]\medskip
\item  $\set{f \in \F(\R) \st f(2)=0 }$.\medskip \medskip
%
\item\sov  $\set{f \in \F(\R) \st f(2)=1 }$.\medskip \medskip
% 

\item  $\set{f \in \F(\R) \st f(1)=2 }$.\medskip \medskip
%
\item\sov  $\set{f \in \F(\R) \st \text{ For all } x\in \R,   \, f(x)\le 0}$\medskip 
%
\item  $\set{f \in \F(\R) \st \text{ For all } x\in \R,   \, f(-x)= f(x)}$\medskip 
%
\item\sov  $\set{f \in \F(\R) \st \text{ For all } x\in \R,   \, f(-x)= -f(x)}$\medskip 
%
 \item  $\set{f \in \F(\R)   \st \text{$f$ is twice-differentiable, and  for all } x\in \R,   \, f''(x)+ f(x)=0}$ \medskip  
%
\end{enumerate}


\end{prob} \begin{prob} \label{prob04.5} Determine whether the following sets are closed under the standard operation of addition of matrices in $\M_{2 \,2}(\R)$. 
\begin{enumerate}[a)]\medskip


\item  $\Bigg\{  \bmatrix a&b\\ c&d\endbmatrix \in \M_{2 \,2}(\R) \;\Bigg|\; b=c\Bigg\}$.\medskip \medskip
%

\item\sov  $\Bigg\{  \bmatrix a&b\\ c&d\endbmatrix \in \M_{2 \,2}(\R) \;\Bigg|\; a+d=0\Bigg\}$. \medskip
%

\item  $\Bigg\{  \bmatrix a&b\\ c&d\endbmatrix \in \M_{2 \,2}(\R) \;\Bigg|\; ad-bc=0\Bigg\}$. \medskip
%


\item\sov  $\Bigg\{  \bmatrix a&b\\ c&d\endbmatrix \in \M_{2 \,2}(\R) \;\Bigg|\; ad=0\Bigg\}$.      \medskip
%

\item  $\Bigg\{  \bmatrix a&b\\ c&d\endbmatrix \in \M_{2 \,2}(\R) \;\Bigg|\; bc=1\Bigg\}$.      \medskip
%

\end{enumerate}

\end{prob} \begin{prob} \label{prob04.6} Determine whether the following sets are closed under the standard rule for multiplication of matrices by scalars in $\M_{2 \,2}(\R)$. 
\begin{enumerate}[a)]\medskip


\item  $\Bigg\{  \bmatrix a&b\\ c&d\endbmatrix \in \M_{2 \,2}(\R) \;\Bigg|\; b=c\Bigg\}$.\medskip \medskip
%

\item\sov  $\Bigg\{  \bmatrix a&b\\ c&d\endbmatrix \in \M_{2 \,2}(\R) \;\Bigg|\; a+d=0\Bigg\}$. \medskip
%

\item  $\Bigg\{  \bmatrix a&b\\ c&d\endbmatrix \in \M_{2 \,2}(\R) \;\Bigg|\; ad-bc=0\Bigg\}$. \medskip
%


\item\sov  $\Bigg\{  \bmatrix a&b\\ c&d\endbmatrix \in \M_{2 \,2}(\R) \;\Bigg|\; ad=0\Bigg\}$.      \medskip
%
\item  $\Bigg\{  \bmatrix a&b\\ c&d\endbmatrix \in \M_{2 \,2}(\R) \;\Bigg|\; bc=1\Bigg\}$.      \medskip
%
\end{enumerate}

 

\end{prob} \begin{prob} \label{prob04.7} The following sets have been given the indicated rules for addition of vectors,  and multiplication of objects  by real scalars (the so-called  {\it `vector operations' }). If possible, check if there is a zero vector in the subset in each case. If it is possible, show your choice works in {\it all } cases, and if it is not possible, give an example to illustrate your answer. 

(Note: in the last two parts, since the vector operations are not the standard ones, the zero vector will probably not be the one you're accustomed to.)
\begin{enumerate}[a)]
\medskip
\item  $\set{(x, x+2) \in \R^2\st x\in \R}$; standard  vectors operations in $\R^2$. \medskip
% no

\item\sov  $\set{(x, y) \in \R^2\st x -3y=0 }$; standard  vectors operations in $\R^2$. \medskip
%

\item $\set{(x, y) \in \R^2\st x -3y=1 }$; standard  vectors operations in $\R^2$.\medskip
%

\item\sov  $\set{(x, y) \in \R^2\st xy \ge 0 }$;  standard  vectors operations in $\R^2$.\medskip
%
 
\item  $\set{(x, y, z) \in \R^3\st x+2y+z=0 }$; standard  vectors operations in $\R^3$.\medskip \medskip
% 
\item\sov  $\set{(x, y, z) \in \R^3\st x+2y+z=1 }$; standard  vectors operations in $\R^3$.\medskip  
%
\item  $\set{(x, y, z, w) \in \R^4\st x-y+z-w=0 }$; ; standard  vectors operations in $\R^4$.\medskip 
%
\item\sov  $\set{(x, x+2) \in \R^2\st x\in \R}$; \underbar{\it Non-standard operations:--} Addition: $(x,y) \tilde+ (x',y')=(x+x', y+y' -2)$. Multiplication of vectors  by  scalars $k\in \R$: $k\circledast (x,y)=(kx, ky-2k+2)$.     \medskip
%
\item  $\set{(x, y, z) \in \R^3\st x+2y+z=1 }$ ; \underbar{\it Non-standard operations:--} Addition: $(x,y) \tilde+ (x',y')=(x+x', y+y',z+z'-1)$. Multiplication of vectors  by  scalars $k\in \R$: $k\circledast (x,y,z)=(kx, ky, kz-k+1)$. \medskip 
%
\end{enumerate}
 
\end{prob} \begin{prob} \label{prob04.8} Explain your answers to the following:

\begin{enumerate}[a)] 

\item\sov  Determine whether the  zero function  of $\F(\R)$ belongs to each of the subsets in  question 3. \medskip 
% 
 
\item  Determine whether the  zero matrix of $\M_{2 \,2}(\R)$  belongs to each of the subsets in  question 5. \medskip 
%
   
\end{enumerate}  


\end{prob} \begin{prob} \label{prob04.9}  The following sets have been given the indicated rules for addition of vectors,  and multiplication of objects  by real scalars. In each case, If possible, check if vector in the subset has a `negative' in the subset.  

Again, since the vector operations are not the standard ones, the negative of a vector will probably not be the one you're accustomed to seeing.

\begin{enumerate}[a)]
\medskip

\item  $\set{(x, x+2) \in \R^2\st x\in \R}$; \underbar{\it Non-standard Operations:--} Addition: $(x,y) \tilde+ (x',y')=(x+x', y+y'-2)$. Multiplication of vectors  by  scalars $k\in \R$: $k\circledast (x,y)=(kx, ky-2k+2)$.     \medskip
%
\item\sov  $\set{(x, y, z) \in \R^3\st x+2y+z=1 }$ ; \underbar{\it Non-standard Operations:--} Addition: $$(x,y) \tilde+ (x',y')=(x+x', y+y',z+z'-1).$$ Multiplication of vectors  by  scalars $k\in \R$: $k\circledast (x,y,z)=(kx, ky, kz-k+1)$. \medskip 
%
\end{enumerate}

\end{prob} \begin{prob} \label{prob04.10} Explain your answers to the following:

\begin{enumerate}[a)] 

\item\sov  Determine whether the subsets in  question 1  (given the operations described in questions 1 and 2)  are vector spaces. \medskip 
%

\item  Determine whether the  subsets  of $\F(\R)$ in  question 3, equipped with the standard vector operations of  $\F(\R)$ are vector spaces. \medskip 
% 
 
\item  Determine whether the subsets of $\M_{2 \,2}(\R)$ in  question 5, equipped with the standard vector operations of $\M_{2 \,2}(\R)$ are vector spaces. \medskip 
%
 
\end{enumerate}


\end{prob} \begin{prob} \label{prob04.11} Explain your answers to the following:

\begin{enumerate}[a)] 

\item\sov  Determine whether the subsets of $\F(\R)$ in  question 3 are vector spaces. \medskip 
%

\item  Determine whether the  zero function  of $\F(\R)$ belongs to each of the subsets in  question 3. \medskip 
%

\item  Determine whether the  zero matrix of $\M_{2 \,2}(\R)$  belongs to each of the subsets in  question 5. \medskip 
%
 
\end{enumerate}


\end{prob} \begin{prob} \label{prob04.12} Justify your answers to the following:
\begin{enumerate}[a)]
\item Equip the set $V=\R^2$ with the \underbar{\it non-standard operations:--} Addition: $$(x,y) \tilde+ (x',y')=(x+x', y+y'-2).$$ Multiplication of vectors  by  scalars $k\in \R$: $$k\circledast (x,y)=(kx, ky-2k+2).$$  Check that  $\R^2$, with these new operations, is indeed a vector space. \medskip
%
\item  Equip the set $V=\R^3$ \underbar{\it non-standard operations:--} Addition: $$(x,y,z) \tilde+ (x',y',z')=(x+x', y+y', +y,z+z'-1).$$ Multiplication of vectors  by  scalars $k\in \R$: $$k\circledast (x,y,z)=(kx, ky, kz-k+1).$$ Check that  $\R^3$, with these new operations, is indeed a vector space.

\medskip

In these cases, you will even need to check the arithmetic axioms, as the operations are weird. For example, the distributive axiom:  $k\circledast (u \tilde+ v)  = k\circledast u \tilde+ k\circledast v$ is true, but definitely not {\it obviously} so! 

 
\end{enumerate}
 
\end{prob} \begin{prob} \label{prob04.13}  Let $\E=\set{``ax+by+ cz=d"\st a,b,c, d\in\R}$ be the set of linear equations with real coefficients in the variables $x$, $y$ and $z$. Equip $\E$ with the usual operations on equations that you learned in high school: addition of equations, denoted here by ``$\dsum$" and multiplication by scalars, denoted here by ``$\circledast$", as follows: 

$$``ax+by+ cz=d" \dsum ``ex+fy+gz=h" =``(a+e)x + (b+f)y + (c+g)z=d+h"$$and
$$ \forall   k\in \R,\quad    k\circledast``ax+by+ cz=d" = `` ka\, x+ kb \,y+ kc \,z = k\,d".$$

Prove that $\E$ is a vector space.  \bigskip
 
\end{prob} \begin{prob} \label{prob04.14} (For the mathematically curious)
\label{exVS}
\begin{enumerate}[a)] 

\item  It is {\it not} amongst the axioms for a vector space $V$ that $0 \,\vv =\zero$ for all vectors $\vv\in V$. (Here the zero on the left hand side of the equation is the {\it scalar} zero, while the zero on the right hand side of the equation is the zero {\it vector}.)

Nevertheless, it is indeed true in every vector space that $0\, \vv =\zero$ for all vectors $\vv\in V$. 

Prove this, using a few of the axioms for a vector space. \medskip 
%

\item   Neither it is   amongst the axioms for a vector space $V$ that $(-1)\vv =-\vv$ for all vectors $\vv\in V$, where the `$(-1)\vv$' on the left hand side of the equation indicates the result of multiplication of $\vv$ by the scalar $-1$, and the  `$ -\vv$' on the right hand side of the equation indicates the negative of $\vv$ -- whose existence is guaranteed by one of the axioms.

Nonetheless, it is indeed true in every vector space that $(-1) \vv =-\vv$ for all vectors $\vv\in V$. 
 
Prove this, using a few of the axioms for a vector space.  You might find part (a) useful.\medskip 
%
 \item Let  $\emptyset=\set{}$ denote the empty set. Could it be made into a vector space?
\medskip 
%
\item And here's another interesting vector space:  Define
$V$ to be the set of \emph{formal power series}.
A formal power series is an expression of the form
$$
\sum_{n=0}^\infty a_nx^n = a_0 + a_1x+ a_2x^2 + \cdots
$$
where the coefficients $a_n$ are real numbers.  Define
addition by the formula
$$
\sum_{n=0}^\infty a_nx^n + \sum_{n=0}^\infty b_n x^n =  \sum_{n=0}^\infty (a_n+b_n) x^n
$$    
and scalar multiplication by
$$
c  \sum_{n=0}^\infty a_n x^n =  \sum_{n=0}^\infty ca_n x^n.
$$
Show that this is a vector space.\footnote{Note that in general these
are not functions on $\R$, since most of the time, if you
plug in any non-zero value for $x$, the sum makes no sense
(technical term: ` the series diverges').  The convergent ones (that is,
ones where you can plug in  some values of $x$ and it actually makes sense) include the power series for $e^x$, $\cos(x)$
and $\sin(x)$, which we glimpsed in our first class.  Power
series are terrifically useful tools in Calculus. If you're lucky, you'll learn what `convergence' really means for an infinite series in MAT1325 or MAT2125.}
\end{enumerate}
\end{prob}