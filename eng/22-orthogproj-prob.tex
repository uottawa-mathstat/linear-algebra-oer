\section*{Problems}
\addcontentsline{toc}{section}{Problems}
 
\medskip {\bf Remarks:} 
\begin{enumerate}
\item A question with an asterisk `$ ^\ast$' (or two) indicates a bonus-level question.
 \item You must justify all your responses.
\end{enumerate}
\bigskip

\begin{prob} \label{prob19.1} In each case, find the Fourier coefficients of the vector $\vv$ with respect to the given orthogonal basis $\mathcal B$  of the indicated vector space $W$.
\medskip

\begin{enumerate}[a)]
\item  $\vv=(1,2,3)$, $\mathcal B = \set{(1,0,1),(-1,0,1), (0,1,0) }$, $W=\R^3$.
\medskip
%
\item\sov  $\vv=(1,2,3)$, $\mathcal B = \set{(1, 2 , 3 ),(-5, 4, -1),(1, 1, -1)}$, $W=\R^3$.
\medskip
%
\item  $\vv=(1,2,3)$, $\mathcal B = \set{(1, 0, 1),(-1, 2, 1)}$, $$W=\set{(x,y,z)\in\R^3 \st x+y-z=0}.$$
 
%
\item\sov $\vv=(4,-5,0)$, $\mathcal B = \set{(-1, 0, 5),(10, 13, 2)}$, $$W=\set{(x,y,z)\in\R^3 \st 5x-4y+z=0}.$$
 
%
\item $\vv=(1,1,1,1)$,  $\mathcal B =\set{(1, 0, 1, 1), (0, 1, 0, 0), (0, 0, 1, -1)}$, $$W=\set{(x,y,z, w)\in\R^4 \st x-w=0}.$$
\medskip
%
\item\sov $\vv=(1,0,1,2)$,  $\mathcal B =\set{(1, 0, 1, 1), (0, 1, 0, 0), (0, 0, 1, -1),(1, 0, 0, -1)}$, $W= \R^4$.
\medskip
%
\end{enumerate}

\end{prob} \begin{prob} \label{prob19.2} Find the formula for the orthogonal projection onto the subspaces in parts c), d) \sov and e) above.

\end{prob} \begin{prob} \label{prob19.3} Apply the Gram-Schmidt algorithm  to each of the following linearly independent sets (to obtain an orthogonal set), and check that your resulting set of vectors is orthogonal.
\medskip
\begin{enumerate}[a)]
\item $\set{(1,1,0),(2,0,3)}$
\medskip
%
\item\sov $\set{(1, 0, 0, 1),(0, 1, 0, -1),(0, 0, 1, -1)}$
\medskip
%
\item $\set{(1, 1, 1, 1),(0, 1, 0, 0),(0, 0, 1, -1)}$
\medskip
%
\item\sov $\set{(1, 1, 0),(1, 0, 2),(1, 2, 1)}$
\medskip
%
\end{enumerate}
\end{prob} \begin{prob} \label{prob19.4} Find an orthogonal basis for each of the following subspaces, and check that your basis is orthogonal. (First, find a basis in the standard way, and then apply the Gram-Schmidt algorithm.)
\medskip
\begin{enumerate}[a)]
\item $W=\set{(x,y,z)\in\R^3 \st x+y+z=0}$
\medskip
%
\item\sov 
 $U=\set{(x,y,z, w)\in\R^4 \st x+y-w=0}$\medskip
%
\item $X=\set{(x,y,z, w)\in\R^4 \st x+y-w=0 \, \text{ and } \, z+y=0}$
\medskip
%
\item\sov $V=\ker \bmatrix 1 & 2 & -1 & -1 \\
 2 & 4 & -1 & 3 \\
 -3 & -6 & 1 & -7 \endbmatrix$
\medskip
%
\end{enumerate}

\end{prob} \begin{prob} \label{prob19.5} Find the best approximation to each of the given vectors $\vv$ from the given subspace $W$. \medskip
\begin{enumerate}[a)]
\item  $\vv=(1,1,1)$,  $W=\set{(x,y,z)\in\R^3 \st x+y-z=0}$
\medskip
%
\item\sov $\vv=(1,1,1)$,   $W=\set{(x,y,z)\in\R^3 \st 5x-4y+z=0}$
\medskip
%
\item  $\vv=(1,1,1,2)$,   $W=\set{(x,y,z, w)\in\R^4 \st x-w=0}$
\medskip
%
\end{enumerate}
\end{prob} \begin{prob} \label{prob19.6} State whether each of the following is (always) true,
or is (possibly) false.    
   \smallskip    
\begin{enumerate}[$\bullet$]
\item If you say the statement may be false, you    must give an explicit example.   
\item If you say the statement is true, you must give a clear explanation -   by quoting a theorem presented in class, or by giving a {\it proof valid for every  case}. 
\end{enumerate}
\medskip
\begin{enumerate}[a)]
\item Every orthogonal set is linearly independent.
\medskip
%
\item\sov Every linearly independent set is orthogonal.
\medskip
%
\item When finding the orthogonal projection of a vector $\vv$ onto a subspace $W$, plugging {\it any} basis of $W$ into the formula in Definition~\ref{projdef} will work. 
\medskip
%
\item\sov When finding the orthogonal projection of a vector $\vv$ onto a subspace $W$, once the answer is obtained, it's OK to re-scale the answer to eliminate fractions.
\medskip
%
\item When applying the Gram-Schmidt algorithm to a basis, at each step, it's OK to re-scale the vector obtained to eliminate fractions.
\medskip
% 
\item\sov When finding the orthogonal projection of a vector $\vv$ onto a subspace $W$, using different orthogonal bases of $W$ in the formula in Definition~\ref{projdef} can give different answers.
\medskip
%
\item If $\proj_W(\vv)$ denotes the orthogonal projection of a vector $\vv$ onto a subspace $W$, then $\vv-\proj_W(\vv)$ is always orthogonal to every vector in $W$.
\medskip
%
\item\sov To check that a vector, say $\uu$, is orthogonal to every vector in $W$, it suffices to check that $\uu$ is orthogonal to every vector in {\it any} basis of $W$.
\medskip
%
\item$^{\ast\ast}$ If you apply the the Gram-Schmidt algorithm to a {\it spanning set} $\set{\ww_1, \dots , \ww_m}$ of a subspace $W$, rather than a basis of $W$, and if you obtain a zero vector at step $k$, that means $\ww_k \in \spn\{\ww_1, \dots , \ww_{k-1}\}$.
\medskip
%
\item $^{\ast\ast}$ If you apply the the Gram-Schmidt algorithm to a {\it spanning set} $\set{\ww_1, \dots , \ww_m}$ of a subspace $W$, rather than a basis of $W$, and discard any zero vectors that appear in the process, you will in the end still obtain an orthogonal basis for $W$.
\medskip
%
\end{enumerate}
\end{prob}
  
