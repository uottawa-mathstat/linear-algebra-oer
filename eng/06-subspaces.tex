\chapter{Subspaces and Spanning Sets}\label{Chapter:05subspaces}

In the last chapter, we established the concept of a \emph{vector space},
which is a set $V$ on which we can perform two operations
(addition and scalar multiplication) such that 10 axioms
are satisfied.  The result is:  a vector space is algebraically
indistinguishable from things like $\R^n$ (as far
as addition and scalar multiplication go).

We met some particularly nice vector spaces:
\begin{itemize}
\item $\R^n$, $n\geq 1$
\item $\mathcal{E}$, a space of linear equations in 3 variables
\item $F[a,b]$, functions on the interval $[a,b]$
\item $F(\R)$, functions on the real line
\item $M_{m\times n}(\R)$, $m\times n$ real matrices
\end{itemize}
And for each of these, we had to check 10 axioms.  But we
noticed that sometimes, the last 6 axioms (the ones dealing
purely with algebraic properties) were ``obvious''.

Let's consider this more carefully.




\section{Subsets of Vectors Spaces}  
Suppose $V$ is a vector space, and $W \subseteq V$ is a \emph{subset} of 
$V$.  

\begin{myexample}
Let $V = \R^2$ (we know it's a vector space), and let $W = \{(x,2x) | x\in \R\}$.  Use the \emph{same} operations of addition and scalar multiplication
on $W$ as we use in $\R^2$.  What do we really need to verify to decide
if
it's a vector space?

{\bf closure:}  (1) closure under addition:  if $(x,2x)$ and $(y,2y)$ are in $W$,
then $(x,2x) + (y,2y) = (x+y, 2(x+y))$, which is in $W$ because it has
the correct form $(z,2z)$ with $z=x+y$.  YES.

(2) closure under scalar multiplication:  if $(x,2x) \in W$ and $c\in \R$
then $c(x,2x) = (cx,2(cx))\in W$ as well.  YES.

{\bf existence:}  (3) the zero vector is $(0,0)$, which is in $W$.  But we
don't need to bother checking if $\zero + \uu = \uu$ for
each $\uu \in W$, because we know this is true for all $\uu \in \R^2$
already (since $V$ is a vector space and so satisfies axiom (3)).  So
it's enough to know that $\zero \in W$.

(4) if $\uu \in W$, then we know that $-\uu = (-1)\uu \in W$
because (2) is true.  And again, we know that the negative has the 
correct property because we checked it when we decided that $V$
was a vector space.

{\bf algebraic properties:} (5)-(10)  All of these properties are
known to be true for any vectors in $V$; so in particular they
are true for any vectors in the smaller set $W \subset V$.  So
we don't need to check them again!

We conclude:  $W$ is a vector space.  \end{myexample}

\begin{definition}
A subset $W$ of a vector space $V$ is called a \defn{subspace}
(or \emph{subspace of $V$}) if it is a vector space when
given the \stress{same
operations} of addition and scalar multiplication as $V$.
\end{definition}

\begin{myexample}
$W = \{ (x,2x) | x\in \R\}$ is a subspace of $\R^2$. \end{myexample}

\begin{theorem}[Subspace Test]\index{subspace test} : 
If $V$ is a vector space and $W \subseteq V$, then $W$ is 
a subspace of $V$ if and only if the following 3 conditions
hold:
\begin{enumerate}
\item $\zero \in W$.
\item $W$ is closed under addition:  for every $\uu, \vv \in W$, $\uu+\vv$ is again in $W$.
\item $W$ is closed under multiplication by scalars: for every $\uu \in W$ and $r\in \R$, $r\uu$ is again in $W$.
\end{enumerate}
\end{theorem}

Think of this theorem as giving you a shortcut to working out
if certain sets are vector spaces.  We saw in the example above
why it comes down to just these three axioms. \footnote{You might wonder why we have to include the first one, though,
since $0 \uu = \zero$ so it ought to follow from closure
under scalar multiplication.  It does, but only if the set
$W$ is not the empty set.  So if $W$ isn't empty, then it's 
enough to check axioms (2) and (3).  BUT as a general rule, 
(1) is super-easy to check;  and if it fails you know $W$ isn't
a subspace.  So it's a quick way to exclude certain sets and
we always use it.}

\section{Many examples}  Now let us apply the subspace
test to acquire many more examples of vector spaces.  On
the homework and in the suggested exercises, you'll see
many examples of subsets to which the subspace test does
not apply (because the operations aren't the same) or where
the subspace test fails.  Do many examples to develop your
intution about what does and does not constitute a vector
space!

\begin{myprob} Let $T = \{ \uu \in \R^3 | \uu \cdot (1,2,3) = 0\}$,
with the usual operations on $\R^3$.  Is this a subspace?

\begin{mysol} Notice that $T \subset \R^3$ and $T \neq \R^3$.  In fact, 
$T$ is the plane with equation
$$
x + 2y + 3z= 0.
$$
Since we're using the usual operations on $\R^3$, we apply
the subspace test:
\begin{enumerate}
\item Is $\zero \in T$?  Yes, since $\zero = (0,0,0)$ satisfies
the condition $\uu \cdot (1,2,3) = 0$.
\item Is $T$ closed under addition? \\
 Well, suppose $\uu$ and $\vv$
are in $T$. \\
 That means $\uu \cdot (1,2,3) = 0$ and $\vv \cdot (1,2,3) = 0$.  \\
We need to decide if $\uu + \vv \in T$.  \\
That means we
need to decide if $(\uu + \vv) \cdot (1,2,3) = 0$. \\
 We calculate: $(\uu + \vv) \cdot (1,2,3) = (\uu \cdot (1,2,3)) + (\vv \cdot (1,2,3)) = 0 + 0 = 0$.\\
So $\uu + \vv \in T$, and we conclude $T$ is closed under addition.
\item Is $T$ closed under multiplication by scalars?\\
Let $k \in \R$ and $\uu \in T$.  \\
So we have that $\uu \cdot (1,2,3) = 0$.\\
We want to decide if $k\uu \in T$.\\
That means we need to see if $(k\uu)\cdot (1,2,3) = 0$.\\
We calculate: $(k\uu)\cdot (1,2,3) = k(\uu \cdot (1,2,3)) = k(0) = 0$.\\
So $k\uu \in T$ and so $T$ is closed under scalar multiplication.
\end{enumerate}
So YES, $T$ is a subspace of $\R^3$. \end{mysol}\end{myprob}

Notice that we could have replaced the vector $(1,2,3)$ with any
vector $\nn$, and the result would have been the same.  In
fact:

\standout{Any plane through the origin in $\R^3$ is a subspace.}

Furthermore, if a plane in $\R^3$ \emph{doesn't} go through
the origin, then it fails the first condition of the subspace
test.  So we have:

\standout{Any plane in $\R^3$ which doesn't go through the origin is not a subspace.}



\begin{myprob} Let $\vv \in \R^n$ and set $L = \{ t\vv | t\in \R\}$ to be
the line in $\R^n$ through the origin with direction vector $\vv$
(with the usual operations from $\R^n$).  Is $L$ a subspace?

\begin{mysol} We apply the subspace test.
\begin{enumerate}
\item Yes, $\zero \in L$: take $t=0$.
\item If $t\vv$ and $s\vv$ are two points in $L$, then $t\vv + s\vv = (t+s)\vv$
which is again a multiple of $\vv$, so lies in $L$.  So $L$ is closed
under addition.
\item If $t\vv \in L$ and $k\in\R$ then $k(t\vv) = (kt)\vv$, which is a 
multiple of $\vv$, so it lies in $L$.  Thus $L$ is closed under scalar
multiplication.
\end{enumerate}
We conclude that $L$ is a subspace of $\R^n$. \end{mysol}\end{myprob}

Similarly to above, we deduce:

\standout{Any line through the origin in $\R^n$ is a subspace.  Any line in $\R^n$ which does not go through the origin is not a subspace.}

\begin{myprob} Let $V = F(\R)$, the vector space of all functions with domain $\R$.
Let $\mathbb{P}$ be the set of all \emph{polynomial functions}, that is,
$\mathbb{P}$ consists of all functions that can be written as
$$
p(x) = a_0 + a_1 x + a_2 x^2 + \cdots + a_n x^n
$$
for some $n \geq 0$ and $a_i \in \R$.  Is $\PP$ a vector space?

\begin{mysol} We first note (it may not seem obvious) that the natural
operations of adding two polynomials, and multiplying a polynomial
by a scalar, are exactly the usual operations on functions $F(\R)$.
So $\PP \subset F(\R)$ with the same operations, so we can apply
the subspace test.  (Phew!)
\begin{enumerate}
\item Is $\zero \in \PP$?  Remember that $\zero$, here, is the
function which when you plug in any $x$ you get $0$ as an answer.
Well, that's the zero polynomial:  $p(x) = 0 + 0x + 0x^2$.  (Or just
$p(x) = 0$, for short.)  It's a polynomial, so $\zero \in \PP$.
\item Is $\PP$ closed under addition?  Yes: the sum of two polynomials
is again a polynomial (it's not as though you could get an exponential
function or something like that).
\item Is $\PP$ closed under scalar multiplication?  Yes: multiplying
a polynomial by a scalar gives another polynomial.
\end{enumerate}
So yes, $\PP$ is a subspace, so a vector space.
\end{mysol}\end{myprob}


\begin{definition}\label{transpose}
The \defn{transpose} of an $m\times n$ matrix $A$ is the $n \times m$ matrix $A^T$
whose rows are the columns of $A$.  For example,
$$
\mat{a&b\\c&d}^T = \mat{a&c\\b&d}.
$$
\end{definition}

\begin{myprob} Let $S = \{ A \in M_{22}(\R) | A^T = A\}$ be the set of \emph{symmetric} $2 \times 2$ matrices (with the usual operations).  Is this a vector space?

\begin{mysol} Since $S \subset M_{22}(\R)$, and we're using the same operations,
we can apply the subspace test.

But it's helpful to get more comfortable with this set $S$, first.
Rewrite it as:
\begin{align*}
S &= \left\{ \mat{a&b\\c&d} |  \mat{a&b\\c&d}^T = \mat{a&b\\c&d}\right\}\\
&= \left\{ \mat{a&b\\c&d} |  \mat{a&c\\b&d} = \mat{a&b\\c&d}\right\}\\
&= \left\{ \mat{a&b\\b&d} | a,b,d \in \R\right\}
\end{align*}
Oh, that's more clear!

\begin{enumerate}
\item By taking $a=0$, $b=0$ and $d=0$, we get the zero matrix; so this
is in $S$.
\item Take two arbitary matrices in $S$, and add them:
$$
\mat{a&b\\b&d} + \mat{a'&b' \\ b'&d'} = \mat{a+a' & b+b' \\ b+b' & d+d'}
$$
This is again in $S$, since it matches the required pattern to be
in $S$ (that the (1,2) and (2,1) entries of the matrix are equal).
So $S$ is closed under addition.
\item Let $k\in \R$; then
$$
k\mat{a&b\\b&d} = \mat{ka & kb \\ kb& kd}
$$
is again in $S$, so $S$ is closed under scalar multiplication.
\end{enumerate}
So $S$ is a subspace of $M_{22}(\R)$.
\end{mysol}\end{myprob}




\section*{Problems}
\addcontentsline{toc}{section}{Problems}
%
% Use the following environment.
% Don't forget to label each problem;
% the label is needed for the solutions' environment


