\section*{Problems}
\addcontentsline{toc}{section}{Problems}


\medskip {\bf Remarks:} 
\begin{enumerate}
\item A question with an asterisk `$ ^\ast$' (or two) indicates a bonus-level question.
 \item You must justify all your responses.
\end{enumerate}
\bigskip

%\centerline{\bf \underbar{}}

 \begin{prob} \label{prob24.1} State whether each of the following defines a linear transformation.    
   \smallskip    
\begin{enumerate}[$\bullet$]
\item If you say it isn't linear, you must give an explicit example to illustrate.   
\item If you say it is linear, you must give a clear explanation -   by quoting a theorem presented in class, or by verifying the conditions in the definition {\it  in every  case}. 
\end{enumerate}
\medskip
\begin{enumerate}[a)]
\item  $T:\R^2 \to \R^3$ defined by $T(x,y)=(x, y, x+y)$
\medskip
%
\item\sov $T:\R^3 \to \R^2$ defined by $T(x,y,z)=(2 z+x, y)$
\medskip
%
\item $T:\R^2 \to \R^2$ defined by $T(x,y)=(x, x y)$
\medskip
%

\item\sov $T:\R^2 \to \R^2$ defined by $T(\vv)=\bmatrix 0&-1\\ 1&0 \endbmatrix \vv$
\medskip
%
\item $T:\R^3 \to \R^3$ defined by $T(\vv)= \vv\times (1,2,3)$, where `$\times$' denotes the cross product.
\medskip
%

\item\sov $T:\R^3 \to \R^3$ defined by $T(\vv)= \proj_{(1,1,-1)}(\vv)$.
\medskip
%
\item $T:\R^3 \to \R^3$ defined by $T(\vv)= v-\proj_{(1,1,-1)}(\vv)$.
\medskip
%
\item\sov $T:\R^3 \to \R^3$ defined by $T(\vv)= \proj_{\vv}(1,1,-1)$.
\medskip
%
\item $T:\R^3 \to \R^3$ defined by $T(\vv)= \big(\vv \cdot (1,1,-1)\big) (1,0,1)$.
\medskip
%
\item\sov $T:\R^3 \to \R^3$ defined by $T(\vv)= 2 \vv$.
\medskip
%
\item $T:\R^3 \to \R^3$ defined by $T(\vv)= \proj_H(\vv)$, where $H$ is the plane through the origin with normal $(1,1,0)$.
\medskip
%
\item\sov $T:\R^3 \to \R^2$ defined by $T(\vv)= A\vv$, where $A=\bmatrix 1&0&1\\ 1&2&3\endbmatrix$.
\medskip

\end{enumerate} 

\end{prob} \begin{prob} \label{prob24.2} In each of the following, find the standard matrix of $T$ and use it to give a basis for $\ker T$ and $\im T$ and verify the conservation of dimension.

\medskip
\begin{enumerate}[a)]
\item $T:\R^2 \to \R^3$ defined by $T(x,y)=(x, y, x+y)$
\medskip
%
\item\sov $T:\R^3 \to \R^2$ defined by $T(x,y,z)=(2 z+x, y)$
\medskip
%
\item $T:\R^3 \to \R^3$ defined by $T(\vv)= \vv\times (1,2,3)$, where `$\times$' denotes the cross product.
\medskip
%
\item\sov $T:\R^3 \to \R^3$ defined by $T(\vv)= \proj_{(1,1,-1)}(\vv)$.
\medskip
%
\item $T:\R^3 \to \R^3$ defined by $T(\vv)= \vv-\proj_{(1,1,-1)}(\vv)$.
\medskip
%
\item\sov $T:\R^3 \to \R^3$ defined by $T(\vv)= \proj_H(\vv)$, where $H$ is the plane through the origin with normal $(1,1,0)$.
\medskip
%
\end{enumerate}

\end{prob} \begin{prob} \label{prob24.3} State whether each of the following is (always) true,
or is (possibly) false.    
   \smallskip    
\begin{enumerate}[$\bullet$]
\item If you say the statement may be false, you must give an explicit example.   
\item If you say the statement is true, you must give a clear explanation -   by quoting a theorem presented in class, or by giving a {\it proof valid for every  case}. 
\end{enumerate}
\medskip
\begin{enumerate}[a)]
\item If $T:\R^3 \to \R^2$ is linear, then $\ker T \not= \set{0}$.
\medskip
%
\item\sov If $T:\R^4 \to \R^2$ is linear, then $\dim \ker T \ge 2$.
\medskip
% 
\item If $T:\R^4 \to \R^5$ is linear, then $\dim \im T \le 4$.
\medskip
%
\item\sov If $T:\R^3 \to \R^2$ is linear, and $\set{\vv_1,\vv_2} \subset \R^3$ is linearly independent, then $\set{T(\vv_1),T(\vv_2)} \subset \R^2$ is linearly independent.
\medskip
%
\item If $T:\R^3 \to \R^2$ is linear, $\ker T=\set{\zero}$, and $\set{\vv_1,\vv_2} \subset \R^3$ is linearly independent, then $\set{T(\vv_1),T(\vv_2)} \subset \R^2$ is linearly independent.
\medskip
%
\item\sov If $T:\R^3 \to \R^3$ is linear, and $\ker T=\set{\zero}$, then $\im T=\R^3$.
\medskip
%
\item If $T:\R^2 \to \R^3$ is linear, and $\ker T=\set{\zero}$, then $\im T=\R^3$.

\end{enumerate}

\end{prob} \begin{prob} \label{prob24.4}$^\ast$ State whether each of the following defines a linear transformation.    
   \smallskip    
\begin{enumerate}[$\bullet$]
\item If you say it isn't linear, you must give an explicit example to illustrate.   
\item If you say it is linear, you must give a clear explanation -   by quoting a theorem presented in class, or by verifying the conditions in the definition {\it  in every  case}. 
\end{enumerate}
 \medskip
\begin{enumerate}[a)]
\item $T: \PP \to \PP$ defined by $T(p)=p'$, where $p'$ denotes the derivative of $p$.
\medskip
%
\item\sov $T: \PP \to \PP$ defined by $T(p)(t)=\dsize \int_0^tp(s) ds$.
\medskip
%
\item $\tr: \M_{2\,2} \to \R$ defined by $\tr\bmatrix a&b\\c&d\endbmatrix = a+d.$
\medskip
%
\item\sov $\det: \M_{2\,2} \to \R$ defined by $\det \bmatrix a&b\\c&d\endbmatrix = ad-bc.$
\medskip
%
\item $T: \F(\R) \to \R$ defined by $T(f)=f(1)$.
\medskip
%
\end{enumerate}
\end{prob}