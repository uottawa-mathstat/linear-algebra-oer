\section*{Problems}
\addcontentsline{toc}{section}{Problems}


\medskip {\bf Remarks:} 
\begin{enumerate}
\item A question with an asterisk `$ ^\ast$' (or two) indicates a bonus-level question.
 \item You must justify all your responses.
\end{enumerate}
\bigskip

%\centerline{\bf \underbar{}} 

\begin{prob} \label{prob21.1} Find the determinant of each of the following matrices. In the following, $\lambda$ represents a variable.  Use appropriate row and/or column operations where useful: the definition is the tool of last resort!
\medskip
\begin{enumerate}[a)]
\item 
$\bmatrix
2&-1\\3&0 \endbmatrix $
\medskip
%

\item\sov $\bmatrix 
2&-1&3\\3&0&-5\\1&1&2 \endbmatrix $
\medskip
% 30
\item 
$\bmatrix
1&1&1\\1&2&3\\0&1&1 \endbmatrix $
\medskip
% -1
\item\sov $\bmatrix 3&4&-
1\\1&0&3\\2&5&-4\endbmatrix$\medskip
% -10
\item 
$\bmatrix 0&1&0&0\\
2&0&0&0\\
0&0&3&0\\
0&0&0&4\\
\endbmatrix$ 
\medskip
% -24
\item\sov $\bmatrix \lambda-6&0&0\\0&\lambda&3\\0&4&\lambda+4\endbmatrix $\medskip
%-(\lam -6)(\lam +6)(\lam +2)
 \item $\bmatrix 2&0&0&0&0&0\cr0&1&0&0&0&0\cr0&0&0&1&0&0\cr
0&0&0&0&0&3\cr0&0&1&0&0&0\cr0&0&0&0&3&0\endbmatrix$
\medskip
%
\item\sov $\bmatrix
-\lambda&2&2\\ 2&-\lambda&2\\ 2&2&-\lambda \endbmatrix$
\medskip 
%

\end{enumerate}

\end{prob} \begin{prob} \label{prob21.2} If $\left|\begin{matrix} a&b&c\cr d&e&f\cr g&h&i\end{matrix} \right|=3$,
find 
\medskip
\begin{enumerate}[a)]
 
\item $\left|\begin{matrix}  d&e&f \\ a&b&c\\ g&h&i\end{matrix}\right|$
\medskip
%-3
\item\sov $\left|\begin{matrix}  b&a&c\\ e&d&f\\ h&g&i\end{matrix}\right|$
\medskip
%-3
\item $\left|\begin{matrix}  b&3a&c\\ e&3d&f\\ h&3g&i\end{matrix}\right|$
\medskip
%-9
\item\sov $\left|\begin{matrix}  b&3a&c-4b\\ e&3d&f-4e\\ h&3g&i-4h\end{matrix}\right|$
\medskip
%-9
\item $\left|\begin{matrix} 4g&a&d-2a\cr4h&b&e-2b\cr4i&c&f-2c \end{matrix}\right|$
%12
\medskip

\end{enumerate}

\end{prob} \begin{prob} 
\label{prob21.3} 

 

\begin{enumerate}[a)]


\item If $Q$ is a $3\times 3$ matrix and $\hbox{det}\,Q=2$, find $\det((3Q)^{-1})$.
\medskip
% $\frac{1}{54}
\item\sov If $B$ is a $4\times 4$ matrix and $\hbox{det}(2BB^T)=64$, find $|\det(3B^2B^T)|$.
\medskip
%
\item If $A$ and $B$ are $4\times 4$ matrices with $\det(A)=2$
and $\det(B)=-1$, find $\hbox{det}(3AB^TA^{-2}BA^TB^{-1})$.
\medskip
%
\item\sov Compute the determinant of
$\bmatrix 1&2\cr3&4 \endbmatrix\bmatrix 5&6\cr7&8\endbmatrix 
\bmatrix9&10\cr11&12\endbmatrix\bmatrix 13&14\cr15&16\endbmatrix.$
\medskip
%16
\item  Find all $x$ so that the matrix $\bmatrix 0&x&-
4\cr2&3&-2\cr1&4&1\endbmatrix$ is not invertible.
\medskip
%-5
\end{enumerate}
\end{prob} \begin{prob} \label{prob21.4} State whether each of the following is (always) true,
or is (possibly) false.    
   \smallskip    
\begin{enumerate}[$\bullet$]
\item If you say the statement may be false, you must give an explicit example.   
\item If you say the statement is true, you must give a clear explanation -   by quoting a theorem presented in class, or by giving a {\it proof valid for every  case}. 

\medskip In the following $A$ and $B$ are $n\times n$ matrices (with $n>1$) and $k$ is a scalar.
\end{enumerate}
\medskip
\begin{enumerate}[a)]
\item $\det (AB) = \det(A) \, \det(B)$
\medskip
%
\item\sov $\det (A +B) = \det(A) +\det(B)$
\medskip
%
\item $\det (k A)= k \det(A)$
\medskip
%
\item\sov $\det (k A)= k^n \det(A)$
\medskip
%
\item $\det  A^T = \det(A)$
\medskip
%
\item\sov If $A$ and $B$ are the same except the first row of $A$ is twice the first row of $B$, then $\det(A)=2 \det(B)$
\medskip
%
\item$^{\ast\ast}$ If $A=\bmatrix \cc_1 +\bb_1 & \cc_2 & \cdots& \cc_n \endbmatrix$ is written in block column form, meaning that the first column of $A$ is $\cc_1 +\bb_1$ and $\cc_2, \cc_3, \dots, \cc_n$ are the other columns of $A$, then $\det(A)=   \det \bmatrix \cc_1  & \cc_2 & \cdots& \cc_n \endbmatrix + \det \bmatrix  \bb_1 & \cc_2 & \cdots& \cc_n \endbmatrix $ 
\medskip
%

\end{enumerate}
\end{prob} \begin{prob} \label{prob21.5}
\medskip
\begin{enumerate}[a)]
\item 
\medskip If $A$ is any 2 by  2 matrix, and $\vv_1, \vv_2, \vv_3$ and $\vv_4$ are column vectors in $\R^3$ satisfying $$\bmatrix \vv_1 &\vv_2 \endbmatrix= A  \bmatrix \vv_3 &\vv_4 \endbmatrix,$$show that $\vv_1\times \vv_2 =(\det(A))\, \vv_3\times \vv_4$.
\medskip
%
\item\sov If   $\uu, \vv$ and $\ww$ are  vectors in $\R^3$, use properties of 3 by 3 determinants to show that $$ \uu\cdot \vv\times \ww=  \ww\cdot \uu\times \vv= \vv\cdot \ww\times \uu$$


\item Suppose $B$ is an $1\times n$,  matrix $D$ is an $n\times n$  matrix, $a\in \R$. Show that $\det \bmatrix a&B\\0&D\endbmatrix= a\,\det(D)$. (The matrix is  expressed here in block form.)
\medskip
%
\item$^\ast$\footnote{This is for those of you who know about {\it proofs by induction}.    Search for `Mathematical induction', for example.} Suppose $D$ is an $n\times n$  matrix and $B$ an $m\times n$ matrix. Show that $\det \bmatrix I_m&B\\0&D\endbmatrix= \det(D)$. (The matrix $\bmatrix I_m&0\\0&B\endbmatrix$,  of size $(m+n)\times (m+n)$, is expressed here in block form.)
\medskip
%

\item$^\ast$\footnote{Use the same technique as in the previous part.} Suppose $A$ is an $m\times m$  matrix. Show that $\det \bmatrix A&B\\0&I_n\endbmatrix= \det(A)$. 
\medskip
% 
\item Suppose $A, B$ and $D$ are respectively of sizes $m\times m$,   $m \times n$ and $n \times n$. Noting that $ \bmatrix A&B\\0&D\endbmatrix =\bmatrix I_m&0\\0&D\endbmatrix\bmatrix A&B\\0&I_n\endbmatrix$, show that $\det \bmatrix A&B\\0&D\endbmatrix= \det(A) \det(D)$. 
\medskip
%
\item\sov Suppose $A, B, C$ and $D$ are respectively of sizes $m\times m$,   $m \times n$, $n \times m$ and $n \times n$. Suppose that $D$ is invertible. Noting that $ \bmatrix A&B\\C&D\endbmatrix \bmatrix I_m&0\\-D^{-1}C&I_n\endbmatrix  =\bmatrix A-BD^{-1}C&B\\0&D\endbmatrix$, show that $\det \bmatrix A&B\\C&D\endbmatrix= \det (A-BD^{-1}C) \det(D)$.
\medskip
% 
\item Suppose $A, B, C$ and $D$ are $n\times n$ matrices with $D$ invertible and  $CD=DC$. Let $E=\bmatrix A&B\\C&D\endbmatrix$,  an $(2n)\times (2n)$ matrix, expressed here in block form. Use the last part and properties of the determinant to show that $\det \bmatrix A&B\\C&D\endbmatrix= \det (AD-BC)$.  \medskip
%
\item$^{\ast\ast\ast}$ \footnote{This is for those of you who know about {\it continuity}, and the fact that invertible $n\times n$ matrices are {\it dense} in all $n\times n$ matrices. Search for `Invertible\_matrix'. There is another proof of this identity that doesn't use the density argument, by J.R. Silvester, in {\it Determinants of block matrices}, Math. Gaz., 84(501) (2000), pp. 460-467.} Suppose $A, B, C$ and $D$ are $n\times n$ matrices with  $CD=DC$. Show that $\det \bmatrix A&B\\C&D\endbmatrix= \det (AD-BC)$.  \medskip
%
\item$^{\ast\ast}$ Suppose $A$ is a 3 by 3 matrix that satisfies $AA^T=I_3$. \medskip

\begin{enumerate}[i)]
\item Show that $A^TA=I_3$ as well.
\medskip
%
\item Suppose $\set{\ee_1, \ee_2, \ee_3}$ is the standard (orthogonal and orthonormal) ordered basis of $\R^3$. Using the fact that the dot product $\vv\cdot \ww$ is equal to the matrix product $\vv^T \ww$ (writing vectors as $3 \times 1 $ matrices), show that $\set{A\ee_1, A\ee_2, A\ee_3}$ is also an orthogonal set, which is indeed {\it orthonormal.}
\medskip
%
\item If $\uu$ and $\vv$ are any vectors in $\R^3$, use the Expansion theorem \ref{expansion} to show that $$\uu \times \vv=  \displaystyle \sum_{i=1}^3 (\uu \times \vv)\cdot \ee_i$$ and $$A\uu \times A\vv= \dsize\sum_{i=1}^3 (A\uu \times A\vv)\cdot A\ee_i.$$
\medskip
%
\item Recalling the definition of the 3 by 3 determinant, show that $$(\uu \times \vv)\cdot \ee_i=\det\bmatrix \uu&\vv&\ee_i\endbmatrix$$ and   $$(A\uu \times A\vv)\cdot A\ee_i=\det\bmatrix A\uu&A\vv&A\ee_i\endbmatrix,$$ where the matrices $\bmatrix \uu&\vv&\ee_i\endbmatrix$ and $ \bmatrix A\uu&A\vv&A\ee_i\endbmatrix$ on the right have been written in block column form.
\medskip
%
\item Recalling what you know about block multiplication, and properties of determinants, show that $$\det\bmatrix A\uu&A\vv&A\ee_i\endbmatrix=\det(A)\det \bmatrix  \uu& \vv& \ee_i\endbmatrix.$$
\medskip
%
\item Now prove that  for any vectors $\uu, \vv \in \R^3$, (when $AA^T=I_3$) $$A\uu \times A\vv =\det(A) \, (\uu\times \vv)$$
\medskip
%
\item Give an example of a matrix $A$, which does not satisfy $AA^T=I_3$ and two vectors $\uu,\vv \in \R^3$ for which $A\uu \times A\vv \not=\det(A) \, (\uu\times \vv)$
\medskip
%

\end{enumerate}

\end{enumerate}
\end{prob}

