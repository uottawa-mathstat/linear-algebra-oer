%%%%%%%%%%%%%%%%%%%%%%preface.tex%%%%%%%%%%%%%%%%%%%%%%%%%%%%%%%%%%%%%%%%%
% sample preface
%
% Use this file as a template for your own input.
%
%%%%%%%%%%%%%%%%%%%%%%%% Springer %%%%%%%%%%%%%%%%%%%%%%%%%% 

\preface
\addcontentsline{toc}{chapter}{Preface}

This volume grew from sets of lecture notes by Thierry Giordano, Barry Jessup,  and Monica Nevins for teaching the course \emph{Introduction to Linear Algebra} at the University of Ottawa.  This book is intended to serve as a text or companion to the course.


The approach we take in this book is not standard: we introduce vector spaces very early and only treat linear systems after a thorough introduction to  vector spaces. 

We do this for at least two reasons. Our experience in teaching variations of this course over the past 25 years  to thousands of students has taught us that the material on vector spaces, usually found toward the end of the course in a traditional textbook, is generally experienced by students as the most difficult part of the course. In a 12 week course, to have the most difficult material near the end of the course does not give most students enough opportunity to come to grips with the (seemingly) new ideas introduced when we meet vector spaces for the first time. 

In our experience, starting with  vector spaces within the first two weeks allows students much more time to appropriate the `big' ideas in linear algebra: the notions of the {\it set of all linear combinations of vectors} (the `span') and the {\it `linear independence' of vectors}. These notions lie at the heart of linear algebra and are usually experienced as new and challenging by students who see them for the first time. So, the sooner they see them, the better: we can use them, as well as the notions of {\it basis} and {\it dimension} in the rest of the course, in different contexts. In this way, it has been our experience that most students prefer to see the challenging material as early as possible, so that they have time to acquaint themselves with material that is genuinely new and different from what they've seen in high school.

Another reason to tackle vector spaces as soon as possible is to alert students to the fact that there {\it is} genuinely new and different material in the course! If one begins with material on linear systems many students have seen in high school, (in low dimensions, at least),  some can easily fall prey to the idea that there's not much to this course, will and will be at caught later on when vector spaces come along.
 

