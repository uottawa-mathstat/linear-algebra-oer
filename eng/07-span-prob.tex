
  

 \begin{prob} \label{prob06.1} Justify your answers to the following:
\medskip
\begin{enumerate}[a)]
\item Is the vector $(1,2)$ a linear combination of $(1,0)$ and $(1,1)$?
\medskip
%
\item\sov Is the vector $(1,2)$ a linear combination of $(1,1)$ and $(2,2)$?
\medskip
%
\item Is the vector $(1,2)$ a linear combination of $(1,0)$ and $(2,2)$?
\medskip
%
\item\sov Is the vector $(1,2,2,3)$ a linear combination of $(1,0,1,2)$ and $(0,0,1,1)$?
\medskip
%
\item Is the vector $(1,2,2,4)$ a linear combination of $(1,0,1,2)$ and $(0,0,1,1)$?
\medskip
%
\item\sov Is the matrix $\bmatrix 1&2\\2&3\endbmatrix $ a linear combination of $\bmatrix 1&0\\1&2\endbmatrix$ and $ \bmatrix 0&1\\0&1\endbmatrix $?
\medskip
%
\item Is the matrix $\bmatrix 1&2\\2&4\endbmatrix $ a linear combination of $\bmatrix 1&0\\1&2\endbmatrix$ and $ \bmatrix 0&1\\0&1\endbmatrix $?
\medskip
%
\item\sov Is the polynomial $1+x^2$ a linear combination of $1+x-x^2$ and $x$?
\medskip
%
\item Is the polynomial $1+x^2$ a linear combination of $1+x$ and $1-x$?
\medskip
%
\item\sov Is the function $\sin x$ a linear combination of the constant function $1$ and $\cos x$?
\medskip
%
\item Is the function $\sin^2 x$ a linear combination of the constant function $1$ and $\cos^2 x$?
\medskip
%
\item\sov If $u$, $v$ and $w$ are any vectors in  a vector space $V$, is $u-v$ a linear combination of $u$, $v$ and $w$?
\medskip
%
\item If $u$, $v$ and $w$ are any vectors in  a vector space $V$, is $w$ always a linear combination of $u$, $v$ ?
\medskip 
%
 
\end{enumerate}

\end{prob} \begin{prob} \label{prob06.2}  Justify your answers to the following:
\medskip
\begin{enumerate}[a)]
\item Does the vector $(3,4)$ belong to $\spn\{(1,2)\}$?\medskip
%
\item\sov Is  `$(3,4) \in\spn\{(1,2)\} $' true? (Note that this is the same as the previous part, written using mathematical notation.)
\medskip

%
\item Is `$(2,4) \in \spn\{(1,2)\}$' true?
\medskip
%

\item\sov How many vectors belong to $\spn\{(1,2)\}$?
\medskip
%
\item Are the subsets $\set{(1,2)}$ and  $\spn\{(1,2)\}$ equal?
\medskip
%
\item\sov Is  $\set{(1,2)}$ a subset of  $\spn\{(1,2)\}$?
\medskip
%
\item 
\medskip If $\vv$ is a non-zero vector in a vector space, show that there are infinitely many vectors in $\spn\{\vv\}$. (Is this true if $\vv=\zero$?)\medskip
%
\item\sov  Suppose $S$ is a subset of a vector space $V$.  If $S= \spn\, S$,  explain why $S$ must be a subspace of $V$.  
\medskip
%

\item$^\ast$ Suppose $S$ is a subset of a vector space $V$.  If $S$ is actually a {\it subspace} of $V$, show that  $S= \text{span}\, S$.  

\medskip Note that to show that $S= \text{span}\, S$, you must establish two facts:
\begin{enumerate}[(i)]
\item If $\ww\in S$, then $\ww\in \text{span}\, S$, and

\item If $\ww\in \text{span}\, S$, then $\ww\in S$.
\end{enumerate}
\medskip
%

\end{enumerate}

\end{prob} \begin{prob} \label{prob06.3} Give \underbar{two} distinct {\it finite} spanning sets for each of the following subspaces.
\medskip

\begin{enumerate}[a)]

\item  $\set{(2x, x) \in \R^2\st x\in \R}$\medskip
% no

\item\sov  $\set{(x, y) \in \R^2\st 3x - y=0 }$ \medskip
%

 
 
\item  $\set{(x, y, z) \in \R^3\st x+y-2z=0 }$   \medskip
%

\item\sov  $\set{(x, y, z, w) \in \R^4\st x-y+z-w=0 }$   \medskip
%

\item  $\Bigg\{  \bmatrix a&b\\ c&d\endbmatrix \in \M_{2 \,2}(\R) \;\Bigg|\; b=c\Bigg\}$.\medskip \medskip
%
\item\sov  $\Bigg\{  \bmatrix a&b\\ c&d\endbmatrix \in \M_{2 \,2}(\R) \;\Bigg|\;a=d=0\quad \text{and}\quad b=-c  \Bigg\}$.\medskip \medskip
%
\item  $\Bigg\{  \bmatrix a&b\\ c&d\endbmatrix \in \M_{2 \,2}(\R) \;\Bigg|\; a+d=0\Bigg\}$. \medskip
%

\item\sov  $\Bigg\{  \bmatrix a&0\\ 0&b\endbmatrix \in \M_{2 \,2}(\R) \;\Bigg|\;  a, b \in \R\Bigg\}$. \medskip
%


\item  $\Bigg\{  \bmatrix 0&b\\ -b&0\endbmatrix \in \M_{2 \,2}(\R) \;\Bigg|\; b \in \R\Bigg\}$.      \medskip
%

\item\sov  $\Bigg\{  \bmatrix a&b\\ c&d\endbmatrix \in \M_{2 \,2}(\R) \;\Bigg|\; a+b+c+d=0\Bigg\}$.      \medskip
% 
\item  $ \PP_2=\set{p\st p \text{ is a polynomial function with} \deg(p)\le 2} $.  \medskip
%
\item\sov  $ \PP_n=\set{p\st p \text{ is a polynomial function with} \deg(p)\le n} $.  \medskip
%
\item  $ \set{p \in \PP_2 \st  p(2)=0}$.  \medskip
%

\item\sov  $ \set{p \in \PP_3 \st  p(2)=p(3)=0}$.  \medskip
%


\item  $ \set{p \in \PP_2 \st  p(1)+p(-1)=0}$.      \medskip
%
\item\sov  $ \spn\{\sin x, \cos x\}$.      \medskip
%
\item  $ \spn\{1, \sin x, \cos x\}$.      \medskip
%
 \item\sov  $ \spn\{1, \sin^2 x, \cos^2 x\}$.      \medskip
%

\item$^\ast$  $\set{(x, x-3) \in \R^2\st x\in \R}$,  equipped with the \underbar{\it non-standard operations:--} Addition: $$(x,y) \tilde+ (x',y')=(x+x', y+y +3).$$ Multiplication of vectors  by  scalars $k\in \R$: 
$$k \odot (x,y)=(kx, ky+3k-3).$$    
%
\item$^\ast$  $\set{(x, y, z) \in \R^3\st x+2y+z=2 }$; $V=\R^3$, \underbar{\it Non-standard operations:--} Addition: $$(x,y,z) \tilde+ (x',y',z')=(x+x', y+y,z+z'-2).$$ Multiplication of vectors  by  scalars $k\in \R$: $k\odot (x,y,z)=(kx, ky, kz-2k+2)$.\medskip 
% 
\end{enumerate}
{\it(Hint for parts (m)\&(n): Recall the Factor/Remainder theorem from high school: if $p$ is a  polynomial in the variable $x$ of degree at least 1,  and $p(a)=0$ for some $a \in \R$, then $p$ has a factor of $x-a$, i.e., $p(x)=(x-a)q(x)$, where $q$ is a polynomial with $\deg(q)=\deg(p)-1$.)}


\end{prob} \begin{prob} \label{prob06.4} Justify your answers to the following:
\medskip
\begin{enumerate}[a)]
\item Suppose $\uu$ and $\vv$ belong to a vector space $V$. Show carefully that $\spn\{\uu,\vv\}=\spn\{\uu, \uu+\vv\}$. That is , you must show two things:
\begin{enumerate}[(i)]
\item If $\ww\in \spn\{\uu,\vv\}$, then $\ww \in \spn\{\uu, \uu+\vv\}$, and
\item If $\ww\in \spn\{\uu, \uu+\vv\}$,  then $\ww\in \spn\{\uu,\vv\} $.
\end{enumerate}
\medskip
%
\item\sov Suppose $\uu$ and $\vv$ belong to a vector space $V$. Show carefully that $\spn\{\uu,\vv\}=\spn\{\uu-\vv, \uu+\vv\}$. That is , you must show two things:
\begin{enumerate}[(i)]
\item If $\ww\in \spn\{\uu,\vv\}$, then $\ww \in \spn\{\uu-\vv, \uu+\vv\}$, and
\item If $\ww\in \spn\{\uu-\vv, \uu+\vv\}$,  then $\ww\in \spn\{\uu,\vv\} $.
\end{enumerate}
\medskip
%
%
\item  Suppose $\uu \in \spn\{\vv,\ww\}$. Show carefully that $\spn\{\vv,\ww\}=\spn\{\uu,\vv,\ww\}$.
\medskip
%

\item\sov  Suppose $\spn\{\vv,\ww\}=\spn\{\uu,\vv,\ww\}$. Show carefully that $\uu \in \spn\{\vv,\ww\}$. 
\medskip
%
\item Show that $x^2 \notin \spn\{1,x\}$.\footnote{\it  Hint: Proceed by contradiction: Suppose $x^2 \in \spn\{1,x\}$. Write down explicitly what this means, re-write the equation as $q(x)=0$, where $q(x)$ is some quadratic polynomial. Now, remember: every non-zero polynomial of degree 2 has at most 2 distinct roots. Look again at the equation $q(x)=0$, and ask yourself how many roots this equation tells you that $q$ has. Now find your contradiction.}
\medskip
%
\item\sov  Show that $x^{n+1} \notin \PP_n$. \footnote{\it  Hint: Generalize the idea in the previous hint, recalling that every non-zero polynomial of degree $n+1$ has at most $n+1$ distinct roots.}
\medskip
%
\item$^\ast$ Show that $\PP$ does not have a finite spanning set. \footnote{\it  Hint: Proceed by contradiction again. Suppose it did, say,  $\PP=\spn\{p_1, \dots, p_k\}$, for some polynomials $p_1, \dots, p_k$. Now define $n=\max\set{\deg(p_1), \dots, \deg(p_k)}$. Now show that $x^{n+1}\notin\spn\{p_1, \dots, p_k\}$, using the same argument as in the previous part. But of course $x^{n+1} \in \PP$, so there's your contradiction. }
\medskip
%
\item\sov Assume for the moment (we'll prove it later) that if $W$ is a subspace of $V$, and $V$ has a finite spanning set, then so does $W$. 

Use this fact and the previous part to prove that $\F(\R)$ does not have a finite spanning set.
\medskip
%
\end{enumerate}


\end{prob}
