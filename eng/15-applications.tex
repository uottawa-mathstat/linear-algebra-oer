\chapter{Applications of Solving Linear Systems}\label{chapter:13applications}

\section{The rank of a matrix, and its importance}

In the last chapter, we defined the rank of a matrix.

\begin{definition}  The \emph{rank} of a matrix $A$, denoted $\rnk(A)$, is the number of leading ones (`pivots') in any REF of $A$.
\end{definition}

\begin{myexample} Since  $\mat{1 & 4 & 1 \\ 2 & 3 & 0} \sim \mat{1 & 4 & 1 \\ 0 & -5 & -2} \sim \mat{1 & 4  & 1\\ 0 & 1  & \frac25}$, \emph{each} of these matrices has rank $2$. \end{myexample}

\standout{The rank doesn't change when one does elementary row operations; if you're using the Gaussian algorithm, it just
becomes easier to see.}

\begin{myexample} The matrix $\mat{1 & 0 & 0 \\ 0 & 1 & 0 \\ 0 & 0 & 1}$ has rank $3$.\end{myexample}

Note that the rank of a matrix can never exceed the number of rows or 
columns of the matrix, since each leading 1 lies in a different
row and column from all the others.

We also mentioned last time that you could see whether a system is
consistent or inconsistent from the rank of $A$ versus the rank of $[A|\bb]$.  Namely,
suppose you reduce the coefficient matrix $A$ in $[A|\bb]$ to RREF.  Then either you get 
$$
\mat{ * & \cdots & * &|& *\\
0 & \cdots & 0 &|& 0\\
\vdots & \cdots & \vdots&|& \vdots\\
0 & \cdots & 0 &|& 0}
$$
(where the top ($*$)
part means rows with leading ones in the coefficient matrix part)
(a consistent system) or you get
$$
\mat{ * & \cdots & * &|& *\\
0 & \cdots & 0 &|& a_1\\
\vdots & \cdots & \vdots&|& \vdots\\
0 & \cdots & 0 &|& a_k\\
0 & \cdots & 0 &|& 0\\
\vdots & \cdots & \vdots&|& \vdots\\
0 & \cdots & 0 &|& 0}
$$ where $a_i\not=0$
(which tells you you have an inconsistent system).   You can spot the difference by comparing
the ranks of the coefficient matrix and of the augmented matrix:
in the first case, $\rnk(A) = \rnk([A|\bb])$ since all the leading 1s
occured in the coefficient matrix part, and none in the augmented column.
In the second case, $\rnk([A|\bb])=\rnk(A) +1$, since the non-zero entry $a_1$ in the last  column will lead to a (single) new
 leading 1.


\standout{So: $$\rnk(A) \leq \rnk([A|\bb]) \leq \rnk(A)+1$$}

... and where the equality holds tells you whether or not your system is consistent.

\begin{myexample} In a homogeneous linear system, you can never get a nonzero
entry in the augmented column (it's all zeros) so certainly you can
never have a leading 1 there.  Thus $\rnk(A) = \rnk([A|0])$, and the
system is consistent.  (We knew that already.)\end{myexample}


In summary:  suppose $[A|\bb]$ is the augmented matrix of a linear system.
Then:
\begin{itemize}
\item The system is inconsistent if and only if $\rnk(A) < \rnk([A|\bb])$.
\item The system has a unique solution if and only if $\rnk(A)=\rnk([A|\bb])$
{\bf and} $\rnk(A) = $\# columns of $A$.
\item The system has infinitely many solutions   if and only if $\rnk(A)=\rnk([A|\bb])$
{\bf and}  $\rnk(A) < $\# columns of $A$.
\end{itemize}


\section{Application I:  Solving network and traffic flow problems}

Let's consider several different applications of linear systems and
our Gaussian elimination method.  The first is a common application of
linear systems to solving traffic flow/network problems.  The idea
is that you can model the internal flow of a network just by understanding
its inputs and outputs and how traffic is restricted in between.  
You don't expect a unique solution or even finitely many (in theory, of course) because there may be loops --  see problem \ref{prob13.3}, for example.

\begin{myprob} The diagram in the figure below represents a network of one-way
streets.  The numbers on the figure represent the flow of traffic (in cars per hour) along each street, and the intersections are labeled $A$, $B$, $C$ and $D$.  The arrows indicate the direction of the flow of traffic.  The variables $x_1, x_2, x_3, x_4$ represent the (unknown) level of traffic on certain streets.

Analyse this system to answer the following questions:
\begin{enumerate}[(a)]
\item What is the maximum flow along $AB$?
\item If $DA$ is blocked for roadwork, will there be a traffic jam?
\item What if $BC$ is blocked for roadwork?
\end{enumerate}

\begin{mysol}\mbox{}

%\begin{figure}[ht]
\begin{center}
\setlength{\unitlength}{1 true in}
\parbox[t][4in][b]{4in}{
\begin{picture}(4,4)
\put(0,1){\line(1,0){4}}
\put(0,3){\line(1,0){4}}
\put(1,1){\line(0,1){2}}
\put(3,0){\line(0,1){3}}
\put(0.5,1.1){\text{$\rightarrow$}}
\put(2,1.1){\text{$\rightarrow$}}
\put(3.5,1.1){\text{$\rightarrow$}}
\put(0.5,3.1){\text{$\rightarrow$}}
\put(2,3.1){\text{$\rightarrow$}}
\put(3.5,3.1){\text{$\rightarrow$}}
\put(1.1,2){\text{$\uparrow$}}
\put(3.1,2){\text{$\downarrow$}}
\put(2.9,0.5){\text{$\downarrow$}}

\put(0.5,0.8){300}
\put(.95,3.1){A}
\put(0.5,2.8){300}
\put(2.95,3.1){B}
\put(2.85,1.1){C}
\put(0.85,1.1){D}
\put(2,2.8){$x_1$}
\put(3.5,0.8){200}
\put(2,0.8){$x_4$}
\put(3.5,2.8){100}
\put(3.1,0.2){300}
\put(0.8,2){$x_2$}
\put(2.8,2){$x_3$}

%\multiput(0,1.95)(.25,0){17}{\line(0,1){.1}} 
%\multiput(1.95,0)(0,.25){17}{\line(1,0){.1}} 
%\put(0,0.5){\line(2,1){2}}
%\put(2,0){\line(-1,1){1.8}}
\end{picture}}\end{center}

%\caption{Traffic flow diagram}\label{Fig:traffic1}
%\end{figure}


The total flow in is $600$ (cars per hour, say) while
the total flow out is $600$.  Good; let us take that
as our hypothesis for all the intersections as well:
flow in  = flow out  (Kirchhoff's first law).

We have labeled with $x_1$, $x_2$, $x_3$, $x_4$ 
the street segments on which we don't know the flow
of traffic.  So let us set up the flows by writing
down the equation for each intersection (starting
at the northwest and working clockwise):
\begin{alignat*}{2}
\text{\bf Intersection}\quad& \text{\bf Flow in}\quad&&=\quad\text{\bf Flow out}\\
A\qquad \quad&300+x_2 &&=\quad x_1\\
B\qquad \quad &x_1 &&=\quad x_3+100\\
C\qquad \quad &x_3+x_4 &&=\quad 200 + 300\\
D\qquad \quad &300 &&=\quad x_2+x_4.
\end{alignat*}
This is the linear system
\begin{align*}
x_1-x_2 &= 300\\
x_1-x_3 &= 100\\
x_3+x_4 &= 500\\
x_2+x_4 &= 300
\end{align*}
with augmented matrix
$$
\mat{
1 & -1 & 0 & 0 & |& 300\\
1 & 0 & -1 & 0 & |& 100\\
0 & 0 & 1 & 1  &|& 500\\
0 & 1 & 0 & 1  &|& 300}
$$ 
This reduces to
$$
\mat{
1 & 0 & 0 & 1 & |& 600\\
0 & 1 & 0 & 1 & |& 300\\
0 & 0 & 1 & 1  &|& 500\\
0 & 0 & 0 & 0  &|& 0}
$$
So our solution is
$$
\mat{x_1\\x_2\\x_3\\x_4} = \mat{600-s\\300-s\\500-s\\s}
$$
for any $s\in \R$.

Now that we have found a solution to our linear system, let us consider the questions we wanted to answer.
\begin{enumerate}[(a)] 
\item Note that since the streets are one-way, each $x_i\ge 0$, for $1\le i\le 4$. When you implement these constraints, you find that $0\le s\le 300$. So for example, the maximum flow along $AB$ is 300 cars per hour (when the flow along $DA$ is zero). 
\item If $DA$ is blocked for roadwork, then we are setting $x_2=0$. This means $s=300$; we see that the solution is
$(x_1,x_2,x_3,x_4) = (300,0,200,300)$, which is fine.  There are no parameters, meaning that cars now have no choices about how to navigate the network, but they make it.
\item If $BC$ is blocked for roadwork, however, then $x_3=0$.  This means $s=500$,
which gives $(x_1,x_2,x_3,x_4) = (100,-200,0,500)$ --- meaning we would have
to allow two-way traffic on the street $DA$ to accommodate the traffic.
\end{enumerate}
In summary:  we applied the general solution to the model to first establish some additional constraints on our parameters (here, positive flow) and then used this constrained general solution to easily work out
various scenarios in traffic flow (without having to solve the
system all over again!).
\end{mysol}\end{myprob}

\section{Application II: testing scenarios}

\begin{myprob} Find all values of $k$ for which the following
linear system has (a) no solution, (b) a unique solution, and
(c) infinitely many solutions.
\begin{align*}
kx + y + z &=1\\
x +ky + z &= 1\\
x+y+kz &= 1.
\end{align*}

\begin{mysol}  This is a linear system in the variables $x,y,z$.  (It is
{\it not  linear in} $k,x,y,z$.  So this is one way to tackle a special
kind of nonlinear system with linear algebra.)

Let's use row reduction.  The augmented matrix is
$$
\mat{k & 1 & 1 &|& 1\\
1&k&1&|&1\\
1&1&k&|&1}
$$
Now be careful!  We can't divide by $k$, because we don't
know if it's zero or not.  So either we split into two
cases right now ($k=0$ and $k\neq0$) or else we use a 
different row to give us our leading 1.  
$$
\mt{R_1 \leftrightarrow R_2}
\mat{1&k&1&|&1\\
k & 1 & 1 &|& 1\\
1&1&k&|&1}
\mt{\sim \\ -kR_1+R_2\to R_2 \\ -R_1+R_3 \to R_3}
\mat{
1&k&1&|&1\\
0 & 1-k^2 & 1-k &|& 1-k\\
0&1-k&k-1&|&0}
$$
Now we have no choice:  there are variables in both entries from
which we need to choose our leading 1.  

Case 1:  $k = 1$.  Then our matrix is
$$
\mat{
1&1&1&|&1\\
0 & 0 & 0 &|& 0\\
0&0&0&|&0}
$$
which is in RREF and has infinitely many solutions.  So (c) includes
the case $k=1$.

Case 2:  $k \neq 1$.  That means $k-1 \neq 0$ so we can divide by it:
$$
\mt{ \frac{1}{1-k} R_2\to R_2 \\ \frac{1}{1-k}R_3\to R_3}
\mat{
1&k&1&|&1\\
0 & 1+k & 1 &|& 1\\
0&1&-1&|&0}
\mt{R_2 \leftrightarrow R_3}
\mat{
1&k&1&|&1\\
0&1&-1&|&0\\
0 & 1+k & 1 &|& 1}
$$
And now continue row reducing:
$$
\mt{\sim \\ -(1+k)R_2+R_3\to R_3}
\mat{
1&k&1&|&1\\
0&1&-1&|&0\\
0 & 0 & 2+k &|& 1}
$$
For the third column:  if $k = -2$, then we cannot get a leading
1 in the third column; in fact, if $k=-2$ then the system is 
inconsistent.  

Otherwise, we can divide $R_3$ by $2+k$ to get a leading 1 in
the third column; and then we deduce that we have a unique solution.

Conclusion:  if $k=1$, infinitely many solutions.  If $k=-2$, there is no
solution.  If $k \neq 1$ and $k \neq -2$ then there is a unique
solution.
\end{mysol}\end{myprob}

Notice how you get a unique solution ``most of the time'' --- this
coincides with the idea that if you take three random planes in
$\R^3$ then most of the time they should intersect in a single point.

\section{Application III:  solving vector equations}

\begin{myprob} Does the set $\{(1,2,3), (4,5,6), (7,8,9)\}$ span $\R^3$?

\begin{mysol} We need to answer the question:  given an arbitrary vector
$(x,y,z)\in\R^3$, do there exist scalars $a_1, a_2, a_3$ such
that
$$
a_1 \mat{1\\2\\3} + a_2\mat{4\\5\\6} + a_3 \mat{7\\8\\9} = \mat{x\\y\\z}?
$$
When you equate components on each side, we obtain
\begin{align*}
a_1 +4a_2 + 7a_3 &= x\\
2a_1+5a_2+ 8a_3 &=y\\
3a_1+6a_2 +9a_3 &= z
\end{align*}
which corresponds to the augmented matrix
$$
\mat{1 & 4 & 7 &|& x\\
2 & 5 & 8 &|& y\\
3 & 6 & 9 &|& z}
$$
Notice how the columns of this matrix correspond to the vectors in
our linear system!

We row reduce this linear system, and get
$$
\mat{1 & 4 & 7 &|& x\\
0 & 1 & 2 &|& -(y-2x)/3\\
0 & 0 & 0 &|& x-2y+z
}
$$
So:  the linear system is consistent if and only if $x-2y+z=0$.
But this says that the span of the given three vectors is only
a plane in $\R^3$ and not all of $\R^3$ (and this method even
gave us the equation of that plane!).
\end{mysol}\end{myprob}

Notice also that by setting $(x,y,z)=(0,0,0)$ in the preceding
example, you could deduce that the dependence equation has
infinitely many solutions, and thus that these three
vectors are linearly dependent.




