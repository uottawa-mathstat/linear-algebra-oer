
\begin{sol}{prob21.1} Find the determinants of the following matrices:
\medskip

(b) $A=\bmatrix 
2&-1&3\\3&0&-5\\1&1&2 \endbmatrix $
\medskip
\soln  $\det A=30$

(d) $A=\bmatrix 3&4&-
1\\1&0&3\\2&5&-4\endbmatrix$\medskip
\soln  $\det A= -10$

(f) $A=\bmatrix \lambda-6&0&0\\0&\lambda&3\\0&4&\lambda+4\endbmatrix $
\soln $\det A= (\lam -6)(\lam +6)(\lam +2)$
\medskip
 
(h) $A=\bmatrix
-\lambda&2&2\\ 2&-\lambda&2\\ 2&2&-\lambda \endbmatrix$
\soln $\det A= (\lam -4)(\lam +2)^2$
\medskip
  
\end{sol}

\begin{sol}{prob21.2} If $\left|\begin{matrix} a&b&c\cr d&e&f\cr g&h&i\end{matrix} \right|=3$,
find 
\medskip

(b) $A=\left|\begin{matrix}  b&a&c\\ e&d&f\\ h&g&i\end{matrix}\right|$
\soln $\det A= -3$, since $C_1 \leftrightarrow C_2$ (giving a change of sign) brings this matrix back to the original one.
\medskip
%-3

(d) $\left|\begin{matrix}  b&3a&c-4b\\ e&3d&f-4e\\ h&3g&i-4h\end{matrix}\right|$
\soln $\det A= -9$, since $C_1+C_3\to C_3$ (no change to the determinant) followed by noting the factor of 3 in $C_2$ (giving a factor of 3 to the determinant) yields the determinant in part (b).
\medskip

\end{sol}
\begin{sol}{prob21.3} 

 

(b) If $B$ is a $4\times 4$ matrix and $\det (2BB^T)=64$, find $|\det(3B^2B^T)|$.

\soln Since $64=\det (2BB^t)=2^4 (\det B)^2$, $\det B=\pm 2$. Hence $|\det(3B^2B^t)|=3^4|\det(B^2B^t)|=3^4(|\det B|)^3=81( 8)=648$
\medskip
%

(d) Compute the determinant of
$A=\bmatrix 1&2\cr3&4 \endbmatrix\bmatrix 5&6\cr7&8\endbmatrix 
\bmatrix9&10\cr11&12\endbmatrix\bmatrix 13&14\cr15&16\endbmatrix.$

\soln 
\begin{equation*}
\begin{split}
\det A &=\det\Big(\bmatrix 1&2\cr3&4 \endbmatrix\bmatrix 5&6\cr7&8\endbmatrix 
\bmatrix9&10\cr11&12\endbmatrix\bmatrix 13&14\cr15&16\endbmatrix\Big)\\&= \det \bmatrix 1&2\cr3&4 \endbmatrix \det \bmatrix 5&6\cr7&8\endbmatrix  \det \bmatrix9&10\cr11&12\endbmatrix \det \bmatrix 13&14\cr15&16\endbmatrix\\
&= (-2) (-2) (-2) \det \bmatrix 13&14\cr2&2\endbmatrix\\
&=(-2) (-2) (-2) (-2)\\
&=16
\end{split}\end{equation*}
\medskip


\end{sol}

\begin{sol}{prob21.4} State whether each of the following is (always) true,
or is (possibly) false.    
   \smallskip    
\begin{enumerate}[$\bullet$]
\item If you say the statement may be false, you    must give an explicit example.   
\item If you say the statement is true, you must give a clear explanation -   by quoting a theorem presented in class, or by giving a {\it proof valid for every  case}. 

\medskip In the following $A$ and $B$ are $n\times n$ matrices (with $n>1$) and $k$ is a scalar.
\end{enumerate}
\medskip

(b) $\det (A +B) = \det A +\det B$

\soln This is usually false. For example if $A=\bmatrix 1&0\cr0&0\endbmatrix$ and $B=\bmatrix 0&0\cr0&1\endbmatrix$, then $$\det(A+B)=\det \bmatrix 1&0\cr0&1\endbmatrix =1 \not= 0=0+0= \det A + \det B.$$ 
\medskip

(d) $\det (k A)= k^n \det A$

\soln This is always true, as we saw in class: multiplication of one row (or column) of $A$ by $k$ changes the determinant by a factor of $k$, so multiplication of $n$ rows (or $n$ columns) --which is the same  as multiplying the matrix $A$ by $k$ -- will change the determinant by $n$ factors of $k$, i.e., by a factor of $k^n$.
\medskip
%

(f) If $A$ and $B$ are the same except the first row of $A$ is twice the first row of $B$, then $\det A=2 \det B$.

\soln This is always true, as we saw in class. (Expand $\det A$ along the first row.)
\medskip

\end{sol}\begin{sol}{prob21.5}
\medskip
 

(b) If   $u, v$ and $w$ are  vectors in $\R^3$, use properties of 3 by 3 determinants to show that $$ u\cdot v\times w=  w\cdot u\times v= v\cdot w\times u$$

\soln We know, since it's the {\it definition} of the determinant of a 3 by 3 matrix, that $u\cdot v\times w =\det \bmatrix u\\v\\w\endbmatrix$, where we've written the vectors as row vectors and the matrix in block row form. So $w\cdot u\times v=\det \bmatrix w\\u\\v\endbmatrix =-\det \bmatrix u\\w\\v\endbmatrix=\det \bmatrix u\\v\\w\endbmatrix$ (two row interchanges: $R_1
\leftrightarrow R_2,R_2
\leftrightarrow R_3 $).  Similarly, $v\cdot w\times u=  \det \bmatrix v\\w\\u\endbmatrix =-\det \bmatrix u\\w\\v\endbmatrix=\det \bmatrix u\\v\\w\endbmatrix$ (two row interchanges again: $R_1
\leftrightarrow R_3,R_2
\leftrightarrow R_3 $).

\medskip

(h) Suppose $A, B, C$ and $D$ are respectively of sizes $m\times m$,   $m \times n$, $n \times m$ and $n \times n$. Suppose that $D$ is invertible. Noting that $ \bmatrix A&B\\C&D\endbmatrix \bmatrix I_m&0\\-D^{-1}C&I_n\endbmatrix  =\bmatrix A-BD^{-1}C&B\\0&D\endbmatrix$, show that $\det \bmatrix A&B\\C&D\endbmatrix= \det (A-BD^{-1}C) \det D$.

\soln This follows from (g) and the multiplicative property of determinants.
\medskip


\end{sol}

