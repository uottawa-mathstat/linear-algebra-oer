%%%%%%%%%%%%%%%%%%%%%part.tex%%%%%%%%%%%%%%%%%%%%%%%%%%%%%%%%%%
% 
% sample part title
%
% Use this file as a template for your own input.
%
%%%%%%%%%%%%%%%%%%%%%%%% Springer %%%%%%%%%%%%%%%%%%%%%%%%%%

\begin{partbacktext}
\part{Vector Spaces}
\noindent 
Given that the algebra of $\R^2$ and $\R^3$  extended so easily to $\R^n$, we ask ourselves:  what other kinds of mathematical objects behave (algebraically speaking) just like $\R^n$?   We formulate this question precisely in the first
chapter, and proceed to uncover many very familiar examples.

\end{partbacktext}
