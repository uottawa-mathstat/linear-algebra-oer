\section*{Problems}
\addcontentsline{toc}{section}{Problems}

 


\medskip {\bf Remarks:} 
\begin{enumerate}
\item A question with an asterisk `$ ^\ast$' (or two) indicates a bonus-level question.
 \item You must justify all your responses.
\end{enumerate}
\bigskip


\begin{prob} \label{prob23.1} 
\begin{enumerate}[a)]
\item For each of the matrices $A$  in Problem \ref{prob22.1}, if possible, find an invertible matrix $P$ and a diagonal matrix $D$ such that $P^{-1}AP =D$. If this is not possible, explain why. (Solutions to parts b, d, f and h are available.) 

\item Use the fact that $A=\bmatrix 
1&0&1\\0&1&0\\1&1&1 \endbmatrix $ is diagonalizable to compute  $A^{10^{1000}}$ before the sun becomes a red giant and (possibly) engulfs the earth.\footnote{You have between 5 and 6 billion years, but it should only take you less than 5 minutes.} 

\end{enumerate}

 

\end{prob} \begin{prob} \label{prob23.2} State whether each of the following is (always) true,
or is (possibly) false.     
   \smallskip    
\begin{enumerate}[$\bullet$]
\item If you say the statement may be false, you must give an explicit example.   
\item If you say the statement is true, you must give a clear explanation -   by quoting a theorem presented in class, or by giving a {\it proof valid for every  case}. 
\end{enumerate}

\begin{enumerate}[a)]
\item If $3$ is an eigenvalue of an $n \times n$ matrix $A$, there must be a non-zero vector $\vv \in \R^n$ with $Av=3v$.
\medskip
%
\item\sov The matrix $\bmatrix 0&-1\\1&0\endbmatrix$ has no real eigenvalues.
\medskip
%

\item The matrix $\bmatrix -1&1\\0&-1\endbmatrix$ is
diagonalizable.
\medskip
%
\item\sov If $0$ is an eigenvalue of  $n \times n$ matrix  $A$, then $A$ is
not invertible.
\medskip
%
\item If an  $n \times n$ matrix  $A$ is
not invertible, then $0$ is an eigenvalue of $A$.
\medskip
%
\item\sov Every invertible matrix is diagonalizable.
\medskip
%
\item Every diagonalizable  matrix is invertible.
\medskip
%
\item\sov If an $n \times n$ matrix has $n$ distinct eigenvalues, then the matrix is diagonalizable. 
\medskip
%
\item If an $n \times n$ matrix is diagonalizable then it must have $n$ distinct eigenvalues. 
\medskip
%
\item\sov\footnote{ Hint: Use the fact that we know $\det(A-\lam I_n)=(-1)^n(\lam-\lam_1)(\lam-\lam_2)\dots(\lam-\lam_n) $.} If an  $n \times n$ matrix  $A$ has eigenvalues $\lam_1, \dots ,\lam_n$, then $\det A=\lam_1 \dots\lam_n $.
\medskip
%

\item$^{\ast}$ If $\vv$ and $\ww$ are eigenvectors of a symmetric matrix $A$ (i.e. $A=A^T$) which correspond to different eigenvalues, then $\vv \cdot \ww=0$.\footnote{ Hint:  Remember that because $A$ is symmetric, $A\vv\cdot \ww=\vv\cdot A\ww$ (by Problem~\ref{prob14.4}). Now simplify both sides using the fact that $\vv$ and $\ww$ are eigenvectors and see what you obtain.}
\medskip
% 
\end{enumerate}
\end{prob} \begin{prob} \label{prob23.3}\sov Let $A=\bmatrix
0&1&1\\ 1&0&1\\ 1&1&0 \endbmatrix$. 

\begin{enumerate}[a)]

\item Compute $\det(A-\lam I_3)$ and hence show that the eigenvalues of
$A$ are $2$ and $-1$.

\item Find a basis of $E_2 =\set{\xx\in \R^3 \st A\xx= 2\xx}$.
\item Find a basis of $E_{-1} =\set{\xx\in \R^3 \st A\xx=-\xx}$. 
 
\item Find an invertible  matrix 
$P$ such that $P^{-1}AP=D$ is diagonal,  and give this diagonal matrix $D$. Explain why
your choice of $P$ is invertible.
\item Find an invertible  matrix 
$Q \not=P$ such that $Q^{-1}AQ=\tilde D$ is also diagonal,  and give this diagonal matrix $\tilde D$.
\end{enumerate}
  


\end{prob} \begin{prob} \label{prob23.4} $^{\ast}$\footnote{ This example is a simplified version of the equations of motion for 2 masses connected by springs. Search the web for `Normal mode' for an example. } Consider the coupled  system of second order differential equations for the the two functions $f$ and $g$:

$$\begin{matrix} 
\ddot f=-2 f +g  \\
\ddot g=f-2g\\
 \end{matrix} $$(Here $\ddot f$ and $\ddot g$ denote $\dfrac{d^2 f}{dt^2}$ and $\dfrac{d^2 g}{dt^2}$ respectively.)
This can be also written in matrix form as: $\bmatrix \ddot f \,\\ \ddot g
 \endbmatrix =\bmatrix -2 & 1 \\
 1 & -2 \endbmatrix \bmatrix  f \,\\   g
 \endbmatrix$. 
Let $A=\bmatrix -2 & 1 \\
 1 & -2 \endbmatrix $. Diagonalize $A$ to  write $A=PDP^{-1}$ for some invertible matix $P$ and some diagonal matrix $D=\bmatrix \lam_1 & 0 \\
 0 & \lam_2 \endbmatrix $. Now define two new functions $h$ and $k$ by  $\bmatrix h \,\\ k
 \endbmatrix = P^{-1} \bmatrix  f \,\\   g
 \endbmatrix$. Show that $\bmatrix \ddot f \,\\ \ddot g
 \endbmatrix =\bmatrix -2 & 1 \\
 1 & -2 \endbmatrix \bmatrix  f \,\\   g
 \endbmatrix$ is equivalent to $\bmatrix \ddot h\,\\ \ddot k
 \endbmatrix =\bmatrix \lam_1 & 0 \\
 0 & \lam_2 \endbmatrix\bmatrix  h \,\\   k
 \endbmatrix$, or, written out fully as
$$\begin{matrix} 
\ddot h=\lam_1 \,h  \\
\ddot k=\lam_2 \,k\\
 \end{matrix} $$ 

  You will find that both $\lam_1$ and $\lam_2$ are negative, so the solutions to these two (linear) second order differential equations\footnote{ Search the web  for `Linear differential equation' for some details.} are $h(t)=a \sin( \sqrt{|\lam_1|}\,t) + b \cos( \sqrt{|\lam_1|}\,t)$ and $k(t)=c \sin( \sqrt{|\lam_2|}\,t) + c \cos( \sqrt{|\lam_2}|\,t)$, where $a,b,c,d$ are real constants.

\medskip
  Use this to find $f$ and $g$, and solve for $a,b,c$ and $d$  in terms of  the so-called {\it `initial conditions' $f(0), \dot f(0), g(0), \dot g(0)$}.
\medskip
%


\end{prob}

