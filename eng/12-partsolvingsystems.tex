%%%%%%%%%%%%%%%%%%%%%part.tex%%%%%%%%%%%%%%%%%%%%%%%%%%%%%%%%%%
% 
% sample part title
%
% Use this file as a template for your own input.
%
%%%%%%%%%%%%%%%%%%%%%%%% Springer %%%%%%%%%%%%%%%%%%%%%%%%%%

\begin{partbacktext}
\part{Solving Linear Systems}
\noindent 
We have seen how to solve a number of important problems, at least theoretically; but the calculations needed to get to the solution --- solving linear systems --- seems overwhelming.  What we need is a practical, efficient, reliable means to solve these linear systems, one that will minimize the calculations we are required to carry out and the amount of writing we need to do; one that will allow us to deduce not just whether or not there is a solution, but also if there are lots of solutions (\emph{e.g.} nontrivial solutions).  

This is the goal and purpose of Gauss-Jordan elimination (commonly called: row reduction).

\end{partbacktext}
