%%%%%%%%%%%%%%%%%%%%%part.tex%%%%%%%%%%%%%%%%%%%%%%%%%%%%%%%%%%
% 
% sample part title
%
% Use this file as a template for your own input.
%
%%%%%%%%%%%%%%%%%%%%%%%% Springer %%%%%%%%%%%%%%%%%%%%%%%%%%

\begin{partbacktext}
\part{Linear Transformations}
\noindent In mathematics, one might begin by studying the objects under consideration --- in this case, vector spaces --- but then equally important is to understand the maps between them, that is, the maps that are ``compatible'' with the vector space structure.  These maps are called \stress{linear transformations}.  We have seen and used several already: the coordinate map, the projection map.

\end{partbacktext}
